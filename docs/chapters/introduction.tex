\chapter{PENDAHULUAN}

\section{Latar Belakang}

% Avionik Sistem penerbangan atau avionik adalah sistem yang terdiri dari perangkat elektronik yang digunakan pada pesawat terbang atau satelit.
Avionik berfungsi untuk memberikan fasilitas untuk mengoperasikan pesawat terbang.
Sistem yang termasuk dalam avionik adalah sistem komunikasi, navigasi, \textit{dashboard} untuk manajemen sistem, dan sistem-sistem yang dipasang pada pesawat terbang untuk melakukan fungsionalitas tersendiri.


%  Design-issues pada avionik Avionik merupakan komponen penting dalam sebuah pesawat terbang dan merupakan \textit{safety\hyp critical system}, yaitu sistem yang apabila mengalami kegagalan dapat mengakibatkan kematian, kerugian besar, atau kerusakan lingkungan.
Sistem operasi yang digunakan oleh avionik pada sebuah pesawat umumnya merupakan sistem \textit{real-time}, yaitu sistem yang kebenaran komputasinya tidak hanya ditentukan oleh kebenaran hasil komputasi secara logika, tetapi juga waktu dimana hasil komputasi dapat digunakan \citep[p.~6]{Shin1994}.
Maka, manajemen waktu sangat mendasar pada sistem \textit{real-time}.
Perangkat yang digunakan untuk avionik memiliki beberapa standar yang wajib dipenuhi.
Standar tersebut dibuat oleh Aeronautical Radio, Incorporated (ARINC) dalam sebuah seri standar ARINC 600.
Buku ini akan mendalami lebih lanjut mengenai sistem operasi \textit{real-time} pada avionik yang dispesifikasikan pada penerbangan bernama ARINC 653.


% ARINC 653 Standar ARINC 653 menspesifikasikan partisi ruang dan waktu pada sistem berbasis \textit{real-time} agar avionik yang menggunakan sistem tersebut \textit{safety-critical}.
Setiap proses aplikasi disebut sebagai \textbf{partisi}.
Selain spesifikasi \textbf{partisi}, ARINC 653 juga mendefinisikan API untuk memudahkan manajemen \textbf{partisi}.
Sistem operasi yang memenuhi standar ARINC 653 berfungsi untuk memanajemeni beberapa \textbf{partisi} beserta komunikasi antar \textbf{partisi}.
Manajemen yang dilakukan oleh meliputi \textit{resources} komputasi umum seperti CPU, \textit{timing}, memori, dan \textit{devices} serta \textit{resources} untuk komunikasi.
Buku ini akan mendalami lebih lanjut mengenai manajemen CPU, khususnya proses \textit{scheduling}, yang dispesifikasikan oleh standar ARINC 653.


% Hypervisor Arsitektur komputer yang dispesifikasikan pada standar ARINC 653 dapat disimulasikan dengan menggunakan \textit{hypervisor}.
\textit{Hypervisor} adalah sebuah komponen yang membuat dan menjalankan \textit{virtual machine}.
Setiap \textit{virtual machine} akan memiliki ruang memori dan waktu tersendiri sehingga dapat digunakan untuk mengimplementasikan sebuah \textbf{partisi}.
Pada saat penulisan, terdapat sebuah prototipe sistem ARINC 653 yang menggunakan \textit{hypervisor} yang dinamakan ARLX.
Prototipe sistem tersebut dibangun diatas Xen, yaitu \textit{hypervisor} \textit{open-source} dengan lisensi GPL sehingga \textit{hypervisor} ARINC 653 tersebut juga \textit{open-source}.
Solusi open-source akan mempermudah pengembang aplikasi dan peneliti untuk mencoba lingkungan ARINC 653.
Namun, sistem tersebut relatif baru dan tidak dapat dijamin kestabilannya karena pengujian yang tersedia hanya berfungsi untuk menguji Xen dan tidak dapat digunakan untuk menguji ARLX.
Hal ini mengakibatkan sulitnya pengambilan hasil eksperimen pada ARLX.
Karena itu, perlu adanya pengujian khusus untuk ARLX yang menjamin lingkungan ARINC 653 yang ditawarkan sesuai dengan yang lingkungan 653 yang seharusnya.


% Purpose Buku ini akan mendalami proses \textit{scheduling} ARINC 653 dengan cara melakukan eksperimen pada ARLX.
Eksperimen akan dilakukan dengan melakukan perbandingan hasil pengujian apabila menggunakan proses \textit{scheduling} normal pada ARLX dengan proses \textit{scheduling} eksperimen.


\section{Rumusan Masalah}

% TODO: Fix this
Perumusan masalah dari uraian latar belakang adalah perlunya pengujian sistem ARLX guna menjamin lingkungan ARINC 653 yang ditawarkan oleh ARLX agar sesuai dengan lingkungan ARINC 653 yang seharusnya.


\section{Tujuan}

% TODO: Fix this
Berdasarkan rumusan masalah yang ada, tujuan yang ingin dicapai adalah menjamin lingkungan ARINC 653 yang ditawarkan oleh ARLX sesuai dengan lingkungan ARINC 653 yang seharusnya.


\section{Batasan Masalah}
\label{section:batasan_masalah}

Berdasarkan rumusan masalah yang ada, terdapat beberapa asumsi terkait dengan sistem avionik yang akan dikembangkan yaitu:

\begin{itemize}

    \item \textit{Hypervisor} yang digunakan adalah Xen.  Xen merupakan \textit{hypervisor} tipe-1, sehingga mungkin berbeda dengan \textit{hypervisor} tipe-2 karena \textit{hypervisor} tipe-1 berjalan langsung di atas perangkat keras sedangkan \textit{hypervisor} tipe-2 berjalan di atas sistem lain.
        Perbedaan ini mengakibatkan sistem yang dibangun di atas \textit{hypervisor} tipe-1 tidak dapat berjalan pada \textit{hypervisor} tipe-2.
        Penggunaan \textit{hypervisor} tipe-1 lainnya mungkin akan memberikan perilaku yang berbeda dengan perilaku yang terjadi pada Xen.


    \item Pengujian dan eksperimen yang dilakukan hanya akan diuji coba dengan menggunakan \textit{kernel} Linux untuk setiap \textbf{partisi}-nya.
        Hal ini mengakibatkan \textbf{partisi} yang tidak menggunakan \textit{kernel} Linux mungkin memiliki perilaku yang berbeda.


\end{itemize}

\section{Metodologi}

Metodologi yang akan digunakan dalam penilitian tugas akhir adalah sebagai berikut:

\begin{enumerate}

    \item Studi Literatur

        Pengerjaan tugas akhir diawali dengan mempelajari referensi berupa jurnal ilmiah dan dokumen resmi terkait dengan spesifikasi ARINC 653, dan arsitektur Xen dan ARLX.
        Pembelajaran mengenai spesifikasi ARINC 653 dilakukan untuk dapat mengetahui kriteria penilaian yang akan digunakan sebagai patokan dalam pembuatan kerangka pengujian.
        Pembelajaran arsitektur Xen dan ARLX dilakukan untuk mengetahui cara memodifikasi ARLX untuk mencari tahu bagaimana eksperimen pada ARLX akan dilakukan.


    \item Analisis

        Pada tahap ini dilakukan analisis permasalahan yang berkaitan dengan topik tugas akhir ini.
        Analisis permasalahan akan membahas spesifikasi ARINC 653, arsitektur Xen dan ARLX untuk kemudian dijadikan sebagai solusi terhadap permasalahan.


    \item Perancangan Solusi dan Pengujian

        Pada tahap ini dilakukan perancangan pengujian serta eksperimen yang akan dilakukan yang dapat menyelesaikan masalah\hyp{}masalah yang telah dijelaskan pada \autoref{section:batasan_masalah}.



    \item Implementasi

        Pada tahap ini dilakukan pembangunan sistem pengujian serta implementasi modul eksperimen pada ARLX.


    \item Pengujian dan Analisis Hasil

        Pada tahap ini dilakukan pengujian dengan metodologi pengujian yang dibangun dari tahap pembangunan solusi dan pengujian.
        Selanjutnya dilakukan pengujian hasil pengujian dan penarikan kesimpulan.


\end{enumerate}

\section{Jadwal Pelaksanaan Tugas Akhir}

Pengerjaan tugas akhir direncanakan akan dimulai dari September 2016 sampai dengan April 2016.
Pelaksanaan tugas akhir dibagi menjadi 5 tahap untuk masing-masing metodologi pengerjaan sebagai berikut:

\begin{enumerate}

    \item Tahap 1: Studi Literatur
    \item Tahap 2: Analisis Masalah
    \item Tahap 3: Perancangan Solusi
    \item Tahap 4: Implementasi
    \item Tahap 5: Pengujian dan Analisis Hasil

\end{enumerate}

Jadwal pelaksanaan tugas akhir berdasarkan metodologi pengerjaan tugas akhir dapat dilihat pada
\autoref{figure:gantt_chart_jadwal}.


\begin{figure}[htbp]
    \begin{center}
        \begin{ganttchart}[
                vgrid={*{12}{draw=none}, dotted},
                hgrid,
                x unit=.05cm,
                y unit title=.6cm,
                y unit chart=.6cm,
                time slot format=isodate,
                bar/.append style={fill=black},
            time slot format/start date=2016-09-01]{2016-09-01}{2017-04-30}
            \ganttset{bar height=.6}
            \gantttitlecalendar{year, month} \\
            \ganttbar{Tahap 1}{2016-09-05}{2016-11-25}\\
            \ganttbar{Tahap 2}{2016-10-05}{2016-11-20}\\
            \ganttbar{Tahap 3}{2016-11-05}{2016-12-20}\\
            \ganttbar{Tahap 4}{2016-12-15}{2017-03-05}\\
            \ganttbar{Tahap 5}{2017-02-05}{2017-04-30}
        \end{ganttchart}
    \end{center}
    \caption{Gantt Chart jadwal pelaksanaan tugas akhir}
    \label{figure:gantt_chart_jadwal}
\end{figure}

% vim: wrap
