\chapter{PENDAHULUAN}

\section{Latar Belakang}

% Avionik
Sistem penerbangan atau avionik adalah sistem yang terdiri dari perangkat elektronik yang digunakan pada pesawat terbang
atau satelit. Avionik berfungsi untuk memberikan fasilitas untuk mengoperasikan pesawat terbang. Sistem yang termasuk
dalam avionik adalah sistem komunikasi, navigasi, \textit{dashboard} untuk manajemen sistem, dan sistem-sistem yang
dipasang pada pesawat terbang untuk melakukan fungsionalitas tersendiri.

%  Design-issues pada avionik
Avionik merupakan komponen penting dalam sebuah pesawat terbang dan merupakan \textit{safety-critical system}, yaitu
sistem yang apabila mengalami kegagalan dapat mengakibatkan kematian, kerugian besar, atau kerusakan lingkungan. Sistem
operasi yang digunakan oleh avionik pada sebuah pesawat umumnya merupakan sistem \textit{real-time}, yaitu sistem
yang kebenaran komputasinya tidak hanya ditentukan oleh kebenaran hasil komputasi, tetapi juga waktu dimana hasil
komputasi dapat digunakan. % TODO: Cite Shin, K.G for RTC
Maka, manajemen waktu sangat mendasar pada sistem \textit{real-time}. Perangkat yang digunakan untuk avionik memiliki
beberapa standar yang wajib dipenuhi.  Standar tersebut dibuat oleh Aeronautical Radio, Incorporated (ARINC) dalam
sebuah seri standar ARINC 600. Buku ini akan mendalami lebih lanjut mengenai sistem operasi \textit{real-time} pada
avionik yang dispesifikasikan pada penerbangan bernama ARINC 653.

% ARINC 653
Standar ARINC 653 menspesifikasikan partisi ruang dan waktu pada sistem berbasis \textit{real-time} agar avionik yang
menggunakan sistem tersebut \textit{safety-critical}. Setiap proses perangkat lunak disebut sebagai \textbf{partisi}.
Selain spesifikasi \textbf{partisi}, ARINC 653 juga mendefinisikan API untuk memudahkan manajemen \textbf{partisi}.
Sistem operasi yang memenuhi standar ARINC 653 berfungsi untuk memanajemeni beberapa \textbf{partisi} beserta
komunikasi antar \textbf{partisi}. Manajemen yang dilakukan oleh meliputi \textit{resources} komputasi umum seperti CPU,
\textit{timing}, memori, dan \textit{devices} serta \textit{resources} untuk komunikasi. Buku ini akan mendalami lebih
lanjut mengenai manajemen CPU, khususnya proses \textit{scheduling}, yang dispesifikasikan oleh standar ARINC 653.

% Hypervisor
Arsitektur komputer yang dispesifikasikan pada standar ARINC 653 dapat disimulasikan dengan menggunakan
\textit{hypervisor}. \textit{Hypervisor} adalah sebuah komponen yang membuat dan menjalankan \textit{virtual machine}.
Setiap \textit{virtual machine} akan memiliki ruang memori dan waktu tersendiri sehingga dapat digunakan untuk
mengimplementasikan sebuah \textbf{partisi}. Pada saat penulisan, terdapat beberapa \textit{hypervisor} yang memenuhi
standar ARINC 653.  Namun, \textit{hypervisor} ARINC 653 yang sudah matang relatif sangat mahal sehingga hanya
pihak-pihak tertentu saja yang mampu mendapatkannya. Hal ini mengakibatkan eksplorasi pengembangan aplikasi, pengujian,
dan jaminan desain pada \textit{hypervisor} tersebut sulit untuk dilakukan tanpa investasi yang besar. Terdapat sebuah
prototipe \textit{hypervisor} ARINC 653 yang dinamakan ARLX. Prototipe dibangun diatas Xen, yaitu \textit{hypervisor}
\textit{open-source} dengan lisensi GPL sehingga \textit{hypervisor} ARINC 653 tersebut juga \textit{open-source}.
Namun, \textit{hypervisor} tersebut relatif baru dan tidak dapat dijamin kestabilannya.

% Purpose
Buku ini akan mendalami proses \textit{scheduling} ARINC 653 dengan melakukan eksperimen pada ARLX. Eksperimen akan
dilakukan dengan melakukan perbandingan \textit{benchmark} antara \textit{scheduling} normal pada ARLX dengan
\textit{scheduling} percobaan.

\section{Rumusan Masalah}

% TODO: Fix this
Perumusan masalah dari uraian latar belakang adalah tidak adanya implementasi \textit{hypervisor} yang memenuhi kriteria
pada spesifikasi ARINC 653 yang bersifat \textit{free} dan \textit{open source}.

\section{Tujuan}

% TODO: Fix this
Berdasarkan rumusan masalah yang ada, tujuan yang ingin dicapai adalah melakukan implementasi \textit{API} yang
yang dispesifikasikan pada standar ARINC 653 pada \textit{hypervisor} yang bersifat \textit{free} dan \textit{open
source}.

\section{Batasan Masalah}
\label{section:batasan_masalah}

Berdasarkan rumusan masalah yang ada, terdapat beberapa asumsi terkait dengan sistem avionik yang akan
dikembangkan yaitu:

\begin{itemize}

    \item \textit{Hypervisor} yang digunakan adalah \textit{hypervisor} tipe-1. \textit{Hypervisor} tipe-1 berbeda
        dengan tipe-2 karena \textit{hypervisor} tipe-1 berjalan di atas perangkat lunak sedangkan \textit{hypervisor}
        tipe-2 berjalan di atas sistem lain. Perbedaan ini mengakibatkan sistem yang dibangun di atas
        \textit{hypervisor} tipe-1 tidak dapat berjalan pada \textit{hypervisor} tipe-2.

    \item \textit{API} yang disediakan hanya akan diuji coba pada \textit{kernel} Linux. Hal ini mengakibatkan
        \textbf{partisi} yang dijalankan tanpa menggunakan \textit{kernel} Linux akan memiliki perilaku yang tidak
        terdefinisi.

\end{itemize}

\section{Metodologi}

Metodologi yang akan digunakan dalam penilitian tugas akhir adalah sebagai berikut:

\begin{enumerate}

    \item Studi Literatur

        Pengerjaan tugas akhir diawali dengan mempelajari referensi berupa jurnal ilmiah beserta sistem yang sudah
        pernah dibuat sebelumnya yang dapat membantu pengembangan sistem yang dibuat pada tugas akhir ini.  Literatur
        yang dicari dan dipelajari berkaitan dengan topik tugas akhir yaitu avionik, sistem berbasis \textit{real-time},
        kesalahan yang pernah ditemukan sebelumnya, metode pengujian, metode penjaminan desain sistem, serta hal-hal
        lain yang masih berkaitan dengan topik tugas akhir ini.

    \item Analisis

        Pada tahap ini dilakukan analisis permasalahan yang berkaitan dengan topik tugas akhir ini. Analisis
        permasalahan meliputi penentuan spesifikasi dan arsitektur yang ada pada \textit{hypervisor} untuk kemudian
        dijadikan sebagai solusi terhadap permasalahan.

    \item Perancangan Solusi dan Pengujian

        Pada tahap ini dilakukan perancangan sistem yang dapat menyelesaikan masalah\hyp{}masalah yang telah dijelaskan
        pada \autoref{section:batasan_masalah} beserta \textit{tools} untuk pengujian sistem. Bagian perancangan ini
        juga menjelaskan arsitektur yang digunakan untuk membangun sistem berdasarkan spesifikasi dan metode yang
        digunakan.


    \item Implementasi

        Pada tahap ini dilakukan pembangunan sistem sesuai dengan hasil perancangan solusi. Selain sistem, pada tahap
        ini juga dilakukan pembangunan \textit{tools} untuk pengujian sistem.

    \item Pengujian dan Analisis Hasil

        Pada tahap ini dilakukan pengujian dengan metodologi pengujian yang dibangun dari tahap pembangunan solusi dan
        pengujian. Selanjutnya dilakukan pengujian hasil pengujian dan penarikan kesimpulan.

\end{enumerate}

\section{Jadwal Pelaksanaan Tugas Akhir}

\newsavebox\mybox

\begin{lrbox}{\mybox}
    \begin{ganttchart}[
            vgrid={*{6}{draw=none}, dotted},
            x unit=.05cm,
            y unit title=.6cm,
            y unit chart=.6cm,
            time slot format=isodate,
        time slot format/start date=2016-09-01]{2016-09-01}{2017-04-30}
        \ganttset{bar height=.6}
        \gantttitlecalendar{year, month} \\
        \ganttbar[bar/.append style={fill=black}]{Tahap 1}{2016-09-05}{2016-11-25}\\
        \ganttbar[bar/.append style={fill=black}]{Tahap 2}{2016-10-05}{2016-11-20}\\
        \ganttbar[bar/.append style={fill=black}]{Tahap 3}{2016-11-05}{2016-12-20}\\
        \ganttbar[bar/.append style={fill=black}]{Tahap 4}{2016-12-15}{2017-03-05}\\
        \ganttbar[bar/.append style={fill=black}]{Tahap 5}{2017-02-05}{2017-04-30}
    \end{ganttchart}
\end{lrbox}

Pengerjaan tugas akhir direncanakan akan dimulai dari September 2016 sampai dengan April 2016. Pelaksanaan tugas akhir
dibagi menjadi 5 tahap untuk masing-masing metodologi pengerjaan sebagai berikut,

\begin{enumerate}

    \item Tahap 1: Studi Literatur
    \item Tahap 2: Analisis Masalah
    \item Tahap 3: Perancangan Solusi
    \item Tahap 4: Implementasi
    \item Tahap 5: Pengujian dan Analisis Hasil

\end{enumerate}

Jadwal pelaksanaan tugas akhir berdasarkan metodologi pengerjaan tugas akhir dapat dilihat pada
\autoref{table:gantt_chart_jadwal}.

\begin{table}[h]
    \centering
    \tikz{
        \node[inner sep=0pt,outer sep=0pt] (gantt)
        {
            \begin{tabular}{c}
                \toprule
                \resizebox{\textwidth}{!}{\usebox\mybox}\\
                \bottomrule
            \end{tabular}
        };
    }
    \caption{Gantt Chart jadwal pelaksanaan tugas akhir}
    \label{table:gantt_chart_jadwal}
\end{table}

% vim: tw=120
