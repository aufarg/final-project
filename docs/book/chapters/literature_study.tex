%! TEX root = ../final-project.tex
\chapter{STUDI LITERATUR}

\section{ARINC 653}

Standar ARINC 653 menspesifikasikan partisi ruang dan waktu pada sistem operasi
\textit{real-time} avionik.  Avionik adalah sebuah perangkat elektronik yang berfungsi untuk
memberikan fasilitas untuk mengoperasikan pesawat terbang.  Avionik umumnya merupakan sistem
\textit{safety\hyp critical}, yang berarti kegagalan pada sistem dapat berakibat terjadinya
hal-hal berikut.

\begin{enumerate}

    \item Kematian atau cedera serius pada manusia.

    \item Hilangnya atau rusaknya perlengkapan atau properti.

    \item Kerusakan lingkungan.

\end{enumerate}

Pada avionik, \textit{Integrated Modular Avionics} (IMA) merupakan sebuah konsep yang bertujuan
memperkecil penggunaan daya dan meringankan perangkat \citep[p.~2.A.2-1]{Garside2009}.  Dalam
arsitektur IMA, fungsionalitas avionik yang diharapkan menggunakan \textit{resources} yang sama.
Arsitektur IMA melakukan optimasi dengan cara memberikan \textit{resource} seminimum mungkin
pada masing-masing fungsionalitas avionik.  Arsitektur ini memungkinkan eksekusi beberapa
aplikasi avionik pada sebuah komputer yang sama.  Hal tersebut dapat dicapai apabila sistem
melakukan mekanisme partisi aplikasi avionik untuk mengisolasi eksekusi antar aplikasi, sehingga
menjamin aplikasi dari satu partisi tidak dapat mengubah aplikasi, \textit{devices}, aktuator,
maupun data pada partisi lain \citep[pp.~11-12]{Rushby2000}.  Pada IMA, keseluruhan sistem
disebut sebagai \textit{integrated module} yang terdiri dari \textit{core module} dan aplikasi.
Sebuah \textit{core module} terdiri atas \textit{core software} dan \textit{core hardware} yang
menyediakan fungsionalitas \textit{core module} dari sisi perangkat lunak dan perangkat keras.
\textit{Core module} berfungsi untuk menjadi tempat eksekusi aplikasi dan menyediakan layanan
untuk melakukan partisi.  Pada buku ini, istilah \textit{integrated module} dan \textit{core
software} dapat dipertukarkan dengan \textit{module} dan sistem operasi.

Standar ARINC 653 bertujuan untuk mengurangi biaya pengembangan, ukuran sistem, dan biaya
sertifikasi dengan cara menggabungkan beberapa aplikasi pada satu komputer namun tetap menjaga
aplikasi-aplikasi tersebut saling terisolasi
\citep[pp.~3-30]{AirlinesElectronicEngineeringCommittee2012}.  Untuk mencapai tujuan tersebut,
ARINC 653 mendefinisikan \textit{interface} yang disebut sebagai \textit{Application Executive}
(APEX) untuk manajemen partisi, proses dan \textit{timing}, komunikasi antar proses/partisi, dan
penanganan \textit{error} pada arsitektur IMA.  Pada ARINC 653, satu unit partisi aplikasi
disebut sebagai partisi.

Setiap partisi pada dasarnya sama seperti sistem dengan satu aplikasi karena partisi memiliki
\textit{device}, aktuator, konteks, dan data tersendiri.  Hal tersebut mengakibatkan sistem
memiliki partisi ruang dan waktu, sehingga masing-masing partisi tidak akan mempengaruhi partisi
lain pada sistem.  Proses dalam sebuah partisi diperbolehkan untuk melakukan
\textit{multitasking}.

Perangkat lunak pada \textit{platform} ARINC 653 meliputi.

\begin{enumerate}

    \item Partisi aplikasi avionik yang dispesifikasikan secara langsung oleh ARINC 653. Partisi
    	    aplikasi adalah target utama dari partisi ruang dan waktu dan dibatasi hanya
    	    menggunakan \textit{system-call} yang telah disediakan oleh ARINC 653.

    \item Sebuah \textit{kernel} sistem operasi yang menyediakan \textit{interface} dan perilaku
    	    yang telah dispesifikasikan oleh standar ARINC 653. \textit{Kernel} harus
    	    menyediakan lingkungan eksekusi standar untuk menjalankan aplikasi.

    \item Partisi sistem untuk melakukan fungsi di luar lingkup APEX. Fungsi yang dilakukan
    	    partisi sistem meliputi manajemen komunikasi antar \textit{device} perangkat lunak
    	    maupun manajemen \textit{fault}.

    \item Fungsi spesifik sistem seperti \textit{device driver}, \textit{debug}, dan
    	    \textit{test}.

\end{enumerate}

\subsection{Sistem \textit{Real-Time}}
\label{section:sistem_realtime}

Sistem \textit{real-time} adalah sistem yang kebenaran perhitungannya tidak hanya ditentukan
oleh kebenaran hasil perhitungan secara logika, tetapi juga waktu pada saat hasil perhitungan
dapat digunakan \citep[pp.~6-7]{Shin1994}.  Sistem \textit{real-time} digunakan untuk mendukung
aplikasi yang membutuhkan perhitungan \textit{real-time}.  Aplikasi yang membutuhkan perhitungan
\textit{real-time} umumnya terdiri atas beberapa \textit{task} yang saling terkait.  Sebuah
\textit{task} dapat berupa periodik dan aperiodik. \textit{Task} yang bersifat periodik harus
dikerjakan dalam interval reguler dan memiliki \textit{deadline} yang berarti \textit{task}
tersebut harus sudah selesai pada waktu yang telah ditentukan. \textit{Task} yang bersifat
aperiodik tidak memiliki \textit{interval} reguler, namun tetap memiliki \textit{deadline} yang
harus dipenuhi.

Pada umumnya, \textit{task} periodik merupakan \textit{time-critical task} yang harus dijamin
dapat diselesaikan sesuai dengan \textit{deadline}-nya dan kegagalan dalam menyelesaikan
\textit{task} tersebut dapat mengakibatkan kegagalan sistem secara keseluruhan.  Sebagai contoh,
pada aplikasi kendali pesawat, sebuah \textit{task} periodik mungkin melakukan pengaturan
kekuatan mesin dengan cara menghitung jumlah bahan bakar yang harus diberikan dan memberikannya
sejumlah hasil perhitungan pada mesin.  Kegagalan sistem operasi menyelesaikan \textit{task}
tersebut dalam \textit{deadline} yang telah ditentukan dapat mengakibatkan jatuhnya pesawat dan
hilangnya nyawa manusia.  Maka, sistem operasi \textit{real-time} pada aplikasi kendali pesawat
harus dapat memberikan jaminan bahwa \textit{task} periodik aplikasi tersebut dapat diselesaikan
sesuai dengan \textit{deadline}-nya.

Tidak semua \textit{task} harus dikerjakan dalam interval reguler. Beberapa \textit{task}
dikerjakan apabila suatu kejadian terjadi. \textit{Task} demikian disebut dengan \textit{task}
aperiodik.  Sebagai contoh, pintu pesawat dibuka dengan menekan tombol untuk membuka pintu
pesawat.  \textit{Task} tersebut tidak memiliki \textit{deadline} penyelesaian.  Namun,
\textit{task} aperiodik juga dapat memiliki \textit{deadline} seperti \textit{ task} periodik.
Maka, \textit{task} aperiodik demikian harus diselesaikan sebelum \textit{deadline}-nya.

Berdasarkan penjelasan di atas, \textit{deadline} dari sebuah sistem \textit{real-time} dapat
diklasifikasikan menjadi \textit{hard}, \textit{firm}, dan \textit{soft}.  Sebuah
\textit{deadline} dikatakan \textit{hard} apabila konsekuensi kegagalan penyelesaiannya fatal.
Sebuah \textit{deadline} dikatakan \textit{firm} dan \textit{soft} apabila konsekuensi kegagalan
penyelesaiannya tidak fatal, namun akan menurunkan kualitas layanan yang ditawarkan sistem.
Perbedaan antara \textit{soft} dan \textit{firm} adalah apabila hasil komputasi masih dapat
digunakan setelah \textit{deadline}, maka \textit{deadline} tersebut merupakan \textit{soft
deadline}. Sebaliknya, apabila hasil komputasi tidak dapat digunakan setelah \textit{deadline},
maka \textit{deadline} tersebut merupakan \textit{firm deadline}.  Pada umumnya, \textit{task}
periodik memiliki \textit{hard deadline}, sedangkan kebanyakan jenis \textit{task} aperiodik
yang memiliki deadline termasuk dalam kategori \textit{firm deadline}, seperti transaksi pada
sistem basis data \citep[pp.~203-241]{Haritsa1992}.

Penentuan \textit{deadline} suatu aplikasi datang dari aplikasi itu sendiri.  Sebagai contoh,
pada sistem pengaturan kekuatan mesin, kekuatan mesin harus dapat diregulasi setiap
\SI{50}{\milli\second}.  \textit{Deadline} tersebut menjatuhkan \textit{deadline} pada
\textit{subtask} pengaturan kekuatan mesin seperti perhitungan jumlah bahan bakar yang
dibutuhkan dan pemberian bahan bakar.  Sebagai contoh, perhitungan jumlah bahan bakar harus
diselesaikan dalam waktu \SI{5}{\milli\second} dan pemberian bahan bakar harus diselesaikan
dalam waktu \SI{20}{\milli\second}.  \textit{Deadline} tersebut akan menjatuhkan
\textit{deadline} pada \textit{subtask} masing-masing, dan seterusnya.

Dari gambaran di atas, oleh karena pengaturan kekuatan mesin harus dapat diselesaikan tanpa
terkecuali, terlihat bahwa sebuah sistem \textit{real-time} harus dapat diprediksi.  Sebuah
sistem \textit{real-time} dikatakan dapat diprediksi apabila pada saat desain dapat ditunjukkan
bahwa seluruh \textit{deadline} \textit{task} aplikasi dapat dipenuhi selama beberapa asumsi
mengenai sistem tersebut terpenuhi.  Sebagai contoh, apabila asumsi mengenai sistem adalah
\textit{disk fault} hanya terjadi maksimal sebanyak $f$ selama $t$ detik, maka sistem tersebut
dapat  ditunjukkan pada saat desain bahwa seluruh \textit{deadline} \textit{task} aplikasi dapat
dipenuhi selama \textit{disk fault} yang terjadi tidak lebih dari $f$ selama $t$ detik.  Asumsi
tersebut mengakibatkan penentuan sebuah sistem dapat diprediksi ditentukan oleh aplikasi yang
dijalankan pada sistem tersebut.

Pada sistem operasi Linux, kapabilitas \textit{real-time} dapat dicapai dengan menggunakan
\textit{patch} \texttt{PREEMPT\_RT} \citep{RTLinux}. \textit{Patch} tersebut akan meminimalisasi
bagian dari \textit{kernel} yang tidak \textit{preemptible} serta menambah kebijakan mengenai
prioritas pada masing-masing \textit{process} sehingga \textit{process} dengan prioritas yang
lebih tinggi dapat mengambil alih giliran ekseskusi dari \textit{process} dengan prioritas yang
lebih rendah.

\subsection{\textit{Real-Time Scheduling}}

Diberikan beberapa \textit{real-time task} dan \textit{resources} yang terdapat pada sistem,
\textit{real-time scheduling} adalah proses penentuan kapan dan dimana \textit{task} tersebut
akan dikerjakan untuk menjamin seluruh \textit{task} memenuhi \textit{deadline}
\citep[pp.~8-9]{Shin1994}.  \textit{Scheduler} adalah sebuah proses yang melakukan
\textit{scheduling} pada \textit{real-time task} yang diberikan.  Dalam buku ini, istilah
\textit{real-time scheduling} dan \textit{real-time scheduler} secara berurutan dapat
dipertukarkan dengan \textit{scheduling} dan \textit{scheduler}.  Sistem operasi akan secara
periodik memberikan sejumlah \textit{real-time task} yang tersimpan pada memori pada
\textit{scheduler}.  Pada aplikasi \textit{non-real-time}, tujuan utama \textit{scheduling}
adalah untuk meminimumkan total waktu yang dibutuhkan untuk menyelesaikan semua \textit{task}.
Sementara itu, pada aplikasi \textit{real-time}, tujuan utama \textit{scheduling} adalah untuk
memenuhi \textit{deadline} \textit{task} aplikasi.

Sebagai contoh, \autoref{figure:scheduling_comparison}\subref{subfigure:scheduling_comparison_a}
dan \subref{subfigure:scheduling_comparison_b} memperlihatkan perbedaan antara
\textit{scheduling} pada sistem \textit{real-time} dengan \textit{scheduling} pada sistem
\textit{non-real-time}.  \textit{Task} A membutuhkan waktu \SI{6}{\micro\second}, \textit{task}
B membutuhkan waktu \SI{9}{\micro\second}, \textit{task} C membutuhkan waktu
\SI{4}{\micro\second}, dan \textit{task} D membutuhkan waktu \SI{11}{\micro\second}.  Seluruh
\textit{task} tersebut akan dijalankan pada sistem dengan dua buah CPU.  Apabila sistem
merupakan sistem \textit{non-real-time}, maka \textit{scheduling} yang terbaik dapat dicapai
dengan pemilihan urutan proses seperti
\autoref{figure:scheduling_comparison}\subref{subfigure:scheduling_comparison_a}.
\textit{Scheduling} seperti pada
\autoref{figure:scheduling_comparison}\subref{subfigure:scheduling_comparison_b} juga dapat
digunakan, namun tidak optimal.  Namun, apabila sistem merupakan sistem \textit{real-time}, dan
\textit{task} B dan D mempunyai \textit{deadline} \SI{14}{\micro\second} dan
\SI{17}{\micro\second}, maka satu-satunya \textit{scheduling} yang memenuhi adalah seperti pada
\autoref{figure:scheduling_comparison}\subref{subfigure:scheduling_comparison_b}.

\begin{figure}[htbp]
    \centering
    \vspace{18pt}
    \subfloat[]{\label{subfigure:scheduling_comparison_a}\includegraphics[scale=0.6]{resources/scheduling-normal.png}}\\
    \vspace{18pt}\hspace{24pt}
    \subfloat[]{\label{subfigure:scheduling_comparison_b}\includegraphics[scale=0.6]{resources/scheduling-realtime.png}}
    \caption[Ilustrasi perbedaan permasalahan \textit{scheduling} pada aplikasi \textit{real-time} dan \textit{non-real-time}]{Ilustrasi perbedaan permasalahan \textit{scheduling} pada aplikasi \textit{real-time} dan \textit{non-real-time}. (a) \textit{Scheduling} aplikasi \textit{non-real-time} (b) \textit{Scheduling} aplikasi \textit{real-time}}
    \label{figure:scheduling_comparison}
\end{figure}

Algoritma untuk \textit{scheduling} dapat diklasifikasikan dalam banyak dimensi.  Beberapa
algoritma \textit{scheduling} dibuat untuk \textit{task} periodik sedangkan algoritma lain
dibuat untuk \textit{task} aperiodik.  Demikian juga beberapa algoritma dibuat untuk
\textit{preemptible task} sedangkan algoritma lain dibuat untuk \textit{nonpreemptible task}.
Algoritma \textit{scheduling} dipengaruhi oleh faktor-faktor seperti tingkat kekritisan,
keterkaitan, \textit{resource}, dan \textit{deadline} aplikasi.

Algoritma \textit{scheduling} juga berbeda tergantung komputer tempat sistem akan dijalankan.
Beberapa faktor komputer yang memengaruhi algoritma \textit{scheduling} adalah arsitektur
\textit{processor} (\textit{uniprocessor} atau \textit{multiprocessor}), metode komunikasi
\textit{shared memory} atau \textit{message-passing}, dan jaringan.

\subsection{Arsitektur Sistem}

Standar ARINC 653 menspesifikasikan arsitektur sistem menggunakan arsitektur IMA.  Setiap
\textit{core module} dapat memiliki satu atau lebih \textit{processor}. Arsitektur dari
\textit{core module} memengaruhi implementasi sistem operasi, tetapi tidak memengaruhi
\textit{interface} APEX yang digunakan oleh \textit{aplikasi} pada masing-masing partisi.
Aplikasi harus mudah dipindahkan antar \textit{core module} dan antar \textit{processor} dari
sebuah \textit{core module} tanpa perlu adanya modifikasi \textit{interface} dengan sistem
operasi.  \autoref{figure:arinc653_module_architecture} menunjukkan contoh arsitektur sistem
ARINC 653 serta komunikasi antar komponen pada sistem.

\begin{figure}[htbp]
    \centering
    \includegraphics[scale=0.4]{resources/arinc653-architecture.png}
    \caption[Contoh arsitektur \textit{module}]{Contoh arsitektur \textit{module} \citep{AirlinesElectronicEngineeringCommittee2012}}
    \label{figure:arinc653_module_architecture}
\end{figure}

\subsection{Fungsionalitas Sistem}

Pada tingkat \textit{core module}, sistem operasi memanajemeni partisi dan komunikasi antar
partisi.  Pada tingkat partisi, sistem operasi memanajemeni proses aplikasi dalam sebuah partisi
dan komunikasi antar proses dalam partisi tersebut.  Sistem operasi menyediakan fasilitas pada
kedua level tersebut, dengan beberapa perbedaan pada fungsi dan konten tergantung pada lingkup
operasinya.  Oleh karena itu, definisi dari fasilitas yang diberikan sistem operasi membedakan pada tingkat
apa fasilitas tersebut beroperasi.

Pada setiap saat, sebuah \textit{module} berada dalam keadaan inisialisasi, operasional, atau
diam.  Pada saat dinyalakan, \textit{module} memulai eksekusinya dalam keadaan inisialisasi.
Setelah inisialisasi berhasil, \textit{module} masuk ke dalam keadaan operasional.  Selama
\textit{module} dalam keadaan operasional, sistem operasi memanajemeni partisi, proses aplikasi,
dan komunikasi.  \textit{Module} akan terus berada pada keadaan operasional sampai dimatikan
atau fungsi \textit{health monitoring} pada \textit{module} memerintahkan \textit{module} untuk
masuk ke dalam keadaan inisialisasi kembali atau keadaan diam.

Berikut adalah fungsionalitas sistem yang dispesifikasikan oleh standar ARINC 653.

\begin{enumerate}

    \item Manajemen Partisi.

    \item Manajemen Proses.

    \item Manajemen Waktu.

    \item Manajemen Memori.

    \item Komunikasi Antar Partisi.

    \item Komunikasi Dalam Partisi.

\end{enumerate}

Penelitian ini hanya akan membahas manajemen partisi dan manajemen waktu yang dispesifikan oleh
ARINC 653.

\subsubsection[Manajemen Partisi]{Manajemen Partisi}

Inti dari standar ARINC 653 adalah konsep partisi, yaitu aplikasi yang beroperasi dalam sebuah
\textit{module} dipartisi terhadap ruang (partisi memori) dan waktu. Sebuah partisi dapat
dikatakan sebagai satu unit aplikasi yang didesain untuk memenuhi batasan-batasan partisi.

Sistem operasi pada dasarnya sudah memiliki cara untuk melakukan partisi \textit{resource} yang
dimilikinya.  Sistem terpartisi memungkinkan partisi dengan tingkat kekritisan yang berbeda
berjalan pada satu \textit{module} yang sama tanpa memengaruhi partisi lain secara
\textit{ruang} dan waktu.  Bagian dari sistem operasi yang bekerja pada tingkat \textit{core
module} bertanggung jawab untuk melakukan partisi dan memanajemeni partisi dalam \textit{module}
tersebut.

Partisi ditentukan urutannya dengan melakukan \textit{scheduling} secara melingkar dan tetap.
Untuk dapat melakukan \textit{scheduling} secara melingkar, sistem operasi menyimpan
\textit{major time frame} untuk durasi tertentu yang kemudian akan diulang secara terus menerus
selama \textit{module} berjalan.  Partisi akan terdaftar pada \textit{major time frame} sebagai
\textit{partition window} yang memiliki nilai \textit{offset} sesuai dengan urutan yang telah
ditentukan.  Partisi yang akan dijadwalkan terlebih dahulu akan mendaftarkan \textit{partition
window} dengan nilai \textit{offset} yang lebih kecil.  Urutan yang didapatkan oleh suatu
partisi ditentukan oleh pengaturan urutan yang dibuat pada saat integrasi.  Metode ini menjamin
bahwa \textit{scheduling} yang dilakukan akan bersifat deterministik sehingga dapat diprediksi.

Sebuah \textit{module} dapat memiliki beberapa partisi yang memiliki periode pengerjaan yang
berbeda-beda.  \textit{Major time frame} didefinisikan sebagai kelipatan dari kelipatan
persekutuan terkecil dari seluruh periode pengerjaan partisi pada \textit{module} tersebut.
Setiap \textit{major time frame} mempunyai partisi yang sama.  Periode pengerjaan masing-masing
partisi harus terpenuhi dengan mengatur frekuensi dan ukuran \textit{partition window} dalam
\textit{major time frame}.

\subsubsection{Manajemen Waktu}
\label{section:arinc653_scheduling}

Manajemen waktu adalah proses penting yang menjadi karakteristik sebuah sistem operasi yang
digunakan untuk sistem \textit{real-time}. Waktu harus unik dan tidak bergantung pada eksekusi
partisi apa pun dalam \textit{module} pada sistem ARINC 653. Seluruh nilai waktu yang digunakan
harus terkait dengan waktu unik tersebut. Sistem operasi menyediakan \textit{time-slicing} untuk
\textit{scheduling}, \textit{deadline}, dan \textit{periodicity} dari sebuah partisi.
\textit{Scheduler} yang dispesifikasikan oleh ARINC 653 merupakan \textit{hierarchical
scheduler}, yaitu \textit{scheduler} yang bekerja pada beberapa tingkat. \textit{Scheduling}
pada sistem ARINC 653 bekerja pada dua tingkat, yaitu pada tingkat partisi dan tingkat aplikasi.
Pada tingkat partisi, \textit{scheduler} bekerja untuk menjadwalkan partisi-partisi pada sistem
tersebut. Proses ini disebut sebagai \textit{partition scheduling}. Pada tingkat aplikasi,
\textit{scheduler} bekerja untuk menjadwalkan aplikasi pada masing-masing partisi. Proses ini
disebut sebagai \textit{process scheduling}. Penelitian ini akan berfokus pada \textit{partition
scheduling}

Ilustrasi \textit{partition scheduling} pada sistem ARINC 653 dapat dilihat pada
\autoref{figure:arinc653_scheduling}.

\begin{figure}[htbp]
    \centering
    \includegraphics[scale=0.8]{resources/arinc653_scheduling.jpg}
    \caption[Contoh \textit{partition scheduling} pada sistem ARINC 653]{Contoh \textit{partition scheduling} pada sistem ARINC 653 \citep{Larry2009}}
    \label{figure:arinc653_scheduling}
\end{figure}

Sebuah kuantum waktu terasosiasi dengan setiap partisi \citep{Larry2009}. Kuantum waktu
merepresentasikan waktu maksimal partisi tersebut dapat berjalan. Pada setiap kuantum waktu,
partisi akan berjalan untuk menyediakan fungsi yang diharapkan. \textit{Deadline} merupakan
batas waktu maksimum ketika hasil perhitungan partisi tersebut dapat digunakan. Batasan tersebut
merupakan syarat yang menentukan apakah partisi tersebut berhasil dikerjakan atau tidak. Selama
partisi tersebut selesai dikerjakan tanpa melebihi \textit{deadline}, maka \textit{deadline}
partisi tersebut dikatakan terpenuhi. Jika tidak, maka \textit{deadline} partisi tersebut tidak
terpenuhi dan sistem akan menghasilkan \textit{error}.  \textit{Error} yang diakibat oleh
\textit{deadline} yang tidak terpenuhi menjadi acuan bahwa sistem tidak berhasil menjamin 100\%
\textit{real-time}.

\subsection{\textit{Health Monitoring}}
\label{section:health_monitoring}

Standar ARINC 653 menspesifikasikan komponen untuk melakukan \textit{health monitoring}, yakni
proses untuk mendeteksi kegagalan yang terjadi pada sistem dan melakukan \textit{recovery
action} untuk mengatasi kegagalan tersebut. Setelah melakukan pendekteksian dan penanganan
kegagalan, komponen tersebut harus dapat melaporkan kegagalan yang terjadi pada sistem.

\textit{Health} monitoring pada sistem ARINC 653 bekerja pada tiga tingkatan kegagalan yang
dapat terjadi pada sistem.

\begin{enumerate}

	\item Kegagalan pada tingkat proses

		Kegagalan pada level ini berakibat pada satu atau lebih proses pada sebuah
		partisi. Contoh kegagalan pada tingkat proses seperti: kegagalan aplikasi, IRQ
		pada O/S yang ilegal, dan kegagalan pada proses eksekusi (\textit{stack
		overflow}, \textit{page fault}, dll).

	\item Kegagalan pada tingkat partisi

		Kegagalan pada level ini berakibat pada satu buah partisi. Contoh kegagalan pada
		tingkat partisi seperti: kegagalan inisialisasi partisi, kegagalan yang terjadi
		akibat manajemen proses, dan kegagalan yang terjadi karena gagal menangani
		kegagalan aplikasi.

	\item Kegagalan pada tingkat modul

		Kegagalan pada level ini berakibat pada seluruh partisi pada sebuah modul.
		Contoh kegagalan pada tingkat modul seperti: kegagalan pada saat melakukan
		pergantian partisi dan kegagalan pada saat mengeksekusi fungsi milik sistem.

\end{enumerate}

Respons yang diberikan oleh sistem pada saat kegagalan terdekteksi pada sistem ARINC 653
bergantung pada implementasi sistem. Namun, standar ARINC 653 memberikan cara pendeteksian
kegagalan serta mekanisme untuk dapat memberikan respons.
\begin{enumerate}
	\item Kegagalan pada tingkat proses

		Kegagalan pada level ini dideteksi menggunakan proses dengan prioritas
		tertinggi. Proses berfungsi hanya untuk menangani kegagalan pada
		aplikasi-aplikasi yang terdapat pada sebuah partisi.

	\item Kegagalan pada tingkat partisi dan tingkat modul

		Kegagalan pada level ini dideteksi dengan melakukan \textit{error handling} pada
		sistem operasi dari sistem ARINC 653.

\end{enumerate}

Meski respons yang diberikan bergantung pada jenis kegagalan yang terjadi, standar ARINC 653
memberikan spesifikasi umum respons.
\begin{enumerate}

	\item Kegagalan pada tingkat proses

		Respons yang dapat dilakukan pada tingkat proses adalah:
		\begin{enumerate}
			\item abaikan sambil mencatat kegagalan yang terjadi;
			\item abaikan sampai $n$ kali sebelum melakukan \textit{recovery action}
				lain;
			\item mengulang proses yang mengalami kegagalan;
			\item mengulang partisi yang memiliki proses tersebut;
			\item mematikan partisi yang memiliki proses tersebut.
		\end{enumerate}

	\item Kegagalan pada tingkat partisi

		Respons yang dapat dilakukan pada tingkat proses adalah:
		\begin{enumerate}
			\item abaikan;
			\item mematikan partisi yang mengalami kegagalan;
			\item mengulang partisi yang mengalami kegagalan.
		\end{enumerate}

	\item Kegagalan pada tingkat modul

		Respons yang dapat dilakukan pada tingkat proses adalah:
		\begin{enumerate}
			\item abaikan;
			\item mematikan modul yang mengalami kegagalan;
			\item mengulang kembali modul yang mengalami kegagalan.
		\end{enumerate}

\end{enumerate}

\section{\textit{Fault-Tolerant Partition Scheduling}}

Studi untuk membuat sistem \textit{real-time} menjadi \textit{fault-tolerant} telah banyak
dilakukan \citep{Campbell1986} \citep{Bertossi2006}. \citet{Campbell1986} menunjukkan penambahan
\textit{deadline} pada setiap \textit{task} akan menghasilkan sistem yang
\textit{fault-tolerant}.  Meski demikian, penambahan \textit{deadline} hanya menjamin waktu
\textit{response} sistem terhadap permintaan aplikasi dan hanya berfungsi sebagai indikator pada
saat melakukan verifikasi sistem.  \textit{Fault} yang dialami aplikasi tidak hanya berupa waktu
\textit{response} saja, sehingga metode tersebut hanya membuat sistem menjadi
\textit{fault-tolerant} secara parsial.  Untuk dapat mengatasi \textit{fault} yang tidak terduga
pada saat verifikasi maupun \textit{fault} pada proses verifikasi itu sendiri, salah satu solusi
yang diusulkan adalah dengan menggunakan skema \textit{primary-backup}.

Sebuah studi menunjukkan bahwa peningkatan keandalan \textit{hierarchical scheduler} seperti
\textit{scheduler} ARINC 653 dapat dilakukan dengan menggunakan skema \textit{primary-backup}
\citep{Hyun2012}. \textit{Primary-backup partition scheduling} merupakan proses
\textit{scheduling} yang terinspirasi oleh metode replikasi data. Pendekatan paling sederhana
untuk dapat melakukan \textit{primary-backup partition scheduling} adalah dengan memberikan
salinan untuk setiap \textit{task}, dan memberikan salinan tersebut pada \textit{processor} yang
berbeda dengan \textit{task} aslinya.  Namun, pendekatan tersebut tidak efisien karena
memerlukan banyak \textit{processor} tambahan \citep{Oh1994}.  \citet{Bertossi2006} menunjukkan
cara mengimplementasikan \textit{primary-backup partition scheduling} yang lebih efisien pada
sistem \textit{real-time} yang menggunakan algoritma \textit{scheduling} \textit{Rate-
Monotonic-First-Fit} (RMFF) dengan memberikan \textit{task} \textit{salinan} pada
\textit{scheduler} jika dan hanya jika \textit{task} utama mengalami kegagalan.  \citet{Jin2013}
berhasil membuktikan secara matematis bahwa penggunaan \textit{primary-backup partition
scheduling} dapat dilakukan pada sistem \textit{uniprocessor} yang terpartisi dengan melakukan
perluasan model \textit{resource}.

\section{Virtualisasi}

Dalam ranah komputasi, virtualisasi adalah tindakan untuk membuat versi virtual dari suatu hal.
Virtualisasi pertama kali digunakan sekitar tahun 1960 sebagai metode untuk membagi
\textit{resources} sebuah komputer secara logika kepada beberapa aplikasi yang akan
menggunakannya.  Sejak saat itu, arti dari istilah virtualisasi mengalami perluasan makna.

Virtualisasi dapat dilakukan pada berbagai macam hal terkait dengan komputer.  Sebagai contoh,
sistem operasi modern sudah menggunakan konsep virtualisasi sederhana.  Sebuah sistem operasi
akan menangani pembagian \textit{resources} untuk diberikan pada aplikasi-aplikasi yang
berjalan pada saat yang bersamaan. Dalam kasus ini, sistem operasi memberikan ilusi kepada
masing-masing proses bahwa proses tersebut memiliki seluruh \textit{resources} yang terdapat
pada komputer tersebut. Namun, yang terjadi adalah sistem operasi membagi \textit{resources}
komputer dengan metode sedemikian rupa sehingga pemakaian \textit{resources} tetap konsisten
dengan ilusi yang diberikan oleh sistem operasi kepada masing-masing proses.

Sebagai contoh, virtualisasi CPU dilakukan dengan melakukan \textit{preemption} pada proses yang
sedang menggunakan CPU sehingga memberikan kesempatan bagi proses lain untuk menggunakan CPU.
Sistem operasi melakukan melakukan \textit{context switch} pada proses yang terkena
\textit{preemption} sehingga pada saat proses tersebut mendapatkan CPU pada giliran selanjutnya,
proses tersebut dapat melanjutkan apa yang dikerjakan sebelumnya.  Sementara itu, virtualisasi
memori dilakukan dengan melakukan pemetaan alamat memori virtual dengan kapasitas
2\textsuperscript{32} (2\textsuperscript{64} pada sistem 64-bit) ke alamat memori fisik dengan
kapasitas sesuai dengan perangkat keras pada komputer tersebut.  Kapasitas yang dapat digunakan
oleh sebuah proses pada alamat memori fisik serta cara pemetaannya dilakukan oleh perangkat
\textit{memory management unit} pada komputer.  \textit{Devices} yang dimiliki oleh sebuah
komputer juga divirtualisasikan oleh sistem operasi.  Contoh \textit{device} yang
divirtualisasikan oleh sistem operasi adalah \textit{storage}.

Salah satu kegunaan metode virtualisasi pada ranah komputasi adalah untuk membuat
\textit{virtual machine}.  Sebuah \textit{virtual machine} adalah representasi mesin secara
logika yang memiliki \textit{resources} layaknya mesin biasa.  Umumnya, \textit{virtual machine}
digunakan untuk memberikan ilusi sebuah komputer sehingga dari sudut pandang perangkat lunak
yang berjalan di atasnya, perangkat lunak tersebut seakan berjalan satu buah komputer fisik
tanpa ada pembagian \textit{resources}.  Tentunya hal tersebut hanya berlaku untuk sistem
operasi atau perangkat lunak yang berjalan langsung di atas perangkat keras.  Sebuah komputer
dapat direpresentasikan menjadi beberapa \textit{virtual machine} sehingga memungkinkan beberapa
sistem operasi beroperasi secara bersamaan pada komputer tersebut.

\section{\textit{Hypervisor}}

Virtualisasi dilakukan menggunakan perangkat lunak virtualisasi yang disebut sebagai
\textit{hypervisor}.  \textit{Hypervisor} adalah sebuah perangkat lunak yang bertugas untuk
membuat dan menjalankan beberapa \textit{virtual machine}.  \textit{Hypervisor} berfungsi untuk
memanajemeni \textit{resources} fisik yang tersedia untuk digunakan oleh \textit{virtual
machine} yang berjalan di bawah \textit{hypervisor} tersebut.  Dengan demikian,
\textit{hypervisor} memiliki peran sama seperti sistem operasi, hanya saja ketimbang
memanajemeni \textit{resources} fisik untuk digunakan oleh proses perangkat lunak,
\textit{hypervisor} memanajemeni \textit{resources} untuk sebuah proses yang juga mengatur
\textit{resources} yang diberikan pada proses tersebut untuk proses perangkat lunak yang
berjalan di bawahnya.  Istilah \textit{hypervisor} didapatkan karena \textit{hypervisor}
membawahkan \textit{supervisor}, pada kasus ini \textit{supervisor}-nya merupakan sistem
operasi.

Terdapat dua tipe \textit{hypervisor}, \textit{hypervisor type-1} dan \textit{hypervisor
type-2}. Pada \textit{hypervisor type-1}, \textit{hypervisor} berjalan langsung di atas
perangkat keras. Karena \textit{hypervisor type-1} terhubung langsung dengan perangkat keras,
maka aplikasi yang berjalan di atas \textit{hypervisor type-1} hanya memerlukan satu lapisan
tambahan sebelum dapat dijalankan oleh perangkat keras. Berbeda dengan \textit{hypervisor
type-1}, \textit{hypervisor type-2} berjalan di atas sistem operasi yang sudah berjalan.
\textit{Hypervisor} berjalan sebagai aplikasi biasa pada sistem operasi, dengan demikian
terdapat dua lapisan tambahan sebelum kode aplikasi dapat dijalankan oleh perangkat keras.

\section{Xen}

Sampai saat ini, sudah terdapat banyak solusi virtualisasi yang tersedia. Karena fokus dari
penelitian ini adalah pada bidang avionika, maka virtualisasi yang digunakan harus memiliki
performa yang tinggi. Dengan demikian, solusi virtualisasi akan dipilih apabila menggunakan
\textit{hypervisor type-1}. Terdapat beberapa solusi virtualisasi dengan \textit{hypervisor
type-1} seperti VMWare ESXi, Microsoft Hyper-V, Oracle VM Server dan Xen. Namun, solusi
virtualisasi dengan \textit{hypervisor type-1} yang gratis dan \textit{open source} hanya Xen.

Kelebihan Xen sebagai \textit{hypervisor} \textit{open-source} memudahkan pengembang dalam
menuliskan komponen baru yang dibutuhkan. Xen juga memberikan \textit{interface} yang
relatif mudah untuk melakukan pengembangan komponen baru. Fokus utama pada buku ini adalah
pembuatan \textit{scheduler} baru dan Xen telah menyediakan \textit{interface} untuk membuat
\textit{scheduler} baru dengan mudah.  Berdasarkan kelebihan-kelebihan tersebut, maka penelitian
akan menggunakan Xen sebagai solusi virtualisasi.

Xen merupakan sebuah perangkat lunak untuk melakukan virtualisasi yang bekerja di antara sistem
operasi dan perangkat keras. Terdapat tiga komponen utama pada sistem yang menggunakan Xen,
yaitu \textit{hypervisor}, \textit{kernel}, dan aplikasi yang bekerja pada \textit{user space}.

\begin{figure}[!ht]
    \includegraphics[scale=0.5]{./resources/xen-ring.png}
    \caption[Perbedaan penggunaan \textit{ring} \textit{native} dan
    \textit{paravirtualized}]{Perbedaan penggunaan \textit{ring} \textit{native} (kiri) dan
    \textit{paravirtualized} (kanan) \citep{Chisnall2014}}
    \label{figure:xen_ring}
\end{figure}

Pada Xen, \textit{kernel} sebuah sistem operasi tidak lagi beroperasi pada \textit{ring} 0.
\textit{Ring} 0 digunakan oleh \textit{hypervisor} yang akan mengatur \textit{resources} yang
akan diberikan pada \textit{virtual machine} di bawahnya.  \textit{Kernel} sistem operasi
dijalankan bergantung pada arsitektur \textit{processor}-nya.  Sebagai contoh, pada sistem IA32,
\textit{kernel} sistem operasi berjalan pada \textit{ring} 1 seperti pada
\autoref{figure:xen_ring}.  Hal tersebut mengakibatkan \textit{kernel} yang berjalan di atas Xen
tetap dapat mengakses memori yang teralokasi untuk aplikasi yang berjalan di atas
\textit{kernel} tersebut dan \textit{kernel} tetap tidak dapat mengakses maupun diakses oleh
aplikasi yang terdapat pada \textit{kernel} lain.  \textit{Hypervisor}, yang berjalan pada
\textit{ring} 0, terproteksi dari \textit{kernel} yang berjalan pada \textit{ring} 1 dan
aplikasi yang berjalan pada \textit{ring} 3.  Oleh karena \textit{kernel} berjalan pada \textit{ring}
1, \textit{kernel} tidak dapat menjalankan \textit{privileged instructions} seperti biasanya.
\textit{Hypervisor} pada Xen menyediakan \textit{hypercall} sebagai pengganti \textit{privileged
instructions}.  \textit{Hypercall} digunakan oleh \textit{hypervisor} untuk menjembatani
permintaan \textit{kernel} pada perangkat lunak. Hal ini dilakukan oleh \textit{hypervisor}
dengan cara menangkap \textit{trap} yang dilakukan oleh \textit{kernel}, kemudian mengeksekusi
\textit{trap} tersebut melalui \textit{ring} 0. Dengan demikian, \textit{kernel} yang seharusnya
berjalan pada \textit{ring} 0 tidak perlu mengubah kode programnya dan tetap dapat melakukan
\textit{privileged instructions} sebagaimana bila \textit{kernel} berjalan di atas \textit{ring}
0.

Fungsi dari sebuah \textit{hypervisor} adalah untuk menjalankan \textit{virtual machine}.  Pada
Xen, \textit{hypervisor} menjalankan \textit{virtual machine} yang disebut sebagai
\textit{domain}.  Terdapat dua jenis \textit{domain} pada Xen, yaitu dom0 dan domU.  Dom0
merupakan \textit{domain} yang dapat menggunakan \textit{privileged instructions} untuk
memanajemeni \textit{devices} yang dijalankan oleh Xen pada saat sistem pertama kali beroperasi.
DomU merupakan \textit{domain} yang tidak dapat menggunakan \textit{privileged instructions}.
Kewajiban dom0 dalam menangani \textit{devices} adalah dom0 harus dapat menyediakan
\textit{devices} untuk semua domU.  Oleh karena perangkat keras tidak dapat diakses oleh
beberapa sistem operasi sekaligus, maka akses \textit{devices} oleh domU harus melalui dom0.
\autoref{figure:xen_split_driver} mengilustrasikan bagaimana paket jaringan yang dikirim oleh
domU melalui dom0.  Perbedaan utama pada pengiriman paket jaringan adalah pada \textit{layer}
\textit{TCP/IP}, \textit{layer} tersebut tidak meneruskan langsung ke \textit{network device}
tetapi menulis paket pada \textit{shared memory}.  Kemudian dom0 akan membaca paket yang
terdapat pada \textit{shared memory} untuk dilanjutkan ke \textit{network device} yang
sebenarnya.  Penggunaan \textit{shared memory} seperti ilustrasi tersebut banyak digunakan pada
berbagai komponen yang membentuk Xen.

\begin{figure}[!ht]
    \includegraphics[scale=0.5]{./resources/xen-split-driver.png}
    \caption[Jalur sebuah paket jaringan yang dikirim dari domU]{Jalur sebuah paket jaringan yang dikirim dari domU \citep{Chisnall2014}}
    \label{figure:xen_split_driver}
\end{figure}

\section{ARLX}

ARLX merupakan prototipe sistem ARINC 653 yang dibangun di atas Xen oleh DornerWorks
\citep{VanderLeest2010}. Prototipe dibangun dengan membuat \textit{scheduler} seperti pada
spesifikasi standar ARINC 653 pada Xen. Dalam melakukan partisi memori, prototipe tersebut
memanfaatkan mekanisme partisi memori menggunakan \textit{memory management unit} yang sudah
merupakan mekanisme bawaan pada Xen.

DornerWorks membuka kode sumber \textit{scheduler} ARINC 653 tersebut dan kode sumber tersebut
telah digabungkan pada \textit{source tree} milik Xen. Dengan demikian, \textit{scheduler} ARINC
653 yang dikembangkan pada ARLX juga dapat digunakan oleh Xen tanpa memerlukan lapisan eksekusi
tambahan. Untuk selanjutnya, ARLX akan merujuk pada Xen yang menggunakan \textit{scheduler}
tersebut.

\begin{figure}[!ht]
    \centering
    \includegraphics[scale=0.6]{resources/xen-arinc653-partitions.png}
    \vspace{10pt}
    \caption{\textit{Hypervisor} ARLX \citep{VanderLeest2010}}
    \label{figure:xen_arinc653_partitions}
\end{figure}

Setiap partisi akan direpresentasikan dengan menggunakan \textit{domain} pada Xen.  Dom0 pada
Xen akan digunakan sebagai partisi tempat APEX dan \textit{device driver} berada, sedangkan
partisi untuk aplikasi avionik akan direpresentasikan menggunakan domU seperti pada
\autoref{figure:xen_arinc653_partitions}.

\begin{figure}[!ht]
    \includegraphics[scale=0.6]{resources/xen-arinc653-partition-schedule.png}
    \vspace{15pt}
    \caption{\textit{Schedule} partisi pada ARLX \citep{VanderLeest2010}}
    \label{figure:xen_arinc653_partitions_schedule}
\end{figure}

\textit{Partition scheduling} dilakukan seperti yang telah dispesifikasikan pada standar ARINC
653. Layanan sebuah partisi akan didaftarkan dengan menggunakan \textit{hypercall} seperti pada
Xen. Partisi akan dipilih sesuai dengan urutan pada daftar jadwal yang diberikan pada saat
melakukan \textit{hypercall}.  Ilustrasi penjadwalan partisi yang dilakukan oleh
\textit{partition scheduler} milik ARLX dapat dilihat pada
\autoref{figure:xen_arinc653_partitions_schedule}. Pada ilustrasi tersebut, ARLX akan
menyediakan dua buah layanan masing-masing pada sebuah partisi tersendiri. Pada partisi pertama
(domU1), ARLX akan menyediakan layanan bernama FMS. Pada partisi kedua (domU2), ARLX akan
menyediakan layanan bernama TAWS.

Secara umum, algoritma tersebut sama seperti algoritma \textit{scheduling} yang telah dipaparkan
pada \autoref{section:arinc653_scheduling}. Pada saat \textit{major time frame} dimulai,
{dom0} akan berjalan dan menjalankan fungsi-fungsi yang seharusnya dilakukan oleh dom0,
seperti manajemen I/O, menyediakan API ARINC 653 untuk partisi-partisi pada sistem, dan
lain-lain. Setelah kuantum waktu yang dimiliki oleh dom0 habis, domU1 akan berjalan dan
menyediakan layanan FMS. Setelah  kuantum waktu yang dimiliki oleh domU1 habis, domU2 akan
berjalan dan menyediakan layanan TAWS. Hal ini akan dilakukan secara terus menerus sampai sistem
dimatikan. Slot \textit{hypervisor} pada selang \textit{major time frame} digunakan oleh
\textit{hypervisor} untuk menentukan partisi mana yang akan mendapatkan waktu CPU.

Spesifikasi ARINC 653 berlaku apabila sistem operasi yang digunakan merupakan sistem operasi
\textit{real-time}. Pada dasarnya, Xen tidak dibangun sebagai \textit{hypervisor} dengan
kapabilitas \textit{real-time}. Terdapat sebuah proyek yang ingin membangun kapabilitas
\textit{real-time} pada Xen. Namun, proyek tersebut tidak mendukung \textit{scheduler} ARINC 653
yang terdapat pada ARLX. Dengan demikian, penelitian ini akan tetap menggunakan
\textit{scheduler} ARINC 653 pada ARLX meskipun Xen tidak memiliki kapabilitas
\textit{real-time}.

% vim: tw=96
