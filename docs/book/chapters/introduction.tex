%! TEX root = ../final-project.tex
\chapter{PENDAHULUAN}

\section{Latar Belakang}

% Avionik
Sistem penerbangan atau avionik adalah sistem yang terdiri dari perangkat elektronik yang
digunakan pada pesawat terbang. Avionik berfungsi untuk memberikan fasilitas untuk
mengoperasikan pesawat terbang. Sistem yang termasuk dalam avionik adalah sistem komunikasi,
navigasi, \textit{dashboard} untuk manajemen sistem, dan sistem-sistem yang dipasang pada
pesawat terbang untuk melakukan fungsionalitas tersendiri.

%  Design-issues pada avionik
Avionik merupakan komponen penting dalam sebuah pesawat terbang dan merupakan \textit{safety\hyp
critical system}, yaitu sistem yang apabila mengalami kegagalan dapat mengakibatkan kematian,
kerugian besar, atau kerusakan lingkungan.  Sistem operasi yang digunakan oleh avionik pada
sebuah pesawat umumnya merupakan sistem \textit{real-time}, yaitu sistem yang kebenaran
komputasinya tidak hanya ditentukan oleh kebenaran hasil komputasi secara logika, tetapi juga
waktu pada saat hasil komputasi dapat digunakan \citep[p.~6]{Shin1994}. Dengan demikian,
manajemen waktu adalah aspek yang sangat mendasar pada sistem \textit{real-time}. Perangkat yang
digunakan untuk avionik memiliki beberapa standar yang wajib dipenuhi. Standar tersebut dibuat
oleh Aeronautical Radio, Incorporated (ARINC) dalam sebuah seri standar ARINC 600. Penelitian
ini akan mendalami lebih lanjut mengenai sistem operasi \textit{real-time} pada avionik yang
dispesifikasikan pada penerbangan bernama ARINC 653.

% ARINC 653
Standar ARINC 653 menspesifikasikan partisi ruang dan waktu pada sistem operasi
\textit{real-time} agar avionik yang menggunakan sistem operasi tersebut
\textit{safety-critical}. Setiap proses aplikasi disebut sebagai partisi.  Selain spesifikasi
partisi, ARINC 653 juga mendefinisikan API untuk memudahkan manajemen partisi.  Sistem operasi
yang memenuhi standar ARINC 653 berfungsi untuk memanajemeni beberapa partisi beserta komunikasi
antar partisi.  Manajemen yang dilakukan meliputi \textit{resources} komputasi umum seperti CPU,
\textit{timing}, memori, dan \textit{devices} serta \textit{resources} untuk komunikasi.
Penelitian ini akan mendalami lebih lanjut mengenai manajemen CPU, khususnya proses
\textit{partition scheduling}, yang dispesifikasikan oleh standar ARINC 653.

% Fault-tolerant scheduling
Partisi pada standar ARINC 653 dapat membuat eksekusi dan pembaharuan perangkat lunak pada
sistem tersebut lebih tepercaya dan fleksibel \citep{Jin2013}.  Partisi dapat diganti atau
ditambahkan selama partisi tersebut telah mendapatkan sertifikasi. Sebagai contoh, aplikasi
navigasi yang dijalankan pada suatu partisi tertentu pada avionik dapat diganti dengan versi
terbaru yang menggunakan algoritma navigasi yang lebih baik tanpa mengganggu partisi lain.
Namun, standar ARINC 653 tidak menghasilkan sistem yang andal. Akan terdapat kemungkinan
terjadinya \textit{fault} pada perangkat lunak selama sistem beroperasi, meskipun keamanan dan
kebenaran perangkat lunak tersebut sudah terverifikasi. 

Permasalahan pada partisi yang mengalami kegagalan adalah layanan yang disediakan oleh partisi
tersebut akan berhenti bekerja sebagaimana seharusnya. Pada avionik, hal ini dapat berakibat
terjadinya kesalahan kalkulasi, sehingga meningkatkan ancaman keselamatan bagi manusia maupun
lingkungan yang terlibat. Untuk mempermudah penjelasan, asumsikan terdapat hubungan satu ke satu
antara partisi dengan layanan. Cara termudah untuk mengembalikan layanan pada keadaan semula
adalah dengan mengembalikan keadaan partisi seperti semula. Namun, dalam kasus tertentu, keadaan
partisi tidak dapat dikembalikan seperti semula dikarenakan kegagalan yang terjadi bersifat
tetap. Karena itu, perlu ada mekanisme untuk menggantikan layanan pada partisi yang mengalami
kegagalan.

Untuk meningkatkan keandalan sistem dalam sistem \textit{real-time}, salah satu cara yang dapat
digunakan adalah dengan menggunakan \textit{fault-tolerant scheduling} \citep{Campbell1986}
\citep{Han2003} \citep{Shin2008}. Sebuah studi menunjukkan bahwa peningkatan keandalan
\textit{hierarchical scheduler} seperti \textit{scheduler} ARINC 653 dapat dilakukan dengan
menggunakan skema \textit{primary-backup} \citep{Hyun2012}. Untuk menghindari permasalahan yang
akan muncul akibat partisi yang gagal, \textit{scheduler} dapat menjalankan partisi pengganti
yang memilki layanan yang bersesuaian dengan partisi yang mengalami kegagalan. Partisi pengganti
ini disebut sebagai partisi \textit{backup}. Layanan yang diberikan partisi \textit{backup} akan
identik dengan layanan yang diberikan oleh partisi \textit{primary}. Apabila partisi
\textit{primary} mengalami kegagalan ketika menyediakan sebuah layanan, partisi \textit{backup}
akan melindungi sistem dari kegagalan dengan mengambil alih peran partisi \textit{primary}
sebagai penyedia layanan. Hal tersebut mengakibatkan layanan yang disediakan oleh partisi yang
mengalami \textit{fault} akan tetap tersedia dan sistem akan tetap berjalan sebagaimana
seharusnya, sehingga sistem dapat dikatakan menjadi lebih andal.

Meski demikian, peningkatan tersebut hanya terlihat secara teoretis dan belum ada studi mengenai
penggunaan \textit{primary-backup scheduling} pada sistem ARINC 653. Untuk dapat melihat
perbandingan antara \textit{primary-backup partition scheduling} dengan \textit{scheduling}
sistem ARINC 653, perlu dilakukan pengujian keandalan sistem pada sistem yang menggunakan
\textit{primary-backup partition scheduling} dan sistem yang menggunakan \textit{scheduling}
sistem ARINC 653.

% Hypervisor
Arsitektur komputer yang dispesifikasikan pada standar ARINC 653 dapat divirtualisasikan dengan
menggunakan \textit{hypervisor}. \textit{Hypervisor} adalah sebuah komponen yang membuat dan
menjalankan \textit{virtual machine}.  Setiap \textit{virtual machine} akan memiliki ruang
memori dan waktu tersendiri sehingga dapat digunakan untuk mengimplementasikan sebuah partisi.
Salah satu solusi sistem ARINC 653 yang menggunakan \textit{hypervisor} adalah Xen ARINC 653.
ARLX adalah sebuah prototipe sistem ARINC 653 yang dibangun di atas Xen, yaitu
\textit{hypervisor} \textit{open-source} dengan lisensi GPL sehingga ARLX juga
\textit{open-source}.  Solusi \textit{open-source} akan mempermudah pengembang aplikasi dan
peneliti untuk melakukan pengembangan dan percobaan pada lingkungan ARINC 653 sebelum membeli
solusi berbayar. Xen ARINC 653 merupakan solusi yang sangat tepat untuk melakukan studi
penggunaan \textit{fault-tolerant partition scheduling} dan perbedaan antara sistem yang
menggunakan \textit{fault-tolerant partition scheduling} dengan sistem yang menggunakan
\textit{partition scheduling} pada spesifikasi ARINC 653.

\section{Rumusan Masalah}

Permasalahan yang terjadi berdasarkan uraian latar belakang adalah standar ARINC 653 tidak
membuat sistem menjadi andal. Padahal, fungsi dari standar ARINC 653 adalah sebagai panduan
pengembangan sistem operasi \textit{real-time} pada \textit{safety-critical system}.  Masalah
tersebut dapat diselesaikan dengan menggunakan \textit{fault-tolerant partition scheduling}.
Meski \textit{fault-tolerant partition scheduling} membuat sistem menjadi andal secara teoretis,
perlu adanya studi lebih lanjut apakah metode \textit{fault-tolerant partition scheduling} dapat
digunakan pada sistem ARINC 653. Selain itu, perlu adanya perbandingan antara
\textit{scheduling} normal dengan \textit{fault-tolerant partition scheduling} pada ARINC 653
untuk melihat kemungkinan peningkatan atau penurunan efisiensi sistem.

Berikut adalah pertanyaan permasalahan yang akan dijawab dan diselesaikan pada tugas akhir ini.

\begin{enumerate}

    \item Bagaimana cara mengimplementasikan \textit{fault-tolerant partition scheduling} pada
    	    sistem ARINC 653 sehingga keandalan sistem meningkat?

    \item Bagaimana cara melakukan pengujian keandalan sistem ARINC 653 setelah menggunakan
    	    \textit{fault-tolerant partition scheduling}?

    \item Apakah terdapat peningkatan atau penurunan kinerja akibat peningkatan keandalan
    	    menggunakan \textit{fault-tolerant partition scheduling} pada sistem yang memenuhi
    	    standar ARINC 653?

\end{enumerate}

\section{Tujuan}
\label{section:tujuan}

Berdasarkan uraian pada rumusan masalah, tujuan yang ingin dicapai adalah meningkatkan keandalan
sistem ARINC 653 tanpa mengurangi efisiensi sistem secara signifikan. Hal ini dapat dicapai
dengan mengimplementasikan \textit{fault-tolerant partition scheduling} pada sistem ARINC 653,
melakukan pengujian keandalan sistem, dan melakukan perbandingan keandalan sistem pada saat
sebelum dan sesudah implementasi \textit{fault-tolerant partition scheduling}.

\section{Batasan Masalah}
\label{section:batasan_masalah}

Berdasarkan rumusan masalah yang ada, terdapat beberapa asumsi terkait dengan sistem avionik
yang akan dikembangkan yaitu.

\begin{enumerate}

    \item Pengujian yang dilakukan hanya akan diuji coba pada sistem dengan
    	    arsitektur x86-64.

    \item Pengujian yang dilakukan hanya akan diuji coba dengan menggunakan
    	    \textit{kernel} Linux untuk setiap partisi-nya.  Hal ini mengakibatkan partisi yang
    	    tidak menggunakan \textit{kernel} Linux mungkin memiliki perilaku yang berbeda.

\end{enumerate}

\section{Metodologi}

Metodologi yang akan digunakan dalam penelitian tugas akhir adalah sebagai berikut.

\begin{enumerate}

    \item Studi Literatur

	Pengerjaan tugas akhir diawali dengan mempelajari referensi berupa jurnal ilmiah dan
	dokumen resmi terkait dengan spesifikasi ARINC 653, arsitektur Xen, dan cara kerja ARLX.
	Pembelajaran mengenai spesifikasi ARINC 653 dilakukan untuk dapat mengetahui kriteria
	penilaian yang akan digunakan sebagai patokan dalam pembuatan kerangka pengujian.
	Pembelajaran arsitektur Xen dilakukan untuk mengetahui cara memodifikasi ARLX untuk
	mencari tahu bagaimana modifikasi dan pengujian pada prototipe tersebut akan dilakukan.

    \item Analisis

	Pada tahap ini dilakukan analisis permasalahan yang berkaitan dengan topik tugas akhir
	ini.  Analisis permasalahan akan membahas spesifikasi ARINC 653 dan arsitektur Xen untuk
	kemudian dijadikan sebagai solusi terhadap permasalahan.

    \item Perancangan Solusi dan Pengujian

	Pada tahap ini dilakukan perancangan solusi dan pengujian yang akan dilakukan yang dapat
	menyelesaikan masalah\hyp{}masalah yang telah dijelaskan pada
	\autoref{section:batasan_masalah}.

    \item Pengembangan dan Pengujian

	Pada tahap ini dilakukan pembangunan sistem pengujian serta pengembangan modul
	pengujian pada ARLX.

    \item Kesimpulan dan Saran

	Pada tahap ini akan dilakukan analisis hasil pengembangan dan pengujian serta efeknya
	terhadap sistem. Hasil analisis akan digunakan dalam penarikan kesimpulan serta
	pembuatan saran untuk penelitian selanjutnya.

\end{enumerate}

\section{Sistematika Pembahasan}

Struktur laporan tugas akhir ini adalah sebagai berikut.

\begin{enumerate}

    \item Bab I Pendahuluan : Bab ini berisi latar belakang, masalah yang dibahas, tujuan,
    	    batasan masalah, metodologi, dan sistematika pembahasan tugas akhir.

    \item Bab II Studi Literatur : Bab ini berisi dasar teori yang digunakan pada tugas akhir.

    \item Bab III Analisis Masalah dan Perancangan Solusi : Bab ini berisi analisis terhadap
    	    permasalahan yang telah dipaparkan pada Pendahuluan serta perancangan solusi yang
    	    dapat menyelesaikan permasalahan tersebut.

    \item Bab IV Pengembangan dan Pengujian : Bab ini menjelaskan pembangunan solusi pada ARLX
    	    secara rinci.

    \item Bab V Kesimpulan dan Saran : Bab ini berisi kesimpulan dari hasil pengujian dan saran
    	    untuk penelitian selanjutnya.

\end{enumerate}

% vim: tw=96
