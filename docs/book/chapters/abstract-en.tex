\clearpage
\chapter*{ABSTRACT}
\addcontentsline{toc}{chapter}{ABSTRAK}

\begin{center}

	\large \bfseries
	\MakeUppercase{Improving ARINC 653 Systems Reliability by Using Fault-Tolerant Partition Scheduling}

	\normalsize \normalfont By

	\large \penulis

	\large NIM: \nimpenulis

\end{center}

\singlespacing

The ARINC 653 specifies multiple real-time operating system components to provide isolation between
partitions. This means failure on one partition does not affect any other partition. When a
failure occurs, the system will handle the failure through health monitor component. Health
monitor will detect a failure when it occurs and try to do recovery acts to make sure the system
stays stable.  Unfortunately, some failure cannot be handled by recovery procedures defined in
the standard.  This means the failure will persists and while each partition cannot affect the
other partitions, the failure still happens and possibly leads to failure to the whole system. 

In this research, partition scheduler specified in ARINC 653 standard is extended to improve
reliability in ARINC 653 compliant systems. Scheduler is developed based on ARLX, an existing
ARINC 653 prototype built on top of Xen hypervisor. The scheduler will modified to support
primary-backup scheduling scheme when scheduling partitions on the systems. This means, the
scheduler can choose a backup partition if primary partition experienced a failure.

The extended scheduler then tested to measure increase in reliability and worst-case latency when
the system used said scheduler. Test result shows that system reliability increased
significantly on the system, but with high worst-case latency. This means the system is less
affected by failures, but cannot be used as productiono technology yet because the system could
not provide real-time performance by industry standards.

\textbf{Keywords}: hierarchical scheduling, real-time system,
scheduling, reliability, fault-tolerance

\clearpage

\defaultspacing
% vim: tw=96
