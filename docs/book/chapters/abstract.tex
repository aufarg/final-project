\clearpage
\chapter*{ABSTRAK}
\addcontentsline{toc}{chapter}{ABSTRAK}

\begin{center}

	\large \bfseries
	Pengembangan Keandalan Sistem ARINC 653 dengan Menggunakan \textit{Fault-Tolerant Partition Scheduling}

	\normalsize \normalfont Oleh

	\large \penulis

	\large NIM: \nimpenulis

\end{center}

Standar ARINC 653 memberikan spesifikasi pada komponen-komponen milik sistem operasi guna
menyediakan isolasi antara partisi pada sistem tersebut. Hal tersebut mengakibatkan kegagalan
pada sebuah partisi tidak berakibat apapun pada partisi lain. Meskipun demikian, kegagalan pada
partisi tersebut tetap terjadi. ARINC 653 memberikan spesifikasi komponen \textit{health
monitor} untuk mendeteksi dan menangani kegagalan yang terjadi pada sistem. Namun, beberapa
kegagalan dapat bersifat persisten dan tidak dapat ditangani dengan metode penanganan yang
dispesifikasikan oleh ARINC 653.  Kegagalan pada partisi dapat mengakibatkan layanan yang
diberikan oleh partisi tersebut tidak bekerja sebagaimana seharusnya sehingga mengakibatkan
kegagalan sistem secara keseluruhan.

Pada Tugas Akhir ini, sebuah \textit{scheduler} yang memanfaatkan skema \textit{primary-backup}
untuk meningkatkan keandalan sistem berbasis ARINC 653 telah dikembangkan.  \textit{Scheduler}
dibangun dengan memodifikasi kode \textit{scheduler} ARLX, yaitu prototipe sistem berbasis ARINC
653 yang dibangun di atas Xen \textit{hypervisor}. Modifikasi dilakukan agar \textit{scheduler}
dapat melakukan penjadwalan partisi \textit{backup} apabila partisi \textit{primary} mengalami
kegagalan.

\textit{Scheduler} hasil implementasi diuji untuk melihat apakah terjadi peningkatan keandalan
sistem ketika menggunakan \textit{scheduler} tertentu. Selain itu, pengujian \textit{latency}
juga dilakukan untuk melihat \textit{latency} maksimal pada sistem sehingga terlihat apakah
sistem layak digunakan sebagai sistem \textit{real-time} atau tidak.

\textbf{Kata kunci}: \textit{hierarchical scheduling}, sistem \textit{real-time},
\textit{scheduling}, keandalan, \textit{fault-tolerance}

\clearpage

% vim: tw=96
