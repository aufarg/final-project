%! TEX root = ../final-project.tex
\chapter{PENGEMBANGAN DAN PENGUJIAN}

\section{Pengembangan \textit{Primary-Backup Partition Scheduling}}
\label{section:implementasi}

Pengembangan dilakukan dengan memodifikasi kode \textit{scheduler} ARINC 653
\textit{source-tree} pada ARLX. ARLX telah menggabungkan sumber kode \textit{scheduler} miliknya
dengan \textit{source-tree} milik Xen. Versi Xen yang digunakan untuk implementasi adalah versi
\textit{4.8-stable}.  Masing-masing partisi akan menggunakan Linux 4.4.75 dengan \textit{patch}
\texttt{PREEMPT\_RT} versi 4.4.75-rt88. Hal ini diperlukan untuk menjamin \textit{latency}
seminimal mungkin pada partisi.

Modifikasi kode dilakukan pada modul-modul berikut.

\begin{enumerate}

	\item Modul libxl

	Modul ini digunakan untuk membuat \textit{tools} yang dapat digunakan pada \textit{user
	space} \textit{domain} \code{dom0} untuk melakukan pengaturan \textit{hypervisor}.

	\item Modul libxc

	Modul ini berisi fungsi-fungsi untuk menangani permintaan \textit{hypercall} dari
	\textit{domain}.

	\item Modul \textit{domain hypercall}

	Modul ini berisi definisi \textit{interface} yang digunakan untuk mengatur struktur data
	yang dikirimkan pada saat mengirim data dari \textit{domain} kepada \textit{hypervisor}.

	\item Modul \textit{scheduler}

	Modul ini berisi fungsi-fungsi yang digunakan untuk mengisi definisi \textit{interface}
	\textit{scheduler} pada Xen. Fungsi-fungsi tersebut meliputi algoritma
	\textit{scheduler} dan penanganan \textit{hypercall} yang ditujukan kepada
	\textit{scheduler}.

\end{enumerate}

Agar algoritma \textit{primary-backup partition scheduling} yang dijelaskan pada
\autoref{section:solution} dapat berjalan dengan seharusnya, perlu ada mekanisme penanganan data
yang dibutuhkan pada masing-masing domain. Hal tersebut dapat dicapai dengan mendefinisikan
struktur data yang dapat dimengerti oleh \textit{hypervisor} dan \textit{user space} untuk
berkomunikasi melalui \textit{hypercall}, membuat struktur data untuk menggunakan
\textit{hypercall}, membuat mekanisme pengiriman data melalui \textit{hypercall}, membuat
struktur data untuk menyimpan informasi \textit{domain} pada \textit{scheduler}, serta mengubah
data yang didapat oleh \textit{hypervisor} setelah menangani \textit{hypercall} sesuai dengan
struktur data \textit{domain} pada \textit{scheduler}.

Implementasi mekanisme pengiriman data dari \textit{user space} kepada \textit{hypervisor} akan
dipaparkan pada \autoref{section:modul_libxl}. Mekanisme \textit{hypercall} yang
diimplementasikan akan dipaparkan pada \autoref{section:modul_libxc}. Pendefinisian struktur
data yang digunakan pada saat melakukan \textit{hypercall} akan dipaparkan pada
\autoref{section:modul_public_hypercall}. Mekanisme konversi data yang didapat oleh
\textit{hypervisor} setelah menangani \textit{hypercall} serta definisi struktur data
masing-masing \textit{domain} pada \textit{scheduler} akan dijelaskan pada
\autoref{section:modul_scheduler_arinc653}.

Penjelasan akan dilakukan secara \textit{bottom-up} dimulai dari definisi
\textit{interface}/struktur data, kemudian fungsi dari \textit{interface}/struktur data yang
sudah didefinisikan. Masing-masing mekanisme akan dijelaskan apabila seluruh
\textit{interface}/struktur data yang digunakan pada mekanisme tersebut telah dipaparkan. Secara
garis besar, penjelasan akan dilakukan dimulai dari bagian \textit{low-level} dan
\textit{high-level} pada \textit{hypervisor}, kemudian memasuki bagian \textit{low-level} dan
\textit{high-level} pada \textit{user space}. Urutan penjelasan dilakukan sedemikian dengan
harapan akan mempermudah pembaca dalam mengerti masing-masing langkah yang akan dipaparkan pada
setiap mekanisme.

\subsection{Modul \textit{scheduler}}
\label{section:modul_scheduler_arinc653}

Subbab ini akan membahas implementasi dari \textit{primary-backup partition scheduling} pada
modul \textit{scheduler}. Implementasi \textit{scheduler} pada ARLX meliputi tiga buah
bagian.

\begin{enumerate}
	
	\item Pendefinisian \textit{scheduler}

		Pada ARLX, algoritma \textit{scheduling} yang akan digunakan \textit{hypervisor}
		untuk menjadwalkan partisi dapat dipilih pada saat \textit{boot}. Hal ini
		mempermudah dalam melakukan uji coba layanan dengan \textit{scheduler} tertentu
		guna menentukan algoritma \textit{scheduling} yang akan digunakan.

	\item Daftar Jadwal

		\textit{Scheduler} ARINC 653 memerlukan daftar partisi yang akan dijadwalkan
		pada rentang waktu tertentu. Selain itu, \textit{scheduler} ARINC 653 juga
		memerlukan periode sebuah penjadwalan. Daftar partisi yang akan dijadwalkan
		dapat diberikan melalui \textit{hypercall}. Daftar jadwal perlu dimodifikasi
		untuk memodelkan daftar jadwal seperti yang telah dipaparkan pada
		\autoref{section:daftar_jadwal}.
		
	\item Informasi partisi

		\textit{Scheduler} ARINC 653 yang terdapat pada ARLX tidak memerlukan informasi
		mengenai partisi. Pengguna hanya perlu memberikan daftar partisi beserta batasan
		\textit{real-time} yang harus dipenuhi. Daftar tersebut diberikan melalui
		\textit{global hypercall}, yaitu \textit{hypercall} yang tidak dalam konteks
		partisi tertentu. \textit{Primary-backup partition scheduling} memerlukan
		informasi mengenai partisi tertentu seperti yang telah dipaparkan pada
		\autoref{section:informasi_partisi}. Dengan demikian, \textit{scheduler}
		memerlukan implementasi \textit{hypercall} spesifik domain agar \textit{user
		space} dapat mengatur informasi partisi pada \textit{scheduler}. Untuk
		selanjutnya, \textit{hypercall} akan mengacu pada \textit{hypercall} spesifik
		\textit{domain} karena \textit{global hypercall} tidak relevan dengan tugas
		akhir ini.
		
	\item Fungsi \textit{scheduling}

		Algoritma \textit{scheduling} pada ARLX diimplementasikan dengan mendefinisikan
		fungsi yang memiliki deklarasi seperti pada
		\autoref{code:do_schedule_declaration}. Fungsi tersebut kemudian akan dipanggil
		apabila \textit{hypervisor} menerima \textit{timer interrupt} atau partisi yang
		sedang mendapatkan waktu CPU telah menghabiskan quantum waktu miliknya.

\end{enumerate}

\begin{figure}[!ht]
\begin{lstlisting}[
	caption={Deklarasi fungsi \textit{scheduling} pada ARLX},
	label=code:do_schedule_declaration
]
struct task_slice (*do_schedule) (const struct scheduler *, s_time_t,
                                  bool_t tasklet_work_scheduled);
\end{lstlisting}
\end{figure}

\subsubsection{Pendefinisian \textit{scheduler}}
\label{section:definisi_scheduler}

Pada Xen, pendefinisian \textit{scheduler} dilakukan dengan menggunakan \textit{interface} yang
telah disediakan~\citep[p.~218]{Chisnall2014}. \autoref{code:struct_scheduler} menampilkan
\textit{interface} tersebut. \textit{Interface} tersebut kemudian didefinisikan sebagai
struktur pada saat \textit{runtime} dengan masing-masing \textit{function pointer} menunjuk pada
sebuah fungsi.

\begin{figure}[!ht]
\begin{lstlisting}[
	caption={\textit{Interface} dari fungsi \textit{scheduler}},
	label=code:struct_scheduler
]
struct scheduler {
    char ∗name ;
    char ∗opt name ;
    unsigned int sched id ;

    void         (*free_domdata)   (const struct scheduler *, void *);
    void *       (*alloc_domdata)  (const struct scheduler *, struct domain *);
    int          (*init_domain)    (const struct scheduler *, struct domain *);
    void         (*destroy_domain) (const struct scheduler *, struct domain *);

    struct task_slice (*do_schedule) (const struct scheduler *, s_time_t,

    int          (*adjust)         (const struct scheduler *, struct domain *,
    int          (*adjust_global)  (const struct scheduler *,
};
\end{lstlisting}
\end{figure}

Setiap fungsi yang ditunjuk oleh \textit{function pointer} pada struktur
tersebut akan berlaku sebagai \textit{handler} yang akan dipanggil oleh \textit{hypervisor}
apabila terjadi \textit{event} yang bersesuaian. Asosiasi antara \textit{handler} dan
\textit{event} telah ditentukan sebelumnya. Setiap \textit{handler} pada struktur tersebut
kemudian diisi dengan alamat dari fungsi yang nantinya akan menangani permintaan pada
\textit{scheduler}. Setiap \textit{scheduler} diharuskan mengisi \textit{handler}
\code{do\_schedule()}. Apabila \textit{scheduler} tidak memerlukan sebuah \textit{handler}, maka
\textit{handler} dapat diisi dengan nilai \code{NULL}.

Pada saat \textit{boot}, konfigurasi \textit{scheduler} yang diinginkan dapat dipilih dengan
menambahkan argumen "sched=\{scheduler\}". \textit{Hypervisor} kemudian akan mencari
\textit{scheduler} yang didefinisikan dengan nilai \code{opt\_name} sama dengan "scheduler".

Untuk mengimplementasikan \textit{scheduler}, \textit{function pointer} \code{do\_schedule()}
harus diisi dengan alamat dari fungsi yang akan melakukan pemilihan partisi. Pada implementasi
\textit{primary-backup partition scheduler}, \textit{function pointer} tersebut akan diisi
dengan alamat dari fungsi \code{a653sched\_do\_schedule()} yang akan dipaparkan pada
\autoref{section:scheduling_impl}. Agar \textit{hypervisor} dapat mengenali \textit{scheduler}
yang diimplementasikan, \code{opt\_name} akan diberi nilai "arinc653pb". Dengan demikian,
\textit{primary-backup partition scheduler} dapat digunakan dengan cara memberikan parameter
"sched=arinc653pb" pada \textit{hypervisor} ketika \textit{boot}.

\subsection{Daftar Jadwal}

Pada implementasi \textit{primary-backup partition scheduler}, daftar jadwal akan
merepresentasikan layanan yang harus dijadwalkan pada rentang waktu tertentu. Masing-masing
layanan akan memiliki daftar partisi yang dapat menyediakan layanan tersebut dan memiliki durasi
layanan tersebut akan disediakan. Pemodelan daftar jadwal yang baru dapat dilihat pada
\autoref{code:daftar_jadwal}

\begin{figure}[!ht]
\begin{lstlisting}[
	caption={Struktur untuk menyimpan daftar jadwal},
	label=code:daftar_jadwal
]
typedef struct sched_providers_s
{
    xen_domain_handle_t dom_handle;
    int                 vcpu_id;
    struct vcpu *       vc;
} sched_providers_t;

typedef struct sched_entry_s
{
    int service_id;
    s_time_t runtime;
    unsigned int num_providers;
    sched_providers_t providers[ARINC653_MAX_DOMAINS_PER_SERVICE];
} sched_entry_t;
\end{lstlisting}
\end{figure}


Pada saat sistem baru dimulai, maka sistem akan menggunakan jadwal tiruan yang akan menjalankan
dom0 secara terus menerus. Daftar jadwal dapat diberikan dengan menggunakan \textit{global
hypercall}.  Untuk mendukung perubahan jadwal, \textit{global hypercall} akan diperbarui agar
dapat menerima pemodelan data baru. Modifikasi \textit{global hypercall} dapat dilihat pada
\autoref{code:hypercall_daftar_jadwal}.

\begin{figure}[!ht]
\begin{lstlisting}[
	caption={Struktur data untuk pemanggilan \textit{global hypercall}},
	label=code:hypercall_daftar_jadwal
]
struct xen_sysctl_arinc653_schedule {
    uint64_aligned_t     major_frame;
    uint8_t     num_sched_entries;
    struct {
        int service_id;
        uint64_aligned_t runtime;
        uint8_t num_providers;
        struct {
            xen_domain_handle_t dom_handle;
            unsigned int vcpu_id;
        } service_providers[ARINC653_MAX_DOMAINS_PER_SERVICE];
    } sched_entries[ARINC653_MAX_SERVICES_PER_SCHEDULE];
};
\end{lstlisting}
\end{figure}

Setelah melakukan \textit{hypercall}, \textit{scheduler} akan mengganti daftar jadwal yang
sebelumnya dengan daftar jadwal yang baru.

\subsubsection{Informasi Partisi}
\label{section:informasi_partisi_impl}

\begin{figure}[!ht]
\begin{lstlisting}[
	caption={Struktur untuk menyimpan informasi partisi},
	label=code:struct_a653sched_domain_t
]
typedef struct a653sched_domain_s {
    bool healthy;
} a653sched_domain_t;
\end{lstlisting}
\end{figure}

Pada implementasi \textit{primary-backup partition scheduler}, informasi partisi akan disimpan
pada struktur data seperti yang ditampilkan oleh \autoref{code:struct_a653sched_domain_t}.
\textit{Scheduler} dapat mengetahui keadaan partisi apabila partisi tersebut memberikan
informasi tentang dirinya melalui \textit{hypercall} yang akan didefinisikan pada
\autoref{section:modul_public_hypercall}.

\begin{figure}[!hb]
\begin{lstlisting}[
	caption={Fungsi inisialisasi partisi},
	label=code:init_domain
]
static int
a653sched_init_domain(const struct scheduler *ops,
                     struct domain *dom)
{
    a653sched_domain_t *sdom;

    sdom = xzalloc(a653sched_domain_t);
    if ( sdom == NULL )
        return -ENOMEM;

    sdom->healthy = true;

    dom->sched_priv = sdom;

    return 0;
}
\end{lstlisting}
\end{figure}

Pada saat pembuatan partisi, \textit{scheduler} akan memanggil fungsi yang ditunjuk oleh
\textit{function pointer} \code{init\_domain} pada struktur yang telah dijelaskan pada
\autoref{section:definisi_scheduler}. Tujuan fungsi tersebut adalah untuk mengalokasikan memori
serta menginisialisasi informasi partisi pada saat dibuat. Fungsi tersebut diimplementasikan
seperti pada \autoref{code:init_domain}. Setelah dibuat, informasi partisi akan memiliki nilai
awal yang menandakan bahwa partisi tersebut sehat.


\begin{figure}[!ht]
\begin{lstlisting}[
	caption={Fungsi penghapusan partisi},
	label=code:destroy_domain
]
static void
a653sched_destroy_domain(const struct scheduler *ops, struct domain *dom)
{
    dom->sched_priv = NULL;

    xfree(dom->sched_priv);
}
\end{lstlisting}
\end{figure}

Pada saat penghapusan partisi, \textit{scheduler} akan memanggil fungsi yang ditunjuk oleh
\textit{function pointer} \code{destroy\_domain} pada struktur yang telah dijelaskan pada
\autoref{section:definisi_scheduler}. Tujuan fungsi tersebut adalah untuk dealokasi memori yang
digunakan oleh informasi partisi yang tersimpan pada \textit{scheduler}.

Implementasi fungsi penghapusan informasi partisi dapat dilihat pada
\autoref{code:destroy_domain}. Dealokasi memori akan membuat pengaksesan memori tersebut
mengakibatkan kegagalan pada \textit{hypervisor} sehingga \textit{hypervisor} akan berhenti
bekerja. Karena itu, memori tempat penyimpanan struktur informasi partisi milik partisi yang
dihapus tidak boleh diakses lagi. Fungsi tersebut hanya akan dipanggil apabila terdapat
\textit{interrupt} yang meminta agar \textit{domain} tersebut dihapus. Maka, penggunaan
informasi partisi sebaiknya dimulai dengan permintaan \textit{lock} dan mematikan
\textit{interrupt}, kemudian pada saat ingin melakukan pembacaan maupun penulisan pada
informasi, \textit{scheduler} perlu melakukan pengecekan nilai \textit{pointer} untuk memastikan
informasi partisi masih valid, dan diakhiri dengan pelepasan \textit{lock} dan menyalakan
\textit{interrupt}.

\subsubsection{Fungsi \textit{Scheduling}}
\label{section:scheduling_impl}

Algoritma \textit{scheduling} yang akan digunakan pada \textit{primary-backup partition
scheduler} memiliki banyak kesamaan dengan algoritma \textit{scheduling} yang digunakan oleh
\textit{scheduler} ARINC 653 pada ARLX. Dengan demikian, implementasi algoritma
\textit{scheduling} akan menggunakan fungsi yang sudah ada pada \textit{scheduler} ARINC 653
yang kemudian dimodifikasi sehingga bekerja sebagaimana telah dipaparkan pada
\autoref{section:algoritma_scheduling}. Modifikasi yang dilakukan dapat dilihat pada
\autoref{code:fungsi_scheduling}.

Karena fungsi \textit{scheduling} akan dipanggil setiap kali sebuah partisi menghabiskan kuantum
waktu atau \textit{hypervisor} mendapatkan \textit{timer interrupt}, keadaan algoritma pada
fungsi \textit{scheduling} harus bersifat tetap pada setiap pemanggilan. Fungsi
\textit{scheduling} ARINC 653 pada ARLX memiliki kompleksitas waktu $O(1)$ untuk setiap
pemanggilan. Untuk menjaga efisiensi kinerja sistem secara keseluruhan, algoritma
\textit{scheduling} pada \textit{primary-backup partition scheduler} diimplementasikan
sedemikian sehingga memiliki kompleksitas waktu yang sama dengan algoritma \textit{scheduling}
pada \textit{scheduler} ARINC 653.

\begin{figure}
\begin{lstlisting}[
	caption={Implementasi algoritma \textit{scheduling}},
	label=code:fungsi_scheduling,
	linerange={balgo-ealgo},
	escapeinside={(*@}{@*)}
]
static struct task_slice
a653sched_do_schedule(
    const struct scheduler *ops,
    s_time_t now,
    bool_t tasklet_work_scheduled)
{
    { balgo }
    (*@{$\cdots$}@*)
    static sched_entry_t * entry;

    if ( sched_priv->num_schedule_entries < 1 )
        sched_priv->next_major_frame = now + DEFAULT_TIMESLICE;
    else if ( now >= sched_priv->next_major_frame )
    {
        sched_index = 0;
        entry = sched_priv->schedule;

        sched_priv->next_major_frame = now + sched_priv->major_frame;
        next_switch_time = now + entry->runtime;
	current_provider = providers_candidate(entry);
    }
    else
    {
        while ( (now >= next_switch_time)
                && (sched_index < sched_priv->num_schedule_entries) )
        {
            /* time to switch to the next domain in this major frame */
            sched_index++;
            entry = sched_priv->schedule + sched_index;
            next_switch_time += entry->runtime;
        }

        /* Choose the most appropriate provider for current entry */
        current_provider = providers_candidate(entry);
    }

    if ( current_provider != NULL )
        new_task = current_provider->vc;
    (*@{$\cdots$}@*)
    { ealgo }
}
\end{lstlisting}
\end{figure}

\subsection{Modul \textit{domain hypercall}}
\label{section:modul_public_hypercall}

Subbab ini akan membahas implementasi dan modifikasi pada modul \textit{domain hypercall} agar
dapat mengimplementasikan \textit{hypercall} sebagai metode komunikasi dan pemindahan data
antara \textit{user space} dengan \textit{hypervisor} terkait informasi partisi seperti yang
sudah dibahas pada \autoref{section:informasi_partisi}.

\begin{figure}
\begin{lstlisting}[
	caption={Data struktur untuk komunikasi yang dapat dimengerti \newline \textit{scheduler} dan \textit{user space}},
	label=code:struct_xen_domctl_sched_arinc653
]
typedef struct xen_domctl_sched_arinc653 {
    uint8_t healthy;
} xen_domctl_sched_arinc653_t;
\end{lstlisting}
\end{figure}

\begin{figure}
\begin{lstlisting}[
	caption={Data struktur untuk pemindahan data pada saat melakukan \textit{domain hypercall}},
	label=code:struct_xen_domctl_scheduler_op
]
struct xen_domctl_scheduler_op {
    uint32_t sched_id;
    uint32_t cmd;

    union {
        xen_domctl_sched_credit_t credit;
        xen_domctl_sched_credit2_t credit2;
        xen_domctl_sched_rtds_t rtds;
        xen_domctl_sched_arinc653_t arinc653;
        struct {
            XEN_GUEST_HANDLE_64(xen_domctl_schedparam_vcpu_t) vcpus;
            uint32_t nr_vcpus;
            uint32_t padding;
        } v;
    } u;
};
\end{lstlisting}
\end{figure}

Implementasi \textit{hypercall} dilakukan dengan mendeklarasikan struktur seperti pada
\autoref{code:struct_xen_domctl_sched_arinc653}. Deklarasi struktur tersebut bertujuan agar
terdapat \textit{interface} struktur data yang dapat dimengerti oleh \textit{scheduler} dan
\textit{user space} sehingga keduanya dapat berkomunikasi. Struktur tersebut kemudian menjadi
salah satu \textit{union member} pada struktur generik yang digunakan oleh \textit{Xen} untuk
pemindahan data. Struktur tersebut dapat dilihat pada
\autoref{code:struct_xen_domctl_scheduler_op}. Metode transfer informasi partisi menggunakan
struktur tersebut akan dipaparkan pada \autoref{section:modul_libxc}.

\subsection{Modul libxc}
\label{section:modul_libxc}

Subbab ini akan membahas implementasi dan modifikasi pada modul libxc, yaitu modul yang
menangani komunikasi antara \textit{hypervisor} dengan \textit{user space} pada
\textit{low-level}. Fungsi-fungsi pada libxc akan dikompilasi menjadi sebuah \textit{library}
milik sistem sehingga seluruh \textit{tools} pada \textit{user space} dapat menggunakan
fungsi-fungsi yang terdapat pada libxc. Modifikasi pada modul libxc diperlukan agar \textit{user
space} mengetahui cara untuk berkomunikasi dengan \textit{scheduler} yang akan dibuat.

Agar \textit{primary-backup partition scheduling} dapat bekerja, \textit{libxc} harus dapat
berkomunikasi dua arah dengan \textit{hypervisor}. Komunikasi dilakukan dengan mengirimkan
struktur \autoref{code:struct_xen_domctl_scheduler_op} melalui \textit{hypercall}.
\textit{Member} \code{sched\_id} pada struktur tersebut diberi nilai yang menandakan bahwa
struktur yang dikirimkan melalui \textit{hypercall} tersebut ditujukan untuk komponen
\textit{scheduler}. Kemudian, \textit{member} \code{cmd} akan diberi nilai yang menandakan
apakah \textit{user space} akan mengirimkan informasi partisi kepada \textit{scheduler} atau
meminta informasi partisi dari \textit{scheduler}.

Implementasi fungsi untuk mengirimkan informasi partisi kepada \textit{scheduler} oleh \textit{user
space} dapat dilihat pada \autoref{code:xc_sched_sysctl_set} sedangkan fungsi untuk meminta
informasi dari \textit{scheduler} oleh \textit{user space} dapat dilihat pada
\autoref{code:xc_sched_sysctl_get}.

\subsubsection{Fungsi Pengiriman Informasi Partisi}

\begin{figure}[!ht]
\begin{lstlisting}[
	caption={Fungsi yang digunakan \textit{user space} mengirim informasi \newline kepada \textit{scheduler}},
	label=code:xc_sched_sysctl_set
]
int
xc_sched_arinc653_domain_set(
    xc_interface *xch,
    uint32_t domid,
    struct xen_domctl_sched_arinc653 *sdom)
{
    int rc;
    DECLARE_DOMCTL;

    domctl.cmd = XEN_DOMCTL_scheduler_op;
    domctl.domain = (domid_t) domid;
    domctl.u.scheduler_op.sched_id = XEN_SCHEDULER_ARINC653;
    domctl.u.scheduler_op.cmd = XEN_DOMCTL_SCHEDOP_putinfo;
    domctl.u.scheduler_op.u.arinc653 = *sdom;

    rc = do_domctl(xch, &domctl);

    return rc;
}
\end{lstlisting}
\end{figure}

Fungsi untuk mengirimkan informasi partisi kepada \textit{scheduler} membutuhkan identifikasi
partisi dan struktur yang akan dikirim melalui \textit{hypercall}. Identifikasi partisi harus
berupa nilai yang valid (terdapat partisi yang memiliki identifikasi dengan nilai tersebut) agar
pemanggilan \textit{hypercall} berhasil. Struktur yang dikirim adalah struktur yang dapat
dimengerti oleh \textit{scheduler} seperti yang telah ditampilkan pada
\autoref{code:struct_xen_domctl_scheduler_op}.

Fungsi pada \autoref{code:xc_sched_sysctl_set} diimplementasikan dengan meminta alokasi memori
pada \textit{hypervisor} yang berkorespondensi dengan memori pada \textit{user space}.
Permintaan tersebut dilakukan dengan menggunakan \textit{macro} \code{DECLARE\_DOMCTL}. Dengan
menggunakan hasil dari alokasi memori, \textit{hypervisor} dapat menyalin isi dari memori pada
\textit{user space} menjadi isi dari memori korespondensinya pada \textit{hypervisor}.

\subsubsection{Fungsi Permintaan Informasi Partisi}
\label{section:fungsi_permintaan_informasi_partisi}

\begin{figure}[!ht]
\begin{lstlisting}[
	caption={Fungsi yang digunakan \textit{user space} meminta informasi \newline partisi dari \textit{scheduler}},
	label=code:xc_sched_sysctl_get
]
int
xc_sched_arinc653_domain_get(
    xc_interface *xch,
    uint32_t domid,
    struct xen_domctl_sched_arinc653 *sdom)
{
    int rc;
    DECLARE_DOMCTL;

    domctl.cmd = XEN_DOMCTL_scheduler_op;
    domctl.domain = (domid_t) domid;
    domctl.u.scheduler_op.sched_id = XEN_SCHEDULER_ARINC653;
    domctl.u.scheduler_op.cmd = XEN_DOMCTL_SCHEDOP_getinfo;

    rc = do_domctl(xch, &domctl);

    if ( rc == 0 )
        *sdom = domctl.u.scheduler_op.u.arinc653;

    return rc;
}
\end{lstlisting}
\end{figure}

Fungsi untuk meminta informasi partisi kepada \textit{scheduler} membutuhkan identifikasi
partisi dan memori yang akan digunakan untuk menyimpan informasi partisi yang didapat melalui
\textit{hypercall}. Identifikasi partisi harus berupa nilai yang valid (terdapat partisi yang
memiliki identifikasi dengan nilai tersebut) agar pemanggilan \textit{hypercall} berhasil.
Ukuran dari memori yang digunakan untuk melakukan \textit{hypercall} harus sama dengan ukuran
struktur pada \autoref{code:struct_xen_domctl_scheduler_op}.

Fungsi tersebut diimplementasikan seperti pada \autoref{code:xc_sched_sysctl_get}. Secara umum
implementasi fungsi tersebut sama seperti impementasi fungsi permintaan informasi partisi
seperti pada \autoref{section:fungsi_permintaan_informasi_partisi}. Perbedaannya hanya pada
nilai yang digunakan oleh \textit{hypervisor} untuk menentukan arah komunikasi
\textit{hypercall}.

\subsection{Modul libxl}
\label{section:modul_libxl}

Subbab ini akan membahas implementasi dan modifikasi pada modul libxl, yaitu modul yang
menangani konfigurasi \textit{domain} pada \textit{high-level}. Libxl digunakan untuk membangun
kakas xl yang digunakan oleh administrator sistem untuk mengubah konfigurasi terkait dengan
\textit{hypervisor} melalui \textit{user space}. Fungsionalitas yang ditawarkan libxl pada
umumnya menggunakan fungsi-fungsi yang ditawarkan oleh libxc. Dengan demikian, administrator
dapat dengan mudah melakukan \textit{hypercall} tanpa perlu membuat program lain.

\begin{figure}[!h]
\begin{lstlisting}[
	caption={xl berkomunikasi dengan menggunakan fungsi milik xc},
	label=code:xl_domain_getset
]
static int sched_arinc653_domain_get(libxl__gc *gc, uint32_t domid,
                                   libxl_domain_sched_params *scinfo)
{
    struct xen_domctl_sched_arinc653 sdom;
    int rc;

    rc = xc_sched_arinc653_domain_get(CTX->xch, domid, &sdom);

    libxl_domain_sched_params_init(scinfo);
    scinfo->sched = LIBXL_SCHEDULER_ARINC653;
    scinfo->parent = sdom.parent;
    scinfo->healthy = sdom.healthy;

    return rc;
}

static int sched_arinc653_domain_set(libxl__gc *gc, uint32_t domid,
                                     const libxl_domain_sched_params *scinfo)
{
    struct xen_domctl_sched_arinc653 sdom;
    int rc;

    rc = xc_sched_arinc653_domain_get(CTX->xch, domid, &sdom);
    if (rc != 0) {
        LOGE(ERROR, "getting domain sched arinc653");
        return ERROR_FAIL;
    }

    if (scinfo->parent != LIBXL_DOMAIN_SCHED_PARAM_PARENT_DEFAULT)
        sdom.parent = scinfo->parent;
    sdom.healthy = scinfo->healthy;
    rc = xc_sched_arinc653_domain_set(CTX->xch, domid, &sdom);

    return rc;
 }
\end{lstlisting}
\end{figure}

Modifikasi yang dilakukan pada libxl adalah menambahkan opsi pada xl sehingga administrator
sistem dapat mengubah informasi mengenai partisi dan memanggil fungsi yang digunakan untuk
berkomunikasi dengan \textit{hypervisor} melalui \textit{user space}. libxl akan memanggil kode
libxc seperti pada \autoref{code:xl_domain_getset}. Kode tersebut akan memanggil fungsi-fungsi
yang telah dibahas sebelumnya pada \autoref{section:modul_libxc}.

\section{Pengujian}

Pengujian dilakukan dengan menggunakan perangkat lunak yang melakukan pengukuran keandalan serta
\textit{timing} seperti yang telah dipaparkan pada \autoref{section:rancangan_pengujian}.
\textit{Platform} yang digunakan untuk melakukan pengujian sama dengan \textit{platform} yang
telah dipaparkan \autoref{section:implementasi}. Pengujian dilakukan pada Laptop
Lenovo\textsuperscript{®} G40 Series. Laptop tersebut menggunakan Intel\textsuperscript{®} Core™
i5.

Hasil pengujian akan ditampilkan dengan menggunakan dua buah \textit{scatter plot}.
\textit{Scatter plot} pertama akan menampilkan hubungan antara waktu sekarang dengan $delta$,
yaitu selisih waktu antara dua buah \textit{timing} yang berurutan. \textit{Scatter plot} kedua
akan menampilkan hubungan antara waktu sekarang dengan $galat$, yaitu selisih waktu antara
$delta$ dengan nilai ekspektasi. Secara formal, perhitungan $delta$ dapat dilihat pada
\autoref{eq:delta} dan perhitungan $galat$ dapat dilihat pada \autoref{eq:galat}.

\begin{equation}\label{eq:delta}
	delta = now - last
\end{equation}
Pada perhitungan nilai $delta$, nilai $now$ adalah waktu pada saat hasil \textit{timing}
sekarang didapatkan dan nilai $last$ adalah waktu pada saat hasil \textit{timing} sebelumnya
didapatkan. Nilai $delta$ berguna untuk melihat selisih antara dua buah \textit{timing} serta
melihat arah dari perbedaan tersebut.

\begin{equation}\label{eq:galat}
	galat = delta - expected
\end{equation}
Pada perhitungan nilai $galat$, nilai $delta$ didapat melalui \autoref{eq:delta} dan nilai
$expected$ adalah nilai ekspektasi \textit{timing}. Nilai ekspektasi yang digunakan dalam
perhitungan $galat$ adalah periode dari \textit{timing}. Nilai $galat$ berguna untuk melihat
seberapa jauh penyimpangan antara dua buah \textit{timing} yang didapat.

\subsection{Pengujian Keandalan}

Pengujian keandalan akan dilakukan dengan menggunakan skenario-skenario berikut.

\begin{enumerate}
	\item Skenario 1

		Pada skenario ini, daftar jadwal berisi 1 layanan dan memiliki 1 partisi
		penyedia. Masing-masing akan berjalan selama \SI{20}{\milli\second} dengan
		\textit{major time frame} keseluruhan selama \SI{20}{\milli\second}.

		Hasil pengujian dengan menggunakan skenario ini akan digunakan sebagai referensi
		untuk \textit{latency} maksimal terkecil yang mungkin dialami oleh sebuah
		layanan pada pengujian ini. Perlu diperhatikan karena pengujian akan menggunakan
		protokol TCP seperti yang telah dipaparkan pada
		\autoref{section:rancangan_pengujian}, maka \textit{latency} maksimal terkecil
		yang mungkin dialami oleh sebuah layanan seharusnya lebih kecil. Namun, apabila
		layanan memang membutuhkan I/O, maka hasil pengujian dengan menggunakan skenario
		ini juga dapat dijadikan referensi pada saat pembuatan desain layanan.

	\item Skenario 2

		Pada skenario ini, daftar jadwal berisi 1 layanan dan memiliki 2 partisi
		penyedia. Masing-masing akan berjalan selama \SI{20}{\milli\second} dengan
		\textit{major time frame} keseluruhan selama \SI{20}{\milli\second}.

		Hasil pengujian dengan menggunakan skenario akan digunakan untuk melihat
		kinerja \textit{scheduler} dalam menerapkan skema \textit{primary-backup}.
		Skenario akan membuat partisi \textit{primary} gagal dalam rentang waktu
		tertentu. Data yang terkumpul pada saat pergantian partisi penyedia layanan dari
		partisi \textit{primary} menjadi partisi \textit{backup} akan menunjukkan waktu
		pergantian serta \textit{latency} layanan tersebut apabila terjadi pergantian.

	\item Skenario 3

		Pada skenario ini, daftar jadwal berisi 9 layanan dan memiliki 1 partisi
		penyedia. Masing-masing akan berjalan selama \SI{20}{\milli\second} dengan
		\textit{major time frame} keseluruhan selama \SI{180}{\milli\second}.

		Skenario ini dilakukan untuk melihat kinerja sistem apabila terdapat banyak
		partisi tanpa pergantian. Sama seperti skenario 1, skenario ini hanya digunakan
		sebagai referensi untuk \textit{latency} maksimal terkecil yang mungkin dialami
		oleh sebuah layanan. Hasil pengujian dengan menggunakan skenario ini akan
		memperlihatkan hasil yang berbeda dari hasil pengujian apabila terdapat perbedaan
		kinerja sistem pada saat terdapat banyak layanan.

	\item Skenario 4

		Pada skenario ini, daftar jadwal berisi 2 layanan yang sama dan memiliki 1
		partisi penyedia. Karena layanan tersebut merupakan layanan yang sama, maka
		partisi penyedia layanan 1 dan layanan 2 merupakan partisi yang sama.  Partisi
		akan berjalan selama \SI{20}{\milli\second} pada masing-masing layanan.
		\textit{Major time frame} keseluruhan pengujian adalah \SI{40}{\milli\second}.

		Skenario ini dilakukan untuk mengukur waktu tambahan pada \textit{latency} yang
		diakibatkan oleh pergantian partisi. Apabila terdapat perbedaan antara hasil
		pengujian pada saat menggunakan skenario ini dengan hasil pengujian pada
		skenario dengan lebih dari satu partisi, maka dapat disimpulkan bahwa pergantian
		partisi pada Xen akan menambah \textit{latency}. Meski secara sekilas skenario
		ini digunakan hanya untuk mengukur \textit{latency}, namun hasil pengujian
		dengan menggunakan skenario ini akan dibandingkan dengan skenario pengujian
		keandalan lain. Dengan demikian, skenario ini dimasukkan sebagai pengujian
		keandalan.

	\item Skenario 5

		Pada skenario ini, daftar jadwal berisi 4 layanan dan masing-masing memiliki 2
		partisi penyedia. Masing-masing akan berjalan selama \SI{10}{\milli\second}
		dengan \textit{major time frame} keseluruhan selama \SI{50}{\milli\second}
		(terdapat \SI{10}{\milli\second} partisi \textit{idle}).

		Hasil skenario ini digunakan untuk melihat kinerja \textit{scheduler} dalam
		menerapkan skema \textit{primary-backup} pada saat sistem memiliki banyak
		layanan. Partisi \textit{primary} pada masing-masing layanan akan digagalkan
		secara berkala. Pemrosesan data akan sama dengan skenario 2 hanya saja pada
		beberapa layanan yang berjalan pada saat yang bersamaan.

	\item Skenario 6

		Pada skenario ini, daftar jadwal berisi 3 layanan. Layanan 1 memiliki 2 partisi
		penyedia dan layanan akan berjalan selama \SI{30}{\milli\second}. Layanan 2
		memiliki 3 partisi penyedia dan layanan akan berjalan selama
		\SI{20}{\milli\second}. Layanan 3 memiliki 1 buah partisi penyedia dan layanan
		akan berjalan selama \SI{10}{\milli\second}. \textit{Major time frame}
		keseluruhan pengujian adalah \SI{60}{\milli\second}.  

		Skenario ini dilakukan untuk melihat kinerja sistem pada kasus umum. Selain itu,
		skenario juga akan menguji \textit{scheduler} apabila durasi setiap layanan
		berbeda-beda.

\end{enumerate}

Hasil pengujian keandalan sistem berdasarkan skenario-skenario pengujian akan dipaparkan.
Pengujian diambil dengan menjalankan program pengujian selama 2 jam. Namun, hasil pengujian
hanya akan ditampilkan sampai terlihat sebuah kecenderungan pada data.

\begin{enumerate}

	\item Skenario 1
		
		\begin{figure}[!ht]
			\centering
			\begin{tabular}{c}
				\subfloat{
					\tikzsetnextfilename{reliability-s1-delta}
					\begin{tikzpicture}
						\begin{axis}[
							height=4cm, width=9cm,
							title={Timing Plot},
							scaled ticks=false,
							xlabel={Waktu (s)},
							ylabel={Delta (s)},
							grid=major,
							xlabel near ticks,
							ylabel near ticks,
							yticklabel style={
								/pgf/number format/precision=3,
								/pgf/number format/sci,
								/pgf/number format/sci zerofill,
							}
							]
							\addplot[only marks, mark=o, mark size=2.0,color=blue] %
								table[x=lastcall,y=delta,col
								sep=comma]{./data/master-slave/schedule-domU.json-cannot-failure.csv};
							\addlegendentry{Partisi 1};
						\end{axis}
					\end{tikzpicture}
				} \\
				\subfloat{
					\tikzsetnextfilename{reliability-s1-error}
					\begin{tikzpicture}
						\begin{axis}[
							height=4cm, width=9cm,
							title={Timing Plot},
							scaled ticks=false,
							xlabel={Waktu (s)},
							ylabel={Galat (s)},
							grid=major,
							xlabel near ticks,
							ylabel near ticks,
							yticklabel style={
								/pgf/number format/precision=3,
								/pgf/number format/sci,
								/pgf/number format/sci zerofill,
							}
							]
							\addplot[only marks, mark=o, mark size=2.0,color=blue] %
								table[x=lastcall,y=error,col
								sep=comma]{./data/master-slave/schedule-domU.json-cannot-failure.csv};
							\addlegendentry{Partisi 1};
						\end{axis}
					\end{tikzpicture}
				}
			\end{tabular}
			\caption{Hasil pengujian keandalan sistem menggunakan skenario 1}
			\label{figure:hasil_keandalan_s1}
		\end{figure}

		Hasil pengujian keandalan dengan menggunakan skenario 1 dapat dilihat pada
		\autoref{figure:hasil_keandalan_s1}.  Pengujian dengan menggunakan skenario 1
		menghasilkan \textit{timing} yang sangat presisi, dengan galat \textit{timing}
		tertinggi dari ekspektasi periode \textit{timing} yang seharusnya hanya sekitar
		\SI{100}{\micro\second}. Pada pengujian ini, perilaku algoritma
		\textit{primary-backup partition scheduling} tidak dapat terlihat. Namun,
		pengujian ini dapat digunakan sebagai referensi \textit{timing} yang seharusnya
		didapatkan pada pengujian-pengujian berikutnya.

	\item Skenario 2

		\begin{figure}[!ht]
			\centering
			\begin{tabular}{c}
				\subfloat{
					\tikzsetnextfilename{reliability-s2-delta}
					\begin{tikzpicture}
						\begin{axis}[
							height=4cm, width=9cm,
							title={Timing Plot},
							scaled ticks=false,
							xlabel={Waktu (s)},
							ylabel={Delta (s)},
							grid=major,
							xlabel near ticks,
							ylabel near ticks,
							yticklabel style={
								/pgf/number format/precision=2,
								/pgf/number format/sci,
								/pgf/number format/sci zerofill,
							}
							]
							\addplot[only marks, mark=o, mark size=2.0,color=blue] %
								table[x=lastcall,y=delta,col
								sep=comma]{./data/master-slave/schedule-1p-1b.json-random-failure.csv.xen-vm1};
							\addlegendentry{Partisi 1};
							\addplot[only marks, mark=+, mark size=2.0,color=red] %
								table[x=lastcall,y=delta,col
								sep=comma]{./data/master-slave/schedule-1p-1b.json-random-failure.csv.xen-vm2};
							\addlegendentry{Partisi 2};
						\end{axis}
					\end{tikzpicture}
				} \\
				\subfloat{
					\tikzsetnextfilename{reliability-s2-error}
					\begin{tikzpicture}
						\begin{axis}[
							height=4cm, width=9cm,
							title={Timing Plot},
							scaled ticks=false,
							xlabel={Waktu (s)},
							ylabel={Galat (s)},
							grid=major,
							xlabel near ticks,
							ylabel near ticks,
							yticklabel style={
								/pgf/number format/precision=2,
								/pgf/number format/sci,
								/pgf/number format/sci zerofill,
							}
							]
							\addplot[only marks, mark=o, mark size=2.0,color=blue] %
								table[x=lastcall,y=error,col
								sep=comma]{./data/master-slave/schedule-1p-1b.json-random-failure.csv.xen-vm1};
							\addlegendentry{Partisi 1};
							\addplot[only marks, mark=+, mark size=2.0,color=red] %
								table[x=lastcall,y=error,col
								sep=comma]{./data/master-slave/schedule-1p-1b.json-random-failure.csv.xen-vm2};
							\addlegendentry{Partisi 2};
						\end{axis}
					\end{tikzpicture}
				}
			\end{tabular}
			\caption{Hasil pengujian keandalan sistem menggunakan skenario 2}
			\label{figure:hasil_keandalan_s2}
		\end{figure}

		Hasil pengujian keandalan dengan menggunakan skenario 2 dapat dilihat pada
		\autoref{figure:hasil_keandalan_s2}.  Pada pengujian dengan menggunakan skenario
		2, dapat terlihat bahwa layanan tersedia selama berjalannya program pengujian.
		Ketika partisi 1 mengalami kegagalan, partisi 2 mengambil alih peran partisi 1
		dalam menyediakan layanan tersebut.  Namun, semakin lama program pengujian
		berjalan, maka semakin banyak \textit{timing} yang tidak sesuai dengan
		ekspektasi. \textit{Timing} tersebut didapat lebih cepat dari yang seharusnya.
		Selain itu, dapat dilihat bahwa pada saat pergantian partisi, \textit{timing}
		layanan menjadi semakin cepat.

	\item Skenario 3

		\begin{figure}[!ht]
			\centering
			\begin{tabular}{c}
				\subfloat{
					\tikzsetnextfilename{reliability-s3-delta-p1}
					\begin{tikzpicture}
						\begin{axis}[
							height=4cm, width=9cm,
							title={Timing Plot},
							scaled ticks=false,
							xlabel={Waktu (s)},
							ylabel={Delta (s)},
							grid=major,
							xlabel near ticks,
							ylabel near ticks,
							yticklabel style={
								/pgf/number format/precision=2,
								/pgf/number format/sci,
								/pgf/number format/sci zerofill,
							}
							]
							\addplot[only marks, mark=o, mark size=2.0,color=blue] %
								table[x=lastcall,y=delta,col
								sep=comma]{./data/master-slave/schedule-all.json-no-failure.csv.xen-vm1};
							\addlegendentry{Partisi 1};
						\end{axis}
					\end{tikzpicture}
				} \\
				\subfloat{
					\tikzsetnextfilename{reliability-s3-error-1}
					\begin{tikzpicture}
						\begin{axis}[
							height=4cm, width=9cm,
							title={Timing Plot},
							scaled ticks=false,
							xlabel={Waktu (s)},
							ylabel={Galat (s)},
							grid=major,
							xlabel near ticks,
							ylabel near ticks,
							yticklabel style={
								/pgf/number format/precision=2,
								/pgf/number format/sci,
								/pgf/number format/sci zerofill,
							}
							]
							\addplot[only marks, mark=o, mark size=2.0,color=blue] %
								table[x=lastcall,y=error,col
								sep=comma]{./data/master-slave/schedule-all.json-no-failure.csv.xen-vm1};
							\addlegendentry{Partisi 1};
						\end{axis}
					\end{tikzpicture}
				}
			\end{tabular}
			\caption{Hasil pengujian keandalan sistem menggunakan skenario 3 (layanan 1)}
			\label{figure:hasil_keandalan_s3_service1}
		\end{figure}

		Hasil pengujian keandalan dengan menggunakan skenario 3 dapat dilihat pada
		\autoref{figure:hasil_keandalan_s3_service1},
		\autoref{figure:hasil_keandalan_s3_service2},
		\autoref{figure:hasil_keandalan_s3_service3},
		\autoref{figure:hasil_keandalan_s3_service4},
		\autoref{figure:hasil_keandalan_s3_service5},
		\autoref{figure:hasil_keandalan_s3_service6},
		\autoref{figure:hasil_keandalan_s3_service7},
		\autoref{figure:hasil_keandalan_s3_service8}, dan
		\autoref{figure:hasil_keandalan_s3_service9}.  Hasil pengujian pada
		masing-masing layanan adalah sebagai berikut.

		\begin{enumerate}
			\item Pada layanan 1, partisi berjalan
				dengan cukup stabil. Galat antara maksimal pengukuran \textit{timing} dengan
				\textit{timing} ekspektasi cukup tinggi yaitu \SI{800}{\micro\second}.

				\begin{figure}[!ht]
					\centering
					\begin{tabular}{c}
						\subfloat{
							\tikzsetnextfilename{reliability-s3-delta-p2}
							\begin{tikzpicture}
								\begin{axis}[
									height=4cm, width=9cm,
									title={Timing Plot},
									scaled ticks=false,
									xlabel={Waktu (s)},
									ylabel={Delta (s)},
									grid=major,
									xlabel near ticks,
									ylabel near ticks,
									yticklabel style={
										/pgf/number format/precision=2,
										/pgf/number format/sci,
										/pgf/number format/sci zerofill,
									}
									]
									\addplot[only marks, mark=o, mark size=2.0,color=blue] %
										table[x=lastcall,y=delta,col
										sep=comma]{./data/master-slave/schedule-all.json-no-failure.csv.xen-vm2};
									\addlegendentry{Partisi 2};
								\end{axis}
							\end{tikzpicture}
						} \\
						\subfloat{
							\tikzsetnextfilename{reliability-s3-error-p2}
							\begin{tikzpicture}
								\begin{axis}[
									height=4cm, width=9cm,
									title={Timing Plot},
									scaled ticks=false,
									xlabel={Waktu (s)},
									ylabel={Galat (s)},
									grid=major,
									xlabel near ticks,
									ylabel near ticks,
									yticklabel style={
										/pgf/number format/precision=2,
										/pgf/number format/sci,
										/pgf/number format/sci zerofill,
									}
									]
									\addplot[only marks, mark=o, mark size=2.0,color=blue] %
										table[x=lastcall,y=error,col
										sep=comma]{./data/master-slave/schedule-all.json-no-failure.csv.xen-vm2};
									\addlegendentry{Partisi 2};
								\end{axis}
							\end{tikzpicture}
						}
					\end{tabular}
					\caption{Hasil pengujian keandalan sistem menggunakan skenario 3 (layanan 2)}
					\label{figure:hasil_keandalan_s3_service2}
				\end{figure}

			\item Pada layanan 2, partisi berjalan dengan sangat stabil. Galat antara maksimal pengukuran
				\textit{timing} dengan \textit{timing} ekspektasi sangat rendah yaitu
				\SI{15}{\micro\second}.

				\begin{figure}[!ht]
					\centering
					\begin{tabular}{c}
						\subfloat{
							\tikzsetnextfilename{reliability-s3-delta-p3}
							\begin{tikzpicture}
								\begin{axis}[
									height=4cm, width=9cm,
									title={Timing Plot},
									scaled ticks=false,
									xlabel={Waktu (s)},
									ylabel={Delta (s)},
									grid=major,
									xlabel near ticks,
									ylabel near ticks,
									yticklabel style={
										/pgf/number format/precision=2,
										/pgf/number format/sci,
										/pgf/number format/sci zerofill,
									}
									]
									\addplot[only marks, mark=o, mark size=2.0,color=blue] %
										table[x=lastcall,y=delta,col
										sep=comma]{./data/master-slave/schedule-all.json-no-failure.csv.xen-vm3};
									\addlegendentry{Partisi 3};
								\end{axis}
							\end{tikzpicture}
						} \\
						\subfloat{
							\tikzsetnextfilename{reliability-s3-error-p3}
							\begin{tikzpicture}
								\begin{axis}[
									height=4cm, width=9cm,
									title={Timing Plot},
									scaled ticks=false,
									xlabel={Waktu (s)},
									ylabel={Galat (s)},
									grid=major,
									xlabel near ticks,
									ylabel near ticks,
									yticklabel style={
										/pgf/number format/precision=2,
										/pgf/number format/sci,
										/pgf/number format/sci zerofill,
									}
									]
									\addplot[only marks, mark=o, mark size=2.0,color=blue] %
										table[x=lastcall,y=error,col
										sep=comma]{./data/master-slave/schedule-all.json-no-failure.csv.xen-vm3};
									\addlegendentry{Partisi 3};
								\end{axis}
							\end{tikzpicture}
						}
					\end{tabular}
					\caption{Hasil pengujian keandalan sistem menggunakan skenario 3 (layanan 3)}
					\label{figure:hasil_keandalan_s3_service3}
				\end{figure}

			\item Pada layanan 3, partisi berjalan dengan stabil. Namun, galat maksimal antara
				pengukuran \textit{timing} dengan \textit{timing} ekspektasi sangat tinggi yaitu
				hampir mencapai \SI{150}{\milli\second}.

				\begin{figure}[!ht]
					\centering
					\begin{tabular}{c}
						\subfloat{
							\tikzsetnextfilename{reliability-s3-delta-p4}
							\begin{tikzpicture}
								\begin{axis}[
									height=4cm, width=9cm,
									title={Timing Plot},
									scaled ticks=false,
									xlabel={Waktu (s)},
									ylabel={Delta (s)},
									grid=major,
									xlabel near ticks,
									ylabel near ticks,
									yticklabel style={
										/pgf/number format/precision=2,
										/pgf/number format/sci,
										/pgf/number format/sci zerofill,
									}
									]
									\addplot[only marks, mark=o, mark size=2.0,color=blue] %
										table[x=lastcall,y=delta,col
										sep=comma]{./data/master-slave/schedule-all.json-no-failure.csv.xen-vm4};
									\addlegendentry{Partisi 4};
								\end{axis}
							\end{tikzpicture}
						} \\
						\subfloat{
							\tikzsetnextfilename{reliability-s3-error-p4}
							\begin{tikzpicture}
								\begin{axis}[
									height=4cm, width=9cm,
									title={Timing Plot},
									scaled ticks=false,
									xlabel={Waktu (s)},
									ylabel={Galat (s)},
									grid=major,
									xlabel near ticks,
									ylabel near ticks,
									yticklabel style={
										/pgf/number format/precision=2,
										/pgf/number format/sci,
										/pgf/number format/sci zerofill,
									}
									]
									\addplot[only marks, mark=o, mark size=2.0,color=blue] %
										table[x=lastcall,y=error,col
										sep=comma]{./data/master-slave/schedule-all.json-no-failure.csv.xen-vm4};
									\addlegendentry{Partisi 4};
								\end{axis}
							\end{tikzpicture}
						}
					\end{tabular}
					\caption{Hasil pengujian keandalan sistem menggunakan skenario 3 (layanan 4)}
					\label{figure:hasil_keandalan_s3_service4}
				\end{figure}

			\item Pada layanan 4, partisi berjalan dengan stabil. Namun, galat maksimal antara
				pengukuran \textit{timing} dengan \textit{timing} ekspektasi sangat tinggi yaitu
				hampir mencapai \SI{3}{\milli\second}.

				\begin{figure}[!ht]
					\centering
					\begin{tabular}{c}
						\subfloat{
							\tikzsetnextfilename{reliability-s3-delta-p5}
							\begin{tikzpicture}
								\begin{axis}[
									height=4cm, width=9cm,
									title={Timing Plot},
									scaled ticks=false,
									xlabel={Waktu (s)},
									ylabel={Delta (s)},
									grid=major,
									xlabel near ticks,
									ylabel near ticks,
									yticklabel style={
										/pgf/number format/precision=2,
										/pgf/number format/sci,
										/pgf/number format/sci zerofill,
									}
									]
									\addplot[only marks, mark=o, mark size=2.0,color=blue] %
										table[x=lastcall,y=delta,col
										sep=comma]{./data/master-slave/schedule-all.json-no-failure.csv.xen-vm5};
									\addlegendentry{Partisi 5};
								\end{axis}
							\end{tikzpicture}
						} \\
						\subfloat{
							\tikzsetnextfilename{reliability-s3-error-p5}
							\begin{tikzpicture}
								\begin{axis}[
									height=4cm, width=9cm,
									title={Timing Plot},
									scaled ticks=false,
									xlabel={Waktu (s)},
									ylabel={Galat (s)},
									grid=major,
									xlabel near ticks,
									ylabel near ticks,
									yticklabel style={
										/pgf/number format/precision=2,
										/pgf/number format/sci,
										/pgf/number format/sci zerofill,
									}
									]
									\addplot[only marks, mark=o, mark size=2.0,color=blue] %
										table[x=lastcall,y=error,col
										sep=comma]{./data/master-slave/schedule-all.json-no-failure.csv.xen-vm5};
									\addlegendentry{Partisi 5};
								\end{axis}
							\end{tikzpicture}
						}
					\end{tabular}
					\caption{Hasil pengujian keandalan sistem menggunakan skenario 3 (layanan 5)}
					\label{figure:hasil_keandalan_s3_service5}
				\end{figure}

			\item Pada layanan 5, partisi berjalan dengan sangat stabil. Galat antara maksimal pengukuran
				\textit{timing} dengan \textit{timing} ekspektasi sangat rendah yaitu
				sekitar \SI{13}{\micro\second}.

				\begin{figure}[!ht]
					\centering
					\begin{tabular}{c}
						\subfloat{
							\tikzsetnextfilename{reliability-s3-delta-p6}
							\begin{tikzpicture}
								\begin{axis}[
									height=4cm, width=9cm,
									title={Timing Plot},
									scaled ticks=false,
									xlabel={Waktu (s)},
									ylabel={Delta (s)},
									grid=major,
									xlabel near ticks,
									ylabel near ticks,
									yticklabel style={
										/pgf/number format/precision=2,
										/pgf/number format/sci,
										/pgf/number format/sci zerofill,
									}
									]
									\addplot[only marks, mark=o, mark size=2.0,color=blue] %
										table[x=lastcall,y=delta,col
										sep=comma]{./data/master-slave/schedule-all.json-no-failure.csv.xen-vm6};
									\addlegendentry{Partisi 6};
								\end{axis}
							\end{tikzpicture}
						} \\
						\subfloat{
							\tikzsetnextfilename{reliability-s3-error-p6}
							\begin{tikzpicture}
								\begin{axis}[
									height=4cm, width=9cm,
									title={Timing Plot},
									scaled ticks=false,
									xlabel={Waktu (s)},
									ylabel={Galat (s)},
									grid=major,
									xlabel near ticks,
									ylabel near ticks,
									yticklabel style={
										/pgf/number format/precision=2,
										/pgf/number format/sci,
										/pgf/number format/sci zerofill,
									}
									]
									\addplot[only marks, mark=o, mark size=2.0,color=blue] %
										table[x=lastcall,y=error,col
										sep=comma]{./data/master-slave/schedule-all.json-no-failure.csv.xen-vm6};
									\addlegendentry{Partisi 6};
								\end{axis}
							\end{tikzpicture}
						}
					\end{tabular}
					\caption{Hasil pengujian keandalan sistem menggunakan skenario 3 (layanan 6)}
					\label{figure:hasil_keandalan_s3_service6}
				\end{figure}

			\item Pada layanan 6, partisi berjalan dengan sangat stabil. Galat antara maksimal pengukuran
				\textit{timing} dengan \textit{timing} ekspektasi sangat rendah yaitu
				\SI{13}{\micro\second}.

				\begin{figure}[!ht]
					\centering
					\begin{tabular}{c}
						\subfloat{
							\tikzsetnextfilename{reliability-s3-delta-p7}
							\begin{tikzpicture}
								\begin{axis}[
									height=4cm, width=9cm,
									title={Timing Plot},
									scaled ticks=false,
									xlabel={Waktu (s)},
									ylabel={Delta (s)},
									grid=major,
									xlabel near ticks,
									ylabel near ticks,
									yticklabel style={
										/pgf/number format/precision=2,
										/pgf/number format/sci,
										/pgf/number format/sci zerofill,
									}
									]
									\addplot[only marks, mark=o, mark size=2.0,color=blue] %
										table[x=lastcall,y=delta,col
										sep=comma]{./data/master-slave/schedule-all.json-no-failure.csv.xen-vm7};
									\addlegendentry{Partisi 7};
								\end{axis}
							\end{tikzpicture}
						} \\
						\subfloat{
							\tikzsetnextfilename{reliability-s3-error-p7}
							\begin{tikzpicture}
								\begin{axis}[
									height=4cm, width=9cm,
									title={Timing Plot},
									scaled ticks=false,
									xlabel={Waktu (s)},
									ylabel={Galat (s)},
									grid=major,
									xlabel near ticks,
									ylabel near ticks,
									yticklabel style={
										/pgf/number format/precision=2,
										/pgf/number format/sci,
										/pgf/number format/sci zerofill,
									}
									]
									\addplot[only marks, mark=o, mark size=2.0,color=blue] %
										table[x=lastcall,y=error,col
										sep=comma]{./data/master-slave/schedule-all.json-no-failure.csv.xen-vm7};
									\addlegendentry{Partisi 7};
								\end{axis}
							\end{tikzpicture}
						}
					\end{tabular}
					\caption{Hasil pengujian keandalan sistem menggunakan skenario 3 (layanan 7)}
					\label{figure:hasil_keandalan_s3_service7}
				\end{figure}

			\item Pada layanan 7, partisi berjalan dengan sangat stabil. Galat antara maksimal pengukuran
				\textit{timing} dengan \textit{timing} ekspektasi sangat rendah yaitu
				\SI{15}{\micro\second}.

				\begin{figure}[!ht]
					\centering
					\begin{tabular}{c}
						\subfloat{
							\tikzsetnextfilename{reliability-s3-delta-p8}
							\begin{tikzpicture}
								\begin{axis}[
									height=4cm, width=9cm,
									title={Timing Plot},
									scaled ticks=false,
									xlabel={Waktu (s)},
									ylabel={Delta (s)},
									grid=major,
									xlabel near ticks,
									ylabel near ticks,
									yticklabel style={
										/pgf/number format/precision=2,
										/pgf/number format/sci,
										/pgf/number format/sci zerofill,
									}
									]
									\addplot[only marks, mark=o, mark size=2.0,color=blue] %
										table[x=lastcall,y=delta,col
										sep=comma]{./data/master-slave/schedule-all.json-no-failure.csv.xen-vm8};
									\addlegendentry{Partisi 8};
								\end{axis}
							\end{tikzpicture}
						} \\
						\subfloat{
							\tikzsetnextfilename{reliability-s3-error-p8}
							\begin{tikzpicture}
								\begin{axis}[
									height=4cm, width=9cm,
									title={Timing Plot},
									scaled ticks=false,
									xlabel={Waktu (s)},
									ylabel={Galat (s)},
									grid=major,
									xlabel near ticks,
									ylabel near ticks,
									yticklabel style={
										/pgf/number format/precision=2,
										/pgf/number format/sci,
										/pgf/number format/sci zerofill,
									}
									]
									\addplot[only marks, mark=o, mark size=2.0,color=blue] %
										table[x=lastcall,y=error,col
										sep=comma]{./data/master-slave/schedule-all.json-no-failure.csv.xen-vm8};
									\addlegendentry{Partisi 8};
								\end{axis}
							\end{tikzpicture}
						}
					\end{tabular}
					\caption{Hasil pengujian keandalan sistem menggunakan skenario 3 (layanan 8)}
					\label{figure:hasil_keandalan_s3_service8}
				\end{figure}

			\item Pada layanan 8, partisi berjalan dengan sangat stabil. Galat antara maksimal pengukuran
				\textit{timing} dengan \textit{timing} ekspektasi sangat rendah yaitu
				\SI{15}{\micro\second}.


				\begin{figure}[!ht]
					\centering
					\begin{tabular}{c}
						\subfloat{
							\tikzsetnextfilename{reliability-s3-delta-p9}
							\begin{tikzpicture}
								\begin{axis}[
									height=4cm, width=9cm,
									title={Timing Plot},
									scaled ticks=false,
									xlabel={Waktu (s)},
									ylabel={Delta (s)},
									grid=major,
									xlabel near ticks,
									ylabel near ticks,
									yticklabel style={
										/pgf/number format/precision=2,
										/pgf/number format/sci,
										/pgf/number format/sci zerofill,
									}
									]
									\addplot[only marks, mark=o, mark size=2.0,color=blue] %
										table[x=lastcall,y=delta,col
										sep=comma]{./data/master-slave/schedule-all.json-no-failure.csv.xen-vm9};
									\addlegendentry{Partisi 9};
								\end{axis}
							\end{tikzpicture}
						} \\
						\subfloat{
							\tikzsetnextfilename{reliability-s3-error-p9}
							\begin{tikzpicture}
								\begin{axis}[
									height=4cm, width=9cm,
									title={Timing Plot},
									scaled ticks=false,
									xlabel={Waktu (s)},
									ylabel={Galat (s)},
									grid=major,
									xlabel near ticks,
									ylabel near ticks,
									yticklabel style={
										/pgf/number format/precision=2,
										/pgf/number format/sci,
										/pgf/number format/sci zerofill,
									}
									]
									\addplot[only marks, mark=o, mark size=2.0,color=blue] %
										table[x=lastcall,y=error,col
										sep=comma]{./data/master-slave/schedule-all.json-no-failure.csv.xen-vm9};
									\addlegendentry{Partisi 9};
								\end{axis}
							\end{tikzpicture}
						}
					\end{tabular}
					\caption{Hasil pengujian keandalan sistem menggunakan skenario 3 (layanan 9)}
					\label{figure:hasil_keandalan_s3_service9}
				\end{figure}

			\item Pada layanan 9, partisi berjalan dengan stabil. Namun, galat maksimal antara
				pengukuran \textit{timing} dengan \textit{timing} ekspektasi sangat tinggi yaitu
				hampir mencapai \SI{15}{\milli\second}.

		\end{enumerate}

		Secara umum, partisi berjalan dengan sangat stabil. Namun, pada sistem
		\textit{real-time}, hasil pengukuran terpenting adalah \textit{latency} maksimal
		sebuah \textit{task} selesai dari jadwal \textit{task} tersebut yang seharusnya.
		Maka, dapat dikatakan bahwa hasil pengujian dengan menggunakan skenario 3
		sangatlah buruk.

	\item Skenario 4

		\begin{figure}[!ht]
			\centering
			\begin{tabular}{c}
				\subfloat{
					\tikzsetnextfilename{reliability-s4-delta}
					\begin{tikzpicture}
						\begin{axis}[
							height=4cm, width=9cm,
							title={Timing Plot},
							scaled ticks=false,
							xlabel={Waktu (s)},
							ylabel={Delta (s)},
							grid=major,
							xlabel near ticks,
							ylabel near ticks,
							yticklabel style={
								/pgf/number format/precision=2,
								/pgf/number format/sci,
								/pgf/number format/sci zerofill,
							}
							]
							\addplot[only marks, mark=o, mark size=2.0,color=blue] %
								table[x=lastcall,y=delta,col
								sep=comma]{./data/master-slave/schedule-domU-20-20.json-no-failure.csv};
							\addlegendentry{Partisi 1};
						\end{axis}
					\end{tikzpicture}
				} \\
				\subfloat{
					\tikzsetnextfilename{reliability-s4-error}
					\begin{tikzpicture}
						\begin{axis}[
							height=4cm, width=9cm,
							title={Timing Plot},
							scaled ticks=false,
							xlabel={Waktu (s)},
							ylabel={Galat (s)},
							grid=major,
							xlabel near ticks,
							ylabel near ticks,
							yticklabel style={
								/pgf/number format/precision=2,
								/pgf/number format/sci,
								/pgf/number format/sci zerofill,
							}
							]
							\addplot[only marks, mark=o, mark size=2.0,color=blue] %
								table[x=lastcall,y=error,col
								sep=comma]{./data/master-slave/schedule-domU-20-20.json-no-failure.csv};
							\addlegendentry{Partisi 1};
						\end{axis}
					\end{tikzpicture}
				}
			\end{tabular}
			\caption{Hasil pengujian keandalan sistem menggunakan skenario 4}
			\label{figure:hasil_keandalan_s4}
		\end{figure}

		Hasil pengujian keandalan pada skenario 4 dapat dilihat pada
		\autoref{figure:hasil_keandalan_s4}. Hasil pengujian dengan menggunakan skenario
		4 menunjukkan bahwa apabila tidak terdapat \textit{partition switching}, maka
		\textit{timing} yang didapat cukup presisi.  Namun, pada beberapa waktu tertentu
		tetap terdapat galat yang cukup besar pada perbedaan \textit{timing} yang
		didapat dengan \textit{timing} ekspektasi.

	\item Skenario 5

		\begin{figure}[!ht]
			\centering
			\begin{tabular}{c}
				\subfloat{
					\tikzsetnextfilename{reliability-s5-delta-p1}
					\begin{tikzpicture}
						\begin{axis}[
							height=4cm, width=9cm,
							title={Timing Plot},
							scaled ticks=false,
							xlabel={Waktu (s)},
							ylabel={Delta (s)},
							grid=major,
							xlabel near ticks,
							ylabel near ticks,
							yticklabel style={
								/pgf/number format/precision=2,
								/pgf/number format/sci,
								/pgf/number format/sci zerofill,
							}
							]
							\addplot[only marks, mark=o, mark size=2.0,color=blue] %
								table[x=lastcall,y=delta,col
								sep=comma]{./data/master-slave/schedule-4p-4b.json-random-failure.csv.xen-vm1};
							\addlegendentry{Partisi 1};
							\addplot[only marks, mark=+, mark size=2.0,color=red] %
								table[x=lastcall,y=delta,col
								sep=comma]{./data/master-slave/schedule-4p-4b.json-random-failure.csv.xen-vm5};
							\addlegendentry{Partisi 5};
						\end{axis}
					\end{tikzpicture}
				} \\
				\subfloat{
					\tikzsetnextfilename{reliability-s5-error-p1}
					\begin{tikzpicture}
						\begin{axis}[
							height=4cm, width=9cm,
							title={Timing Plot},
							scaled ticks=false,
							xlabel={Waktu (s)},
							ylabel={Galat (s)},
							grid=major,
							xlabel near ticks,
							ylabel near ticks,
							yticklabel style={
								/pgf/number format/precision=2,
								/pgf/number format/sci,
								/pgf/number format/sci zerofill,
							}
							]
							\addplot[only marks, mark=o, mark size=2.0,color=blue] %
								table[x=lastcall,y=error,col
								sep=comma]{./data/master-slave/schedule-4p-4b.json-random-failure.csv.xen-vm1};
							\addlegendentry{Partisi 1};
							\addplot[only marks, mark=+, mark size=2.0,color=red] %
								table[x=lastcall,y=error,col
								sep=comma]{./data/master-slave/schedule-4p-4b.json-random-failure.csv.xen-vm5};
							\addlegendentry{Partisi 5};
						\end{axis}
					\end{tikzpicture}
				}
			\end{tabular}
			\caption{Hasil pengujian keandalan sistem menggunakan skenario 5
			(layanan 1)}
			\label{figure:hasil_keandalan_s5_service1}
		\end{figure}

		Hasil pengujian keandalan pada skenario 5 dapat dilihat pada
		\autoref{figure:hasil_keandalan_s5_service1}.
		\autoref{figure:hasil_keandalan_s5_service2}.
		\autoref{figure:hasil_keandalan_s5_service3}.  dan
		\autoref{figure:hasil_keandalan_s5_service4}.  Hasil pengujian untuk
		masing-masing layanan adalah sebagai berikut.
		
		\begin{enumerate}
			\item Pada layanan 1, partisi berjalan
				dengan stabil. Dari hasil pengujian, terlihat bahwa partisi 5 dapat mengambil
				alih peran partisi 1 dalam menyediakan layanan 1 ketika partisi 1 mengalami
				kegagalan. Dengan demikian, layanan 1 tersedia selama rentang waktu pengujian.
				Namun, galat maksimal antara pengukuran \textit{timing} dengan \textit{timing}
				ekspektasi sangat tinggi yaitu hampir mencapai \SI{40}{\milli\second}.

				\begin{figure}[!ht]
					\centering
					\begin{tabular}{c}
						\subfloat{
							\tikzsetnextfilename{reliability-s5-delta-p2}
							\begin{tikzpicture}
								\begin{axis}[
									height=4cm, width=9cm,
									title={Timing Plot},
									scaled ticks=false,
									xlabel={Waktu (s)},
									ylabel={Delta (s)},
									grid=major,
									xlabel near ticks,
									ylabel near ticks,
									yticklabel style={
										/pgf/number format/precision=2,
										/pgf/number format/sci,
										/pgf/number format/sci zerofill,
									}
									]
									\addplot[only marks, mark=o, mark size=2.0,color=blue] %
										table[x=lastcall,y=delta,col
										sep=comma]{./data/master-slave/schedule-4p-4b.json-random-failure.csv.xen-vm2};
									\addlegendentry{Partisi 2};
									\addplot[only marks, mark=+, mark size=2.0,color=red] %
										table[x=lastcall,y=delta,col
										sep=comma]{./data/master-slave/schedule-4p-4b.json-random-failure.csv.xen-vm6};
									\addlegendentry{Partisi 6};
								\end{axis}
							\end{tikzpicture}
						} \\
						\subfloat{
							\tikzsetnextfilename{reliability-s5-error-p2}
							\begin{tikzpicture}
								\begin{axis}[
									height=4cm, width=9cm,
									title={Timing Plot},
									scaled ticks=false,
									xlabel={Waktu (s)},
									ylabel={Galat (s)},
									grid=major,
									xlabel near ticks,
									ylabel near ticks,
									yticklabel style={
										/pgf/number format/precision=2,
										/pgf/number format/sci,
										/pgf/number format/sci zerofill,
									}
									]
									\addplot[only marks, mark=o, mark size=2.0,color=blue] %
										table[x=lastcall,y=error,col
										sep=comma]{./data/master-slave/schedule-4p-4b.json-random-failure.csv.xen-vm2};
									\addlegendentry{Partisi 2};
									\addplot[only marks, mark=+, mark size=2.0,color=red] %
										table[x=lastcall,y=error,col
										sep=comma]{./data/master-slave/schedule-4p-4b.json-random-failure.csv.xen-vm6};
									\addlegendentry{Partisi 6};
								\end{axis}
							\end{tikzpicture}
						}
					\end{tabular}
					\caption{Hasil pengujian keandalan sistem menggunakan skenario 5
					(layanan 2)}
					\label{figure:hasil_keandalan_s5_service2}
				\end{figure}

			\item Pada layanan 2, partisi berjalan dengan stabil. Sama seperti pada layanan 1,
				terlihat bahwa partisi 6 dapat mengambil alih peran partisi 2 dalam menyediakan
				layanan 2 ketika partisi 2 mengalami kegagalan. Dengan demikian, layanan 2
				tersedia selama rentang waktu pengujian. Namun, galat maksimal antara pengukuran
				\textit{timing} dengan \textit{timing} ekspektasi sangat tinggi yaitu hampir
				mencapai \SI{40}{\milli\second}.

				\begin{figure}[!ht]
					\centering
					\begin{tabular}{c}
						\subfloat{
							\tikzsetnextfilename{reliability-s5-delta-p3}
							\begin{tikzpicture}
								\begin{axis}[
									height=4cm, width=9cm,
									title={Timing Plot},
									scaled ticks=false,
									xlabel={Waktu (s)},
									ylabel={Delta (s)},
									grid=major,
									xlabel near ticks,
									ylabel near ticks,
									yticklabel style={
										/pgf/number format/precision=2,
										/pgf/number format/sci,
										/pgf/number format/sci zerofill,
									}
									]
									\addplot[only marks, mark=o, mark size=2.0,color=blue] %
										table[x=lastcall,y=delta,col
										sep=comma]{./data/master-slave/schedule-4p-4b.json-random-failure.csv.xen-vm3};
									\addlegendentry{Partisi 3};
									\addplot[only marks, mark=+, mark size=2.0,color=red] %
										table[x=lastcall,y=delta,col
										sep=comma]{./data/master-slave/schedule-4p-4b.json-random-failure.csv.xen-vm7};
									\addlegendentry{Partisi 7};
								\end{axis}
							\end{tikzpicture}
						} \\
						\subfloat{
							\tikzsetnextfilename{reliability-s5-error-p3}
							\begin{tikzpicture}
								\begin{axis}[
									height=4cm, width=9cm,
									title={Timing Plot},
									scaled ticks=false,
									xlabel={Waktu (s)},
									ylabel={Galat (s)},
									grid=major,
									xlabel near ticks,
									ylabel near ticks,
									yticklabel style={
										/pgf/number format/precision=2,
										/pgf/number format/sci,
										/pgf/number format/sci zerofill,
									}
									]
									\addplot[only marks, mark=o, mark size=2.0,color=blue] %
										table[x=lastcall,y=error,col
										sep=comma]{./data/master-slave/schedule-4p-4b.json-random-failure.csv.xen-vm3};
									\addlegendentry{Partisi 3};
									\addplot[only marks, mark=+, mark size=2.0,color=red] %
										table[x=lastcall,y=error,col
										sep=comma]{./data/master-slave/schedule-4p-4b.json-random-failure.csv.xen-vm7};
									\addlegendentry{Partisi 7};
								\end{axis}
							\end{tikzpicture}
						}
					\end{tabular}
					\caption{Hasil pengujian keandalan sistem menggunakan skenario 5
					(layanan 3)}
					\label{figure:hasil_keandalan_s5_service3}
				\end{figure}

			\item Pada layanan 3, partisi berjalan dengan stabil. Sama seperti pada
				layanan-layanan sebelumnya, terlihat bahwa partisi 7 dapat mengambil alih peran
				partisi 3 dalam menyediakan layanan 3 ketika partisi 3 mengalami kegagalan.
				Dengan demikian, layanan 3 tersedia selama rentang waktu pengujian. Namun, galat
				maksimal antara pengukuran \textit{timing} dengan \textit{timing} ekspektasi
				sangat tinggi yaitu hampir mencapai \SI{40}{\milli\second}. Selain itu, terlihat
				beberapa titik dengan galat yang cukup tinggi berada pada waktu sekitar
				pergantian partisi \textit{primary} dengan partisi \textit{backup}.

				\begin{figure}[!ht]
					\centering
					\begin{tabular}{c}
						\subfloat{
							\tikzsetnextfilename{reliability-s5-delta-p4}
							\begin{tikzpicture}
								\begin{axis}[
									height=4cm, width=9cm,
									title={Timing Plot},
									scaled ticks=false,
									xlabel={Waktu (s)},
									ylabel={Delta (s)},
									grid=major,
									xlabel near ticks,
									ylabel near ticks,
									yticklabel style={
										/pgf/number format/precision=2,
										/pgf/number format/sci,
										/pgf/number format/sci zerofill,
									}
									]
									\addplot[only marks, mark=o, mark size=2.0,color=blue] %
										table[x=lastcall,y=delta,col
										sep=comma]{./data/master-slave/schedule-4p-4b.json-random-failure.csv.xen-vm4};
									\addlegendentry{Partisi 4};
									\addplot[only marks, mark=+, mark size=2.0,color=red] %
										table[x=lastcall,y=delta,col
										sep=comma]{./data/master-slave/schedule-4p-4b.json-random-failure.csv.xen-vm8};
									\addlegendentry{Partisi 8};
								\end{axis}
							\end{tikzpicture}
						} \\
						\subfloat{
							\tikzsetnextfilename{reliability-s5-error-p4}
							\begin{tikzpicture}
								\begin{axis}[
									height=4cm, width=9cm,
									title={Timing Plot},
									scaled ticks=false,
									xlabel={Waktu (s)},
									ylabel={Galat (s)},
									grid=major,
									xlabel near ticks,
									ylabel near ticks,
									yticklabel style={
										/pgf/number format/precision=2,
										/pgf/number format/sci,
										/pgf/number format/sci zerofill,
									}
									]
									\addplot[only marks, mark=o, mark size=2.0,color=blue] %
										table[x=lastcall,y=error,col
										sep=comma]{./data/master-slave/schedule-4p-4b.json-random-failure.csv.xen-vm4};
									\addlegendentry{Partisi 4};
									\addplot[only marks, mark=+, mark size=2.0,color=red] %
										table[x=lastcall,y=error,col
										sep=comma]{./data/master-slave/schedule-4p-4b.json-random-failure.csv.xen-vm8};
									\addlegendentry{Partisi 8};
								\end{axis}
							\end{tikzpicture}
						}
					\end{tabular}
					\caption{Hasil pengujian keandalan sistem menggunakan skenario 5 (layanan 4)}
					\label{figure:hasil_keandalan_s5_service4}
				\end{figure}

			\item Pada layanan 4, partisi berjalan dengan stabil. Sama seperti pada
				layanan-layanan sebelumnya, terlihat bahwa partisi 8 dapat mengambil alih peran
				partisi 4 dalam menyediakan layanan 4 ketika partisi 4 mengalami kegagalan.
				Dengan demikian, layanan 4 tersedia selama rentang waktu pengujian. Namun, galat
				maksimal antara pengukuran \textit{timing} dengan \textit{timing} ekspektasi
				sangat tinggi yaitu hampir mencapai \SI{40}{\milli\second}.

		\end{enumerate}

		Sama seperti pada skenario 3, secara umum partisi berjalan dengan sangat stabil.
		Namun, pada pengujian ini terlihat bahwa banyak titik tidak stabil berada pada
		sekitar waktu pergantian antara partisi \textit{primary} dengan partisi
		\textit{backup}.

	\item Skenario 6

		\begin{figure}[!ht]
			\centering
			\begin{tabular}{c}
				\subfloat{
					\tikzsetnextfilename{reliability-s6-delta-p1}
					\begin{tikzpicture}
						\begin{axis}[
							height=4cm, width=9cm,
							title={Timing Plot},
							scaled ticks=false,
							xlabel={Waktu (s)},
							ylabel={Delta (s)},
							grid=major,
							xlabel near ticks,
							ylabel near ticks,
							yticklabel style={
								/pgf/number format/precision=2,
								/pgf/number format/sci,
								/pgf/number format/sci zerofill,
							}
							]
							\addplot[only marks, mark=o, mark size=2.0,color=blue] %
								table[x=lastcall,y=delta,col
								sep=comma]{./data/master-slave/schedule-random.json-random-failure.csv.xen-vm1};
							\addlegendentry{Partisi 1};
							\addplot[only marks, mark=+, mark size=2.0,color=red] %
								table[x=lastcall,y=delta,col
								sep=comma]{./data/master-slave/schedule-random.json-random-failure.csv.xen-vm4};
							\addlegendentry{Partisi 4};
						\end{axis}
					\end{tikzpicture}
				} \\
				\subfloat{
					\tikzsetnextfilename{reliability-s6-error-p1}
					\begin{tikzpicture}
						\begin{axis}[
							height=4cm, width=9cm,
							title={Timing Plot},
							scaled ticks=false,
							xlabel={Waktu (s)},
							ylabel={Galat (s)},
							grid=major,
							xlabel near ticks,
							ylabel near ticks,
							yticklabel style={
								/pgf/number format/precision=2,
								/pgf/number format/sci,
								/pgf/number format/sci zerofill,
							}
							]
							\addplot[only marks, mark=o, mark size=2.0,color=blue] %
								table[x=lastcall,y=error,col
								sep=comma]{./data/master-slave/schedule-random.json-random-failure.csv.xen-vm1};
							\addlegendentry{Partisi 1};
							\addplot[only marks, mark=+, mark size=2.0,color=red] %
								table[x=lastcall,y=error,col
								sep=comma]{./data/master-slave/schedule-random.json-random-failure.csv.xen-vm4};
							\addlegendentry{Partisi 4};
						\end{axis}
					\end{tikzpicture}
				}
			\end{tabular}
			\caption{Hasil pengujian keandalan sistem menggunakan skenario 6 (layanan 1)}
			\label{figure:hasil_keandalan_s6_service1}
		\end{figure}

		Hasil pengujian keandalan pada skenario 6 dapat dilihat pada
		\autoref{figure:hasil_keandalan_s6_service1},
		\autoref{figure:hasil_keandalan_s6_service2},
		\autoref{figure:hasil_keandalan_s6_service3}.  Hasil pengujian pada
		masing-masing layanan adalah sebagai berikut.
		
		\begin{enumerate}

			\item Pada layanan 1, partisi berjalan
				dengan stabil.  Dari hasil pengujian, dapat terlihat bahwa partisi 4 dapat
				menggantikan peran partisi 1 ketika mengalami partisi 1 sedang mengalami
				kegagalan. Dengan demikian, layanan dapat tersedia selama rentang waktu
				pengujian. Namun, galat maksimal antara pengukuran \textit{timing} dengan
				\textit{timing} ekspektasi sangat tinggi yaitu hampir mencapai
				\SI{60}{\milli\second}. Selain itu, terlihat beberapa titik dengan galat yang
				cukup tinggi berada pada waktu sekitar pergantian partisi \textit{primary}
				dengan partisi \textit{backup}.

				\begin{figure}[!ht]
					\centering
					\begin{tabular}{c}
						\subfloat{
							\tikzsetnextfilename{reliability-s6-delta-p2}
							\begin{tikzpicture}
								\begin{axis}[
									height=4cm, width=9cm,
									title={Timing Plot},
									scaled ticks=false,
									xlabel={Waktu (s)},
									ylabel={Delta (s)},
									grid=major,
									xlabel near ticks,
									ylabel near ticks,
									yticklabel style={
										/pgf/number format/precision=2,
										/pgf/number format/sci,
										/pgf/number format/sci zerofill,
									}
									]
									\addplot[only marks, mark=o, mark size=2.0,color=blue] %
										table[x=lastcall,y=delta,col
										sep=comma]{./data/master-slave/schedule-random.json-random-failure.csv.xen-vm2};
									\addlegendentry{Partisi 2};
									\addplot[only marks, mark=+, mark size=2.0,color=red] %
										table[x=lastcall,y=delta,col
										sep=comma]{./data/master-slave/schedule-random.json-random-failure.csv.xen-vm6};
									\addlegendentry{Partisi 5};
									\addplot[only marks, mark=square, mark size=2.0,color=green] %
										table[x=lastcall,y=delta,col
										sep=comma]{./data/master-slave/schedule-random.json-random-failure.csv.xen-vm7};
									\addlegendentry{Partisi 6};
								\end{axis}
							\end{tikzpicture}
						} \\
						\subfloat{
							\tikzsetnextfilename{reliability-s6-error-p2}
							\begin{tikzpicture}
								\begin{axis}[
									height=4cm, width=9cm,
									title={Timing Plot},
									scaled ticks=false,
									xlabel={Waktu (s)},
									ylabel={Galat (s)},
									grid=major,
									xlabel near ticks,
									ylabel near ticks,
									yticklabel style={
										/pgf/number format/precision=2,
										/pgf/number format/sci,
										/pgf/number format/sci zerofill,
									}
									]
									\addplot[only marks, mark=o, mark size=2.0,color=blue] %
										table[x=lastcall,y=error,col
										sep=comma]{./data/master-slave/schedule-random.json-random-failure.csv.xen-vm2};
									\addlegendentry{Partisi 2};
									\addplot[only marks, mark=+, mark size=2.0,color=red] %
										table[x=lastcall,y=error,col
										sep=comma]{./data/master-slave/schedule-random.json-random-failure.csv.xen-vm6};
									\addlegendentry{Partisi 5};
									\addplot[only marks, mark=square, mark size=2.0,color=green] %
										table[x=lastcall,y=error,col
										sep=comma]{./data/master-slave/schedule-random.json-random-failure.csv.xen-vm7};
									\addlegendentry{Partisi 6};
								\end{axis}
							\end{tikzpicture}
						}
					\end{tabular}
					\caption{Hasil pengujian keandalan sistem menggunakan skenario 6 (layanan 2)}
					\label{figure:hasil_keandalan_s6_service2}
				\end{figure}

			\item Pada layanan 2, partisi berjalan dengan stabil. Dari hasil pengujian, terlihat
				bahwa partisi 5 dapat menggantikan peran partisi 2 ketika terjadi kegagalan pada
				partisi 2. Dan apabila partisi 5 juga mengalami kegagalan, partisi 6 dapat
				menggantikan peran partisi 5. Ketika partisi 2 kembali dari kegagalan, maka
				partisi 6 tidak lagi berjalan dan digantikan oleh partisi 2. Dengan demikian,
				layanan 2 selalu tersedia selama pengujian.

				Namun, galat maksimal antara pengukuran \textit{timing} dengan \textit{timing}
				ekspektasi sangat tinggi yaitu sekitar \SI{60}{\milli\second}.  Selain itu,
				terlihat beberapa titik dengan galat yang cukup tinggi berada pada waktu sekitar
				pergantian partisi \textit{primary} dengan partisi \textit{backup}.

				\begin{figure}[!ht]
					\centering
					\begin{tabular}{c}
						\subfloat{
							\tikzsetnextfilename{reliability-s6-delta-p3}
							\begin{tikzpicture}
								\begin{axis}[
									height=4cm, width=9cm,
									title={Timing Plot},
									scaled ticks=false,
									xlabel={Waktu (s)},
									ylabel={Delta (s)},
									grid=major,
									xlabel near ticks,
									ylabel near ticks,
									yticklabel style={
										/pgf/number format/precision=2,
										/pgf/number format/sci,
										/pgf/number format/sci zerofill,
									}
									]
									\addplot[only marks, mark=o, mark size=2.0,color=blue] %
										table[x=lastcall,y=delta,col
										sep=comma]{./data/master-slave/schedule-random.json-random-failure.csv.xen-vm3};
									\addlegendentry{Partisi 3};
								\end{axis}
							\end{tikzpicture}
						} \\
						\subfloat{
							\tikzsetnextfilename{reliability-s6-error-p3}
							\begin{tikzpicture}
								\begin{axis}[
									height=4cm, width=9cm,
									title={Timing Plot},
									scaled ticks=false,
									xlabel={Waktu (s)},
									ylabel={Galat (s)},
									grid=major,
									xlabel near ticks,
									ylabel near ticks,
									yticklabel style={
										/pgf/number format/precision=2,
										/pgf/number format/sci,
										/pgf/number format/sci zerofill,
									}
									]
									\addplot[only marks, mark=o, mark size=2.0,color=blue] %
										table[x=lastcall,y=error,col
										sep=comma]{./data/master-slave/schedule-random.json-random-failure.csv.xen-vm3};
									\addlegendentry{Partisi 3};
								\end{axis}
							\end{tikzpicture}
						}
					\end{tabular}
					\caption{Hasil pengujian keandalan sistem menggunakan skenario 6 (layanan 3)}
					\label{figure:hasil_keandalan_s6_service3}
				\end{figure}

			\item Pada layanan 3, partisi berjalan dengan stabil. Namun, galat maksimal antara
				pengukuran \textit{timing} dengan \textit{timing} ekspektasi sangat tinggi yaitu
				sekitar \SI{10}{\milli\second}.

		\end{enumerate}

\end{enumerate}

\subsection{Pengujian \textit{Latency}}

Pada pengujian \textit{latency}, \textit{backup} untuk masing-masing layanan tidak diperlukan
karena diasumsikan partisi tidak akan mengalami kegagalan. Dengan demikian, hasil yang
ditampilkan hanyalah partisi yang merupakan partisi \textit{primary}.

Pengujian \textit{latency} akan dilakukan dengan menggunakan skenario-skenario berikut.

\begin{enumerate}
	\item Skenario 1

		Pada skenario ini, daftar jadwal berisi 1 layanan. Layanan akan berjalan
		selama \SI{20}{\milli\second} dengan \textit{major time frame} keseluruhan
		selama \SI{20}{\milli\second}.

		Skenario ini akan menguji \textit{latency} layanan apabila sistem hanya memiliki
		satu buah layanan. Hasil pengujian akan menjadi referensi \textit{latency}
		terbaik yang dapat dicapai oleh sebuah layanan pada sistem.

	\item Skenario 2

		Pada skenario ini, daftar jadwal berisi 4 layanan. Masing-masing layanan akan
		berjalan selama \SI{10}{\milli\second} dengan \textit{major time frame}
		keseluruhan selama \SI{50}{\milli\second}.

		Skenario ini akan menguji \textit{latency} layanan apabila sistem memiliki
		banyak layanan, namun masih terdapat banyak \textit{resource} yang tersisa.
		Hasil pengujian akan dianalisa bersamaan dengan skenario 1 dan skenario 3 untuk
		melihat apakah terdapat kecenderungan tertentu pada kinerja sistem terhadap
		jumlah layanan pada sistem.

	\item Skenario 3

		Pada skenario ini, daftar jadwal berisi 9 layanan. Masing-masing akan berjalan
		selama \SI{20}{\milli\second} dengan \textit{major time frame} keseluruhan
		selama \SI{180}{\milli\second}.

		Skenario ini akan menguji \textit{latency} layanan apabila sistem memiliki
		banyak layanan. Hasil pengujian akan menunjukkan \textit{latency} pada sistem
		dengan banyak layanan.

\end{enumerate}

Hasil pengujian \textit{latency} sistem berdasarkan skenario-skenario pengujian akan dipaparkan.
Pengujian diambil dengan menjalankan program pengujian selama 2 jam. Namun, hasil pengujian
hanya akan ditampilkan sampai terlihat sebuah kecenderungan pada data.

\begin{enumerate}

	\item Skenario 1

		\begin{figure}[!ht]
			\centering
			\begin{tabular}{c}
				\subfloat{
					\tikzsetnextfilename{latency-s1-delta}
					\begin{tikzpicture}
						\begin{axis}[
							height=4cm, width=9cm,
							title={Timing Plot},
							scaled ticks=false,
							xlabel={Waktu (s)},
							ylabel={Delta (s)},
							grid=major,
							xlabel near ticks,
							ylabel near ticks,
							yticklabel style={
								/pgf/number format/precision=3,
								/pgf/number format/sci,
								/pgf/number format/sci zerofill,
							}
							]
							\addplot[only marks, mark=o, mark size=2.0,color=blue] %
								table[x=time,y=delta,col
								sep=comma]{./data/periodic-task/schedule-domU-monotonic/schedule-domU-monotonic.log};
							\addlegendentry{Partisi 1};
						\end{axis}
					\end{tikzpicture}
				} \\
				\subfloat{
					\tikzsetnextfilename{latency-s1-error}
					\begin{tikzpicture}
						\begin{axis}[
							height=4cm, width=9cm,
							title={Timing Plot},
							scaled ticks=false,
							xlabel={Waktu (s)},
							ylabel={Galat (s)},
							grid=major,
							xlabel near ticks,
							ylabel near ticks,
							yticklabel style={
								/pgf/number format/precision=3,
								/pgf/number format/sci,
								/pgf/number format/sci zerofill,
							}
							]
							\addplot[only marks, mark=o, mark size=2.0,color=blue] %
								table[x=time,y=error,col
								sep=comma]{./data/periodic-task/schedule-domU-monotonic/schedule-domU-monotonic.log};
							\addlegendentry{Partisi 1};
						\end{axis}
					\end{tikzpicture}
				}
			\end{tabular}
			\caption{Hasil pengujian \textit{latency} sistem menggunakan skenario 1}
			\label{figure:hasil_latency_s1}
		\end{figure}

		Hasil pengujian \textit{latency} dengan menggunakan skenario 1 dapat dilihat
		pada \autoref{figure:hasil_latency_s1}.  Pengujian dengan menggunakan skenario 1
		menghasilkan \textit{timing} yang sangat presisi, dengan galat \textit{timing}
		tertinggi dari ekspektasi periode \textit{timing} yang seharusnya hanya sekitar
		\SI{100}{\micro\second}. Pada pengujian ini, perilaku algoritma
		\textit{primary-backup partition scheduling} tidak dapat terlihat. Namun,
		pengujian ini dapat digunakan sebagai referensi \textit{timing} yang seharusnya
		didapatkan pada pengujian-pengujian berikutnya.

	\item Skenario 2

		\begin{figure}[!ht]
			\centering
			\begin{tabular}{c}
				\subfloat{
					\tikzsetnextfilename{latency-s2-delta-p1}
					\begin{tikzpicture}
						\begin{axis}[
							height=4cm, width=9cm,
							title={Timing Plot},
							scaled ticks=false,
							xlabel={Waktu (s)},
							ylabel={Delta (s)},
							grid=major,
							xlabel near ticks,
							ylabel near ticks,
							yticklabel style={
								/pgf/number format/precision=2,
								/pgf/number format/sci,
								/pgf/number format/sci zerofill,
							}
							]
							\addplot[only marks, mark=o, mark size=2.0,color=blue] %
								table[x=time,y=delta,col
								sep=comma]{./data/periodic-task/schedule-4p-4b/xen-vm1/schedule-4p-4b.log};
							\addlegendentry{Partisi 1};
						\end{axis}
					\end{tikzpicture}
				} \\
				\subfloat{
					\tikzsetnextfilename{latency-s2-error-p1}
					\begin{tikzpicture}
						\begin{axis}[
							height=4cm, width=9cm,
							title={Timing Plot},
							scaled ticks=false,
							xlabel={Waktu (s)},
							ylabel={Galat (s)},
							grid=major,
							xlabel near ticks,
							ylabel near ticks,
							yticklabel style={
								/pgf/number format/precision=2,
								/pgf/number format/sci,
								/pgf/number format/sci zerofill,
							}
							]
							\addplot[only marks, mark=o, mark size=2.0,color=blue] %
								table[x=time,y=error,col
								sep=comma]{./data/periodic-task/schedule-4p-4b/xen-vm2/schedule-4p-4b.log};
							\addlegendentry{Partisi 1};
						\end{axis}
					\end{tikzpicture}
				}
			\end{tabular}
			\caption{Hasil pengujian \textit{latency} sistem menggunakan skenario 2
			(layanan 1)}
			\label{figure:hasil_latency_s2_service1}
		\end{figure}

		Hasil pengujian \textit{latency} pada skenario 2 dapat dilihat pada
		\autoref{figure:hasil_latency_s2_service1},
		\autoref{figure:hasil_latency_s2_service2},
		\autoref{figure:hasil_latency_s2_service3}, dan
		\autoref{figure:hasil_latency_s2_service4}. 

		\begin{enumerate}
			\item Pada layanan 1, partisi berjalan
				dengan stabil. Namun, galat maksimal antara pengukuran \textit{timing} dengan
				\textit{timing} ekspektasi sangat tinggi yaitu sekitar \SI{40}{\milli\second}.

				\begin{figure}[!ht]
					\centering
					\begin{tabular}{c}
						\subfloat{
							\tikzsetnextfilename{latency-s2-delta-p2}
							\begin{tikzpicture}
								\begin{axis}[
									height=4cm, width=9cm,
									title={Timing Plot},
									scaled ticks=false,
									xlabel={Waktu (s)},
									ylabel={Delta (s)},
									grid=major,
									xlabel near ticks,
									ylabel near ticks,
									yticklabel style={
										/pgf/number format/precision=2,
										/pgf/number format/sci,
										/pgf/number format/sci zerofill,
									}
									]
									\addplot[only marks, mark=o, mark size=2.0,color=blue] %
										table[x=time,y=delta,col
										sep=comma]{./data/periodic-task/schedule-4p-4b/xen-vm2/schedule-4p-4b.log};
									\addlegendentry{Partisi 2};
								\end{axis}
							\end{tikzpicture}
						} \\
						\subfloat{
							\tikzsetnextfilename{latency-s2-error-p2}
							\begin{tikzpicture}
								\begin{axis}[
									height=4cm, width=9cm,
									title={Timing Plot},
									scaled ticks=false,
									xlabel={Waktu (s)},
									ylabel={Galat (s)},
									grid=major,
									xlabel near ticks,
									ylabel near ticks,
									yticklabel style={
										/pgf/number format/precision=2,
										/pgf/number format/sci,
										/pgf/number format/sci zerofill,
									}
									]
									\addplot[only marks, mark=o, mark size=2.0,color=blue] %
										table[x=time,y=error,col
										sep=comma]{./data/periodic-task/schedule-4p-4b/xen-vm2/schedule-4p-4b.log};
									\addlegendentry{Partisi 2};
								\end{axis}
							\end{tikzpicture}
						}
					\end{tabular}
					\caption{Hasil pengujian \textit{latency} sistem menggunakan skenario 2
					(layanan 2)}
					\label{figure:hasil_latency_s2_service2}
				\end{figure}

			\item Pada layanan 2, partisi berjalan dengan stabil. Namun, galat maksimal antara
				pengukuran \textit{timing} dengan \textit{timing} ekspektasi sangat tinggi yaitu
				sekitar \SI{40}{\milli\second}.

				\begin{figure}[!ht]
					\centering
					\begin{tabular}{c}
						\subfloat{
							\tikzsetnextfilename{latency-s2-delta-p3}
							\begin{tikzpicture}
								\begin{axis}[
									height=4cm, width=9cm,
									title={Timing Plot},
									scaled ticks=false,
									xlabel={Waktu (s)},
									ylabel={Delta (s)},
									grid=major,
									xlabel near ticks,
									ylabel near ticks,
									yticklabel style={
										/pgf/number format/precision=2,
										/pgf/number format/sci,
										/pgf/number format/sci zerofill,
									}
									]
									\addplot[only marks, mark=o, mark size=2.0,color=blue] %
										table[x=time,y=delta,col
										sep=comma]{./data/periodic-task/schedule-4p-4b/xen-vm3/schedule-4p-4b.log};
									\addlegendentry{Partisi 3};
								\end{axis}
							\end{tikzpicture}
						} \\
						\subfloat{
							\tikzsetnextfilename{latency-s2-error-p3}
							\begin{tikzpicture}
								\begin{axis}[
									height=4cm, width=9cm,
									title={Timing Plot},
									scaled ticks=false,
									xlabel={Waktu (s)},
									ylabel={Galat (s)},
									grid=major,
									xlabel near ticks,
									ylabel near ticks,
									yticklabel style={
										/pgf/number format/precision=2,
										/pgf/number format/sci,
										/pgf/number format/sci zerofill,
									}
									]
									\addplot[only marks, mark=o, mark size=2.0,color=blue] %
										table[x=time,y=error,col
										sep=comma]{./data/periodic-task/schedule-4p-4b/xen-vm3/schedule-4p-4b.log};
									\addlegendentry{Partisi 3};
								\end{axis}
							\end{tikzpicture}
						}
					\end{tabular}
					\caption{Hasil pengujian \textit{latency} sistem menggunakan skenario 2 (layanan 3)}
					\label{figure:hasil_latency_s2_service3}
				\end{figure}

			\item Pada layanan 3, partisi berjalan dengan stabil. Namun, galat maksimal antara
				pengukuran \textit{timing} dengan \textit{timing} ekspektasi sangat tinggi yaitu
				sekitar \SI{40}{\milli\second}.

				\begin{figure}[!ht]
					\centering
					\begin{tabular}{c}
						\subfloat{
							\tikzsetnextfilename{latency-s2-delta-p4}
							\begin{tikzpicture}
								\begin{axis}[
									height=4cm, width=9cm,
									title={Timing Plot},
									scaled ticks=false,
									xlabel={Waktu (s)},
									ylabel={Delta (s)},
									grid=major,
									xlabel near ticks,
									ylabel near ticks,
									yticklabel style={
										/pgf/number format/precision=2,
										/pgf/number format/sci,
										/pgf/number format/sci zerofill,
									}
									]
									\addplot[only marks, mark=o, mark size=2.0,color=blue] %
										table[x=time,y=delta,col
										sep=comma]{./data/periodic-task/schedule-4p-4b/xen-vm4/schedule-4p-4b.log};
									\addlegendentry{Partisi 4};
								\end{axis}
							\end{tikzpicture}
						} \\
						\subfloat{
							\tikzsetnextfilename{latency-s2-error-p4}
							\begin{tikzpicture}
								\begin{axis}[
									height=4cm, width=9cm,
									title={Timing Plot},
									scaled ticks=false,
									xlabel={Waktu (s)},
									ylabel={Galat (s)},
									grid=major,
									xlabel near ticks,
									ylabel near ticks,
									yticklabel style={
										/pgf/number format/precision=2,
										/pgf/number format/sci,
										/pgf/number format/sci zerofill,
									}
									]
									\addplot[only marks, mark=o, mark size=2.0,color=blue] %
										table[x=time,y=error,col
										sep=comma]{./data/periodic-task/schedule-4p-4b/xen-vm4/schedule-4p-4b.log};
									\addlegendentry{Partisi 4};
								\end{axis}
							\end{tikzpicture}
						}
					\end{tabular}
					\caption{Hasil pengujian \textit{latency} sistem menggunakan skenario 2 (layanan 4)}
					\label{figure:hasil_latency_s2_service4}
				\end{figure}


			\item Pada layanan 4, partisi berjalan dengan stabil. Namun, galat maksimal antara
				pengukuran \textit{timing} dengan \textit{timing} ekspektasi sangat tinggi yaitu
				sekitar \SI{40}{\milli\second}.

		\end{enumerate}

		Hasil pengujian \textit{latency} dengan menggunakan skenario 2 memperlihatkan
		bahwa \textit{timing} dengan galat tinggi diakibatkan karena \textit{timing}
		berjalan lebih cepat dari yang seharusnya.

	\item Skenario 3

		\begin{figure}[!ht]
			\centering
			\begin{tabular}{c}
				\subfloat{
					\tikzsetnextfilename{latency-s3-delta-p1}
					\begin{tikzpicture}
						\begin{axis}[
							height=4cm, width=9cm,
							title={Timing Plot},
							scaled ticks=false,
							xlabel={Waktu (s)},
							ylabel={Delta (s)},
							grid=major,
							xlabel near ticks,
							ylabel near ticks,
							yticklabel style={
								/pgf/number format/precision=2,
								/pgf/number format/sci,
								/pgf/number format/sci zerofill,
							}
							]
							\addplot[only marks, mark=o, mark size=2.0,color=blue] %
								table[x=time,y=delta,col
								sep=comma]{./data/periodic-task/schedule-all-monotonic/xen-vm1/schedule-all-monotonic.log};
							\addlegendentry{Partisi 1};
						\end{axis}
					\end{tikzpicture}
				} \\
				\subfloat{
					\tikzsetnextfilename{latency-s3-error-p1}
					\begin{tikzpicture}
						\begin{axis}[
							height=4cm, width=9cm,
							title={Timing Plot},
							scaled ticks=false,
							xlabel={Waktu (s)},
							ylabel={Galat (s)},
							grid=major,
							xlabel near ticks,
							ylabel near ticks,
							yticklabel style={
								/pgf/number format/precision=2,
								/pgf/number format/sci,
								/pgf/number format/sci zerofill,
							}
							]
							\addplot[only marks, mark=o, mark size=2.0,color=blue] %
								table[x=time,y=error,col
								sep=comma]{./data/periodic-task/schedule-all-monotonic/xen-vm1/schedule-all-monotonic.log};
							\addlegendentry{Partisi 1};
						\end{axis}
					\end{tikzpicture}
				}
			\end{tabular}
			\caption{Hasil pengujian \textit{latency} sistem menggunakan skenario 3 (layanan 1)}
			\label{figure:hasil_latency_s3_service1}
		\end{figure}

		Hasil pengujian \textit{latency} pada skenario 3 dapat dilihat pada
		\autoref{figure:hasil_latency_s3_service1},
		\autoref{figure:hasil_latency_s3_service2},
		\autoref{figure:hasil_latency_s3_service3},
		\autoref{figure:hasil_latency_s3_service4},
		\autoref{figure:hasil_latency_s3_service5},
		\autoref{figure:hasil_latency_s3_service6},
		\autoref{figure:hasil_latency_s3_service7},
		\autoref{figure:hasil_latency_s3_service8},
		dan \autoref{figure:hasil_latency_s3_service9}. Hasil untuk masing-masing
		layanan adalah sebagai berikut.

		\begin{enumerate}
			\item Pada layanan 1, partisi berjalan dengan sangat stabil. Namun, galat antara
				maksimal pengukuran \textit{timing} dengan \textit{timing} ekspektasi cukup
				tinggi yaitu sekitar \SI{80}{\micro\second}.

				\begin{figure}[!ht]
					\centering
					\begin{tabular}{c}
						\subfloat{
							\tikzsetnextfilename{latency-s3-delta-p2}
							\begin{tikzpicture}
								\begin{axis}[
									height=4cm, width=9cm,
									title={Timing Plot},
									scaled ticks=false,
									xlabel={Waktu (s)},
									ylabel={Delta (s)},
									grid=major,
									xlabel near ticks,
									ylabel near ticks,
									yticklabel style={
										/pgf/number format/precision=2,
										/pgf/number format/sci,
										/pgf/number format/sci zerofill,
									}
									]
									\addplot[only marks, mark=o, mark size=2.0,color=blue] %
										table[x=time,y=delta,col
										sep=comma]{./data/periodic-task/schedule-all-monotonic/xen-vm2/schedule-all-monotonic.log};
									\addlegendentry{Partisi 2};
								\end{axis}
							\end{tikzpicture}
						} \\
						\subfloat{
							\tikzsetnextfilename{latency-s3-error-p2}
							\begin{tikzpicture}
								\begin{axis}[
									height=4cm, width=9cm,
									title={Timing Plot},
									scaled ticks=false,
									xlabel={Waktu (s)},
									ylabel={Galat (s)},
									grid=major,
									xlabel near ticks,
									ylabel near ticks,
									yticklabel style={
										/pgf/number format/precision=2,
										/pgf/number format/sci,
										/pgf/number format/sci zerofill,
									}
									]
									\addplot[only marks, mark=o, mark size=2.0,color=blue] %
										table[x=time,y=error,col
										sep=comma]{./data/periodic-task/schedule-all-monotonic/xen-vm2/schedule-all-monotonic.log};
									\addlegendentry{Partisi 2};
								\end{axis}
							\end{tikzpicture}
						}
					\end{tabular}
					\caption{Hasil pengujian \textit{latency} sistem menggunakan skenario 3 (layanan 2)}
					\label{figure:hasil_latency_s3_service2}
				\end{figure}

			\item Pada layanan 2, partisi berjalan dengan sangat stabil. Namun, galat antara
				maksimal pengukuran \textit{timing} dengan \textit{timing} ekspektasi cukup
				tinggi yaitu sekitar \SI{80}{\micro\second}.

				\begin{figure}[!ht]
					\centering
					\begin{tabular}{c}
						\subfloat{
							\tikzsetnextfilename{latency-s3-delta-p3}
							\begin{tikzpicture}
								\begin{axis}[
									height=4cm, width=9cm,
									title={Timing Plot},
									scaled ticks=false,
									xlabel={Waktu (s)},
									ylabel={Delta (s)},
									grid=major,
									xlabel near ticks,
									ylabel near ticks,
									yticklabel style={
										/pgf/number format/precision=2,
										/pgf/number format/sci,
										/pgf/number format/sci zerofill,
									}
									]
									\addplot[only marks, mark=o, mark size=2.0,color=blue] %
										table[x=time,y=delta,col
										sep=comma]{./data/periodic-task/schedule-all-monotonic/xen-vm3/schedule-all-monotonic.log};
									\addlegendentry{Partisi 3};
								\end{axis}
							\end{tikzpicture}
						} \\
						\subfloat{
							\tikzsetnextfilename{latency-s3-error-p3}
							\begin{tikzpicture}
								\begin{axis}[
									height=4cm, width=9cm,
									title={Timing Plot},
									scaled ticks=false,
									xlabel={Waktu (s)},
									ylabel={Galat (s)},
									grid=major,
									xlabel near ticks,
									ylabel near ticks,
									yticklabel style={
										/pgf/number format/precision=2,
										/pgf/number format/sci,
										/pgf/number format/sci zerofill,
									}
									]
									\addplot[only marks, mark=o, mark size=2.0,color=blue] %
										table[x=time,y=error,col
										sep=comma]{./data/periodic-task/schedule-all-monotonic/xen-vm3/schedule-all-monotonic.log};
									\addlegendentry{Partisi 3};
								\end{axis}
							\end{tikzpicture}
						}
					\end{tabular}
					\caption{Hasil pengujian \textit{latency} sistem menggunakan skenario 3 (layanan 3)}
					\label{figure:hasil_latency_s3_service3}
				\end{figure}

			\item Pada layanan 3, partisi berjalan dengan cukup stabil. Namun, galat maksimal pengukuran
				\textit{timing} dengan \textit{timing} ekspektasi cukup tinggi yaitu sekitar
				\SI{100}{\micro\second}.

				\begin{figure}[!ht]
					\centering
					\begin{tabular}{c}
						\subfloat{
							\tikzsetnextfilename{latency-s3-delta-p4}
							\begin{tikzpicture}
								\begin{axis}[
									height=4cm, width=9cm,
									title={Timing Plot},
									scaled ticks=false,
									xlabel={Waktu (s)},
									ylabel={Delta (s)},
									grid=major,
									xlabel near ticks,
									ylabel near ticks,
									yticklabel style={
										/pgf/number format/precision=2,
										/pgf/number format/sci,
										/pgf/number format/sci zerofill,
									}
									]
									\addplot[only marks, mark=o, mark size=2.0,color=blue] %
										table[x=time,y=delta,col
										sep=comma]{./data/periodic-task/schedule-all-monotonic/xen-vm4/schedule-all-monotonic.log};
									\addlegendentry{Partisi 4};
								\end{axis}
							\end{tikzpicture}
						} \\
						\subfloat{
							\tikzsetnextfilename{latency-s3-error-p4}
							\begin{tikzpicture}
								\begin{axis}[
									height=4cm, width=9cm,
									title={Timing Plot},
									scaled ticks=false,
									xlabel={Waktu (s)},
									ylabel={Galat (s)},
									grid=major,
									xlabel near ticks,
									ylabel near ticks,
									yticklabel style={
										/pgf/number format/precision=2,
										/pgf/number format/sci,
										/pgf/number format/sci zerofill,
									}
									]
									\addplot[only marks, mark=o, mark size=2.0,color=blue] %
										table[x=time,y=error,col
										sep=comma]{./data/periodic-task/schedule-all-monotonic/xen-vm4/schedule-all-monotonic.log};
									\addlegendentry{Partisi 4};
								\end{axis}
							\end{tikzpicture}
						}
					\end{tabular}
					\caption{Hasil pengujian \textit{latency} sistem menggunakan skenario 3 (layanan 4)}
					\label{figure:hasil_latency_s3_service4}
				\end{figure}

			\item Pada layanan 4, partisi berjalan dengan stabil. Namun, galat maksimal pengukuran
				\textit{timing} dengan \textit{timing} ekspektasi sangat tinggi yaitu sekitar
				\SI{160}{\milli\second}.

				\begin{figure}[!ht]
					\centering
					\begin{tabular}{c}
						\subfloat{
							\tikzsetnextfilename{latency-s3-delta-p5}
							\begin{tikzpicture}
								\begin{axis}[
									height=4cm, width=9cm,
									title={Timing Plot},
									scaled ticks=false,
									xlabel={Waktu (s)},
									ylabel={Delta (s)},
									grid=major,
									xlabel near ticks,
									ylabel near ticks,
									yticklabel style={
										/pgf/number format/precision=2,
										/pgf/number format/sci,
										/pgf/number format/sci zerofill,
									}
									]
									\addplot[only marks, mark=o, mark size=2.0,color=blue] %
										table[x=time,y=delta,col
										sep=comma]{./data/periodic-task/schedule-all-monotonic/xen-vm5/schedule-all-monotonic.log};
									\addlegendentry{Partisi 5};
								\end{axis}
							\end{tikzpicture}
						} \\
						\subfloat{
							\tikzsetnextfilename{latency-s3-error-p5}
							\begin{tikzpicture}
								\begin{axis}[
									height=4cm, width=9cm,
									title={Timing Plot},
									scaled ticks=false,
									xlabel={Waktu (s)},
									ylabel={Galat (s)},
									grid=major,
									xlabel near ticks,
									ylabel near ticks,
									yticklabel style={
										/pgf/number format/precision=2,
										/pgf/number format/sci,
										/pgf/number format/sci zerofill,
									}
									]
									\addplot[only marks, mark=o, mark size=2.0,color=blue] %
										table[x=time,y=error,col
										sep=comma]{./data/periodic-task/schedule-all-monotonic/xen-vm5/schedule-all-monotonic.log};
									\addlegendentry{Partisi 5};
								\end{axis}
							\end{tikzpicture}
						}
					\end{tabular}
					\caption{Hasil pengujian \textit{latency} sistem menggunakan skenario 3 (layanan 5)}
					\label{figure:hasil_latency_s3_service5}
				\end{figure}

			\item Pada layanan 5, partisi berjalan dengan sangat stabil. Namun, galat maksimal
				pengukuran \textit{timing} dengan \textit{timing} ekspektasi cukup tinggi yaitu
				sekitar \SI{80}{\micro\second}.

				\begin{figure}[!ht]
					\centering
					\begin{tabular}{c}
						\subfloat{
							\tikzsetnextfilename{latency-s3-delta-p6}
							\begin{tikzpicture}
								\begin{axis}[
									height=4cm, width=9cm,
									title={Timing Plot},
									scaled ticks=false,
									xlabel={Waktu (s)},
									ylabel={Delta (s)},
									grid=major,
									xlabel near ticks,
									ylabel near ticks,
									yticklabel style={
										/pgf/number format/precision=2,
										/pgf/number format/sci,
										/pgf/number format/sci zerofill,
									}
									]
									\addplot[only marks, mark=o, mark size=2.0,color=blue] %
										table[x=time,y=delta,col
										sep=comma]{./data/periodic-task/schedule-all-monotonic/xen-vm6/schedule-all-monotonic.log};
									\addlegendentry{Partisi 6};
								\end{axis}
							\end{tikzpicture}
						} \\
						\subfloat{
							\tikzsetnextfilename{latency-s3-error-p6}
							\begin{tikzpicture}
								\begin{axis}[
									height=4cm, width=9cm,
									title={Timing Plot},
									scaled ticks=false,
									xlabel={Waktu (s)},
									ylabel={Galat (s)},
									grid=major,
									xlabel near ticks,
									ylabel near ticks,
									yticklabel style={
										/pgf/number format/precision=2,
										/pgf/number format/sci,
										/pgf/number format/sci zerofill,
									}
									]
									\addplot[only marks, mark=o, mark size=2.0,color=blue] %
										table[x=time,y=error,col
										sep=comma]{./data/periodic-task/schedule-all-monotonic/xen-vm6/schedule-all-monotonic.log};
									\addlegendentry{Partisi 6};
								\end{axis}
							\end{tikzpicture}
						}
					\end{tabular}
					\caption{Hasil pengujian \textit{latency} sistem menggunakan skenario 3 (layanan 6)}
					\label{figure:hasil_latency_s3_service6}
				\end{figure}

			\item Pada layanan 6, partisi berjalan dengan sangat stabil. Namun, galat maksimal
				pengukuran \textit{timing} dengan \textit{timing} ekspektasi cukup tinggi yaitu
				sekitar \SI{100}{\micro\second}.

				\begin{figure}[!ht]
					\centering
					\begin{tabular}{c}
						\subfloat{
							\tikzsetnextfilename{latency-s3-delta-p7}
							\begin{tikzpicture}
								\begin{axis}[
									height=4cm, width=9cm,
									title={Timing Plot},
									scaled ticks=false,
									xlabel={Waktu (s)},
									ylabel={Delta (s)},
									grid=major,
									xlabel near ticks,
									ylabel near ticks,
									yticklabel style={
										/pgf/number format/precision=2,
										/pgf/number format/sci,
										/pgf/number format/sci zerofill,
									}
									]
									\addplot[only marks, mark=o, mark size=2.0,color=blue] %
										table[x=time,y=delta,col
										sep=comma]{./data/periodic-task/schedule-all-monotonic/xen-vm7/schedule-all-monotonic.log};
									\addlegendentry{Partisi 7};
								\end{axis}
							\end{tikzpicture}
						} \\
						\subfloat{
							\tikzsetnextfilename{latency-s3-error-p7}
							\begin{tikzpicture}
								\begin{axis}[
									height=4cm, width=9cm,
									title={Timing Plot},
									scaled ticks=false,
									xlabel={Waktu (s)},
									ylabel={Galat (s)},
									grid=major,
									xlabel near ticks,
									ylabel near ticks,
									yticklabel style={
										/pgf/number format/precision=2,
										/pgf/number format/sci,
										/pgf/number format/sci zerofill,
									}
									]
									\addplot[only marks, mark=o, mark size=2.0,color=blue] %
										table[x=time,y=error,col
										sep=comma]{./data/periodic-task/schedule-all-monotonic/xen-vm7/schedule-all-monotonic.log};
									\addlegendentry{Partisi 7};
								\end{axis}
							\end{tikzpicture}
						}
					\end{tabular}
					\caption{Hasil pengujian \textit{latency} sistem menggunakan skenario 3 (layanan 7)}
					\label{figure:hasil_latency_s3_service7}
				\end{figure}

			\item Pada layanan 7, galat maksimal pengukuran \textit{timing} dengan \textit{timing}
				ekspektasi tidak terlalu tinggi yaitu sekitar \SI{100}{\micro\second}. Meski
				demikian, \textit{timing} yang didapat dari layanan tersebut tidak stabil.

				\begin{figure}[!ht]
					\centering
					\begin{tabular}{c}
						\subfloat{
							\tikzsetnextfilename{latency-s3-delta-p8}
							\begin{tikzpicture}
								\begin{axis}[
									height=4cm, width=9cm,
									title={Timing Plot},
									scaled ticks=false,
									xlabel={Waktu (s)},
									ylabel={Delta (s)},
									grid=major,
									xlabel near ticks,
									ylabel near ticks,
									yticklabel style={
										/pgf/number format/precision=2,
										/pgf/number format/sci,
										/pgf/number format/sci zerofill,
									}
									]
									\addplot[only marks, mark=o, mark size=2.0,color=blue] %
										table[x=time,y=delta,col
										sep=comma]{./data/periodic-task/schedule-all-monotonic/xen-vm8/schedule-all-monotonic.log};
									\addlegendentry{Partisi 8};
								\end{axis}
							\end{tikzpicture}
						} \\
						\subfloat{
							\tikzsetnextfilename{latency-s3-error-p8}
							\begin{tikzpicture}
								\begin{axis}[
									height=4cm, width=9cm,
									title={Timing Plot},
									scaled ticks=false,
									xlabel={Waktu (s)},
									ylabel={Galat (s)},
									grid=major,
									xlabel near ticks,
									ylabel near ticks,
									yticklabel style={
										/pgf/number format/precision=2,
										/pgf/number format/sci,
										/pgf/number format/sci zerofill,
									}
									]
									\addplot[only marks, mark=o, mark size=2.0,color=blue] %
										table[x=time,y=error,col
										sep=comma]{./data/periodic-task/schedule-all-monotonic/xen-vm8/schedule-all-monotonic.log};
									\addlegendentry{Partisi 8};
								\end{axis}
							\end{tikzpicture}
						}
					\end{tabular}
					\caption{Hasil pengujian \textit{latency} sistem menggunakan skenario 3 (layanan 8)}
					\label{figure:hasil_latency_s3_service8}
				\end{figure}

			\item Pada layanan 8, partisi berjalan dengan sangat stabil. Namun, galat maksimal
				pengukuran \textit{timing} dengan \textit{timing} ekspektasi cukup tinggi yaitu
				sekitar \SI{80}{\micro\second}.


				\begin{figure}[!ht]
					\centering
					\begin{tabular}{c}
						\subfloat{
							\tikzsetnextfilename{latency-s3-delta-p9}
							\begin{tikzpicture}
								\begin{axis}[
									height=4cm, width=9cm,
									title={Timing Plot},
									scaled ticks=false,
									xlabel={Waktu (s)},
									ylabel={Delta (s)},
									grid=major,
									xlabel near ticks,
									ylabel near ticks,
									yticklabel style={
										/pgf/number format/precision=2,
										/pgf/number format/sci,
										/pgf/number format/sci zerofill,
									}
									]
									\addplot[only marks, mark=o, mark size=2.0,color=blue] %
										table[x=time,y=delta,col
										sep=comma]{./data/periodic-task/schedule-all-monotonic/xen-vm9/schedule-all-monotonic.log};
									\addlegendentry{Partisi 9};
								\end{axis}
							\end{tikzpicture}
						} \\
						\subfloat{
							\tikzsetnextfilename{latency-s3-error-p9}
							\begin{tikzpicture}
								\begin{axis}[
									height=4cm, width=9cm,
									title={Timing Plot},
									scaled ticks=false,
									xlabel={Waktu (s)},
									ylabel={Galat (s)},
									grid=major,
									xlabel near ticks,
									ylabel near ticks,
									yticklabel style={
										/pgf/number format/precision=2,
										/pgf/number format/sci,
										/pgf/number format/sci zerofill,
									}
									]
									\addplot[only marks, mark=o, mark size=2.0,color=blue] %
										table[x=time,y=error,col
										sep=comma]{./data/periodic-task/schedule-all-monotonic/xen-vm9/schedule-all-monotonic.log};
									\addlegendentry{Partisi 9};
								\end{axis}
							\end{tikzpicture}
						}
					\end{tabular}
					\caption{Hasil pengujian \textit{latency} sistem menggunakan skenario 3 (layanan 9)}
					\label{figure:hasil_latency_s3_service9}
				\end{figure}

			\item Pada layanan 9, galat maksimal pengukuran \textit{timing} dengan \textit{timing}
				ekspektasi tidak terlalu tinggi yaitu sekitar \SI{100}{\micro\second}. Meski
				demikian, \textit{timing} yang didapat dari layanan tersebut tidak stabil.

		\end{enumerate}

\end{enumerate}

\subsection{Pembahasan Hasil Pengujian}
\label{section:pembahasan_pengujian}

\textit{Primary-backup partition scheduler} yang dihasilkan dapat menjadwalkan partisi pengganti
apabila partisi \textit{primary} mengalami kegagalan. \textit{Scheduler} juga dapat menjadwalkan
partisi pengganti apabila partisi \textit{backup} juga mengalami kegagalan, seperti yang dapat
dilihat pada \autoref{figure:hasil_keandalan_s6_service2}.

\textit{Primary-backup partition scheduler} telah menyelesaikan permasalahan keandalan sistem.
Namun, berdasarkan hasil pengujian \textit{latency}, dapat disimpulkan bahwa sistem hasil
implementasi bukan merupakan sistem \textit{real-time} dikarenakan \textit{latency} maksimal
yang sangat tinggi (\textit{latency} dari \textit{timing} hampir mencapai 100\%) atau
\textit{timing} didapat jauh lebih cepat dari waktu ekspektasi \textit{timing} tersebut
didapatkan.

Untuk mencari tahu penyebab \textit{timing} yang didapat pada waktu yang tidak seharusnya, akan
dilakukan pengujian \textit{time drift} untuk mengecek apakah terdapat \textit{time drift}
ketika \textit{task} periodik berjalan. Pengukuran \textit{time drift} dapat dilakukan dengan
menggunakan rumus pada \autoref{eq:drift}.
\begin{equation}\label{eq:drift}
	drift = now - (start + samples * expected)
\end{equation}
Pada perhitungan $drift$, nilai $now$ adalah waktu pada saat \textit{timing} sekarang
didapatkan, nilai $start$ adalah waktu pada saat pengujian dimulai, nilai $samples$ adalah
jumlah sampel yang sudah terkumpul, dan $nilai$ expected adalah nilai ekspektasi. Nilai
ekspektasi yang digunakan dalam perhitungan $drift$ adalah periode dari \textit{timing}. Nilai
$drift$ akan menunjukkan seberapa jauh penyimpangan waktu \textit{timing} dengan waktu
\textit{timing} yang seharusnya.

Secara intuitif, terlihat bahwa kesalahan \textit{timing} terjadi apabila terjadi pergantian
partisi, termasuk pergantian partisi pada saat pergantian layanan. Oleh karena itu, pengujian
\textit{time drift} akan dilakukan dengan melakukan pengujian pada sistem ketika memiliki satu
buah partisi yang akan dijadwalkan dan sistem ketika memiliki lebih dari satu buah partisi yang
akan dijadwalkan. Hasil pengujian \textit{time drift} pada sistem ketika memiliki satu buah
partisi yang akan dijadwalkan dapat dilihat pada \autoref{figure:hasil_timedrift_s1}, sedangkan
hasil pengujian \textit{time drift} pada sistem ketika memiliki lebih dari satu buah partisi
yang akan dijadwalkan dapat dilihat pada \autoref{figure:hasil_timedrift_s2_p1},
\autoref{figure:hasil_timedrift_s2_p2}, \autoref{figure:hasil_timedrift_s2_p3}, dan
\autoref{figure:hasil_timedrift_s2_p4}. 

\begin{figure}[!ht]
	\centering
	\begin{tabular}{c}
		\tikzsetnextfilename{timedrift-s1}
		\begin{tikzpicture}
			\begin{axis}[
				height=4cm, width=9cm,
				title={Timing Plot},
				scaled ticks=false,
				xlabel={Waktu (s)},
				ylabel={Drift (s)},
				grid=major,
				xlabel near ticks,
				ylabel near ticks,
				yticklabel style={
					/pgf/number format/precision=3,
					/pgf/number format/sci,
					/pgf/number format/sci zerofill,
				}
				]
				\addplot[only marks, mark=o, mark size=2.0,color=blue] %
					table[x=time,y=drift,col
					sep=comma]{./data/periodic-task/schedule-domU-monotonic/schedule-domU-monotonic.log};
				\addlegendentry{Partisi 1};
			\end{axis}
		\end{tikzpicture}
	\end{tabular}
	\caption{Hasil pengujian \textit{time drift} pada sistem dengan satu buah partisi yang
	harus dijadwalkan}
	\label{figure:hasil_timedrift_s1}
\end{figure}

\begin{figure}[!ht]
	\centering
	\begin{tabular}{c}
		\tikzsetnextfilename{timedrift-s2-p1}
		\begin{tikzpicture}
			\begin{axis}[
				height=4cm, width=9cm,
				title={Timing Plot},
				scaled ticks=false,
				xlabel={Waktu (s)},
				ylabel={Drift (s)},
				grid=major,
				xlabel near ticks,
				ylabel near ticks,
				yticklabel style={
					/pgf/number format/precision=3,
					/pgf/number format/sci,
					/pgf/number format/sci zerofill,
				}
				]
				\addplot[only marks, mark=o, mark size=2.0,color=blue] %
					table[x=time,y=drift,col
					sep=comma]{./data/periodic-task/schedule-4p-4b/xen-vm1/schedule-4p-4b.log};
				\addlegendentry{Partisi 1};
			\end{axis}
		\end{tikzpicture}
	\end{tabular}
	\caption{Hasil pengujian \textit{time drift} pada sistem dengan lebih dari satu buah
	partisi yang harus dijadwalkan (layanan 1)}
	\label{figure:hasil_timedrift_s2_p1}
\end{figure}

\begin{figure}[!ht]
	\centering
	\begin{tabular}{c}
		\tikzsetnextfilename{timedrift-s2-p2}
		\begin{tikzpicture}
			\begin{axis}[
				height=4cm, width=9cm,
				title={Timing Plot},
				scaled ticks=false,
				xlabel={Waktu (s)},
				ylabel={Drift (s)},
				grid=major,
				xlabel near ticks,
				ylabel near ticks,
				yticklabel style={
					/pgf/number format/precision=3,
					/pgf/number format/sci,
					/pgf/number format/sci zerofill,
				}
				]
				\addplot[only marks, mark=o, mark size=2.0,color=blue] %
					table[x=time,y=drift,col
					sep=comma]{./data/periodic-task/schedule-4p-4b/xen-vm2/schedule-4p-4b.log};
				\addlegendentry{Partisi 2};
			\end{axis}
		\end{tikzpicture}
	\end{tabular}
	\caption{Hasil pengujian \textit{time drift} pada sistem dengan lebih dari satu buah partisi yang
	harus dijadwalkan (layanan 2)}
	\label{figure:hasil_timedrift_s2_p2}
\end{figure}

\begin{figure}[!ht]
	\centering
	\begin{tabular}{c}
		\tikzsetnextfilename{timedrift-s2-p3}
		\begin{tikzpicture}
			\begin{axis}[
				height=4cm, width=9cm,
				title={Timing Plot},
				scaled ticks=false,
				xlabel={Waktu (s)},
				ylabel={Drift (s)},
				grid=major,
				xlabel near ticks,
				ylabel near ticks,
				yticklabel style={
					/pgf/number format/precision=3,
					/pgf/number format/sci,
					/pgf/number format/sci zerofill,
				}
				]
				\addplot[only marks, mark=o, mark size=2.0,color=blue] %
					table[x=time,y=drift,col
					sep=comma]{./data/periodic-task/schedule-4p-4b/xen-vm3/schedule-4p-4b.log};
				\addlegendentry{Partisi 3};
			\end{axis}
		\end{tikzpicture}
	\end{tabular}
	\caption{Hasil pengujian \textit{time drift} pada sistem dengan lebih dari satu buah partisi yang
	harus dijadwalkan (layanan 3)}
	\label{figure:hasil_timedrift_s2_p3}
\end{figure}

\begin{figure}[!ht]
	\centering
	\begin{tabular}{c}
		\tikzsetnextfilename{timedrift-s2-p4}
		\begin{tikzpicture}
			\begin{axis}[
				height=4cm, width=9cm,
				title={Timing Plot},
				scaled ticks=false,
				xlabel={Waktu (s)},
				ylabel={Drift (s)},
				grid=major,
				xlabel near ticks,
				ylabel near ticks,
				yticklabel style={
					/pgf/number format/precision=3,
					/pgf/number format/sci,
					/pgf/number format/sci zerofill,
				}
				]
				\addplot[only marks, mark=o, mark size=2.0,color=blue] %
					table[x=time,y=drift,col
					sep=comma]{./data/periodic-task/schedule-4p-4b/xen-vm4/schedule-4p-4b.log};
				\addlegendentry{Partisi 4};
			\end{axis}
		\end{tikzpicture}
	\end{tabular}
	\caption{Hasil pengujian \textit{time drift} pada sistem dengan lebih dari satu buah partisi yang
	harus dijadwalkan (layanan 4)}
	\label{figure:hasil_timedrift_s2_p4}
\end{figure}

Hasil pengujian tersebut menunjukkan bahwa apabila terdapat dua atau lebih partisi yang
berbeda, maka akan terdapat \textit{time drift} pada waktu \textit{task} periodik akan
dijalankan. Selain itu, hasil pengujian tersebut juga menunjukkan bahwa terdapat titik puncak
maksimum untuk nilai \textit{time drift} sebelum akhirnya \textit{time drift} kembali menjadi
$0$. Proses kembalinya nilai \textit{time drift} menjadi $0$ hanya dapat terjadi apabila sampel
yang didapat sebelum nilai \textit{time drift} kembali menjadi $0$ didapat dengan sangat cepat,
sehingga jumlah sampel bertambah tanpa terdapat perubahan yang signifikan pada nilai waktu
sekarang. Dengan demikian, dapat disimpulkan bahwa \textit{timing} yang didapat terlalu cepat
pada pengujian keandalan dan pengujian \textit{latency} diakibatkan karena adanya \textit{time
drift} pada \textit{periodic task}.  Selain itu, \textit{timing} dengan \textit{latency}
terjadi dikarenakan \textit{time drift} tersebut, yang mengakibatkan \textit{task} periodik
tidak mendapatkan kesempatan jalan dikarenakan waktu pada saat \textit{task} tersebut
mendapatkan giliran di luar dari rentang waktu layanan tersebut.

\textit{Timekeeping} pada Xen merupakan permasalahan laten pada Xen \citep{Adamczyk2015}
\citep{Broomhead2010}. Hasil pengujian \textit{time drift} menunjukkan bahwa kesalahan mulai
terjadi apabila Xen perlu melakukan \textit{partition switching}. Dalam kasus ini, terdapat
beberapa kemungkinan permasalahan yang terjadi. Berikut adalah daftar kemungkinan penyebab
terjadinya \textit{time drift} pada saat pengujian.
\begin{enumerate}

	\item Perbedaan mekanisme \textit{time keeping} pada Linux dan Xen.

		Linux memiliki metode yang berbeda dengan Xen untuk melakukan \textit{time
		keeping}. Mekanisme tersebut ditunjang oleh Xen dengan menggunakan
		\textit{virtual timer}. Meski demikian, sumber waktu yang digunakan oleh Xen dan
		Linux merupakan sumber yang berbeda. Perbedaan metode kalibrasi waktu antara Xen
		dengan Linux dapat mengakibatkan perbedaan waktu sehingga menghasilkan
		\textit{time drift}. Dengan demikian, pada saat partisi dengan sistem operasi
		Linux tidak mendapatkan waktu CPU, kalibrasi waktu CPU yang dilakukan oleh Linux
		tidak sesuai dengan yang seharusnya.

	\item Kinerja \textit{partition switching} pada Xen

		Pada saat \textit{partition switching}, Xen akan menyimpan keadaan sebuah
		partisi sebelum melakukan pergantian. Pada saat Xen akan kembali menjalankan
		partisi tersebut, Xen perlu mengembalikan keadaan partisi seperti sebelumnya.
		Hal ini akan mengakibatkan waktu kerja tambahan sehingga pada saat konteks
		kembali kepada aplikasi, aplikasi sudah melewatkan waktu ketika \textit{task}
		periodik seharusnya berjalan.

	\item Desain yang tidak sesuai

		Xen dan Linux tidak dibangun untuk menunjang aplikasi \textit{real-time}. Dengan
		demikian, mekanisme internal Xen dan Linux dibangun untuk mendukung aplikasi
		secara umum. Aplikasi secara umum memilih \textit{throughput} yang tinggi
		ketimbang \textit{latency} yang rendah. Dengan demikian, mungkin terdapat
		ketidaksesuaian antara kebutuhan \textit{real-time} aplikasi dengan kapabilitas
		sistem.

\end{enumerate}
Sayangnya, mekanisme internal Xen dan Linux bukan merupakan lingkup penelitian Tugas Akhir ini,
sehingga hasil pengujian tidak dapat digunakan untuk menyimpulkan penyebab permasalahan
tersebut.

Untuk mengatasi permasalahan tersebut, pembuat aplikasi \textit{real-time} dapat mendefinisikan
keadaan sistem secara global yang akan diacu oleh setiap aplikasi. Setiap aplikasi akan
menjalankan fungsionalitas tersebut apabila sudah saatnya fungsionalitas tersebut dijalankan.
Periode dari aplikasi secara tidak langsung akan dicapai apabila daftar jadwal yang diberikan
pada \textit{scheduler} sesuai dengan batasan waktu dan periode yang seharusnya.

% vim: tw=96
