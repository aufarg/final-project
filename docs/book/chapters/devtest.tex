%! TEX root = ../final-project.tex
\chapter{PENGEMBANGAN DAN PENGUJIAN}

\section{Pengembangan \textit{Primary-Backup Scheduling}}

Pengembangan dilakukan dengan memodifikasi kode \textit{scheduler} ARINC 653
\textit{source-tree} pada Xen ARINC 653.  Modifikasi kode dilakukan pada modul-modul berikut:

\begin{itemize}

	\item Modul Libxl

	Modul ini digunakan untuk membuat \textit{tools} yang dapat digunakan pada \textit{user
	space} \textit{domain} \code{dom0} untuk melakukan pengaturan \textit{hypervisor}.

	\item Modul Libxc

	Modul ini berisi fungsi-fungsi untuk menangani permintaan \textit{hypercall} dari
	\textit{domain}.

	\item Modul \textit{Public hypercall}

	Modul ini berisi definisi \textit{interface} yang digunakan untuk mengatur struktur data
	yang dikirimkan pada saat mengirim data dari \textit{domain} kepada \textit{hypervisor}.

	\item Modul \textit{Scheduler} ARINC 653

	Modul ini berisi fungsi-fungsi yang digunakan untuk mengisi definisi \textit{interface}
	\textit{scheduler} pada Xen. Fungsi-fungsi tersebut meliputi algoritma
	\textit{scheduler}, penanganan \textit{hypercall} terhadap \textit{scheduler}.

\end{itemize}

Agar algoritma \textit{primary-backup scheduling} yang dijelaskan pada
\autoref{section:solution} dapat berjalan dengan seharusnya, perlu ada mekanisme penanganan data
yang dibutuhkan pada masing-masing domain. Hal tersebut dapat dicapai dengan mendefinisikan
struktur data pada mekanisme pengiriman data dari \textit{user space} kepada
\textit{hypervisor}, pendefinisian struktur data yang dimengerti oleh \textit{hypervisor} dan
\textit{user space} untuk digunakan pada saat melakukan \textit{hypercall}, mekanisme pengiriman
data melalui \textit{hypercall}, struktur data untuk menyimpan informasi \textit{domain} pada
\textit{scheduler}, serta konversi data yang didapat oleh \textit{hypervisor} setelah menangani
\textit{hypercall} menjadi dengan struktur data \textit{domain} pada \textit{scheduler}.

Implementasi mekanisme pengiriman data dari \textit{user space} kepada \textit{hypervisor} akan
dipaparkan pada \autoref{subsection:modul_libxl}. Mekanisme \textit{hypercall} yang
diimplementasikan akan dipaparkan pada \autoref{subsection:modul_libxc}. Pendefinisian struktur
data yang digunakan pada saat melakukan \textit{hypercall} akan dipaparkan pada
\autoref{subsection:modul_public_hypercall}. Mekanisme konversi data yang didapat oleh
\textit{hypervisor} setelah menangani \textit{hypercall} serta definisi struktur data
masing-masing \textit{domain} pada \textit{scheduler} akan dijelaskan pada
\autoref{subsection:modul_scheduler_arinc653}.

Penjelasan akan dilakukan secara \textit{bottom-up} dimulai dari definisi
\textit{interface}/struktur data, kemudian fungsi dari \textit{interface}/struktur data yang
sudah didefinisikan. Masing-masing mekanisme akan dijelaskan apabila seluruh
\textit{interface}/struktur data yang digunakan pada mekanisme tersebut telah dipaparkan. Secara
garis besar, penjelasan akan dilakukan dimulai dari bagian \textit{low-level} dan
\textit{high-level} pada \textit{hypervisor}, kemudian memasuki bagian \textit{low-level} dan
\textit{high-level} pada \textit{user space}. Urutan penjelasan dilakukan sedemikian dengan
harapan akan mempermudah pembaca dalam mengerti masing-masing langkah yang akan dipaparkan pada
setiap mekanisme.

\subsection{Modul \textit{scheduler} ARINC 653}
\label{subsection:modul_scheduler_arinc653}

Subbab ini akan membahas implementasi dan modifikasi pada komponen \textit{scheduler} agar dapat
mengimplementasikan \textit{primary-backup scheduling} pada \textit{scheduler} ARINC 653.

\subsubsection{Pendefinisian \textit{scheduler}}

Pada Xen, pendefinisian \textit{scheduler} dilakukan dengan menggunakan \textit{interface} yang
telah disediakan~\citep[p.~218]{Chisnall2014}. \autoref{code:struct_scheduler} menampilkan
\textit{interface} tersebut. \textit{Interface} tersebut kemudiakan didefinisikan sebagai
struktur pada saat \textit{runtime} dengan masing-masing \textit{function pointer} menunjuk pada
fungsi yang diinginkan. Setiap fungsi yang ditunjuk oleh \textit{function pointer} pada struktur
tersebut akan berlaku sebagai \textit{handler} yang akan dipanggil oleh \textit{hypervisor}
apabila terjadi \textit{event} yang bersesuaian. Asosiasi antara \textit{handler} dan
\textit{event} telah ditentukan sebelumnya. Setiap \textit{handler} pada struktur tersebut
kemudian diisi dengan alamat dari fungsi yang nantinya akan menangani permintaan pada
\textit{scheduler}. Setiap \textit{scheduler} diharuskan mengisi \textit{handler}
\code{do\_schedule()}. Apabila \textit{scheduler} tidak memerlukan sebuah \textit{handler}, maka
\textit{handler} dapat diisi dengan nilai \code{NULL}.

Pada saat \textit{boot}, konfigurasi \textit{scheduler} yang diinginkan dapat dipilih dengan
menambahkan argumen "sched=\{scheduler\}". \textit{Hypervisor} kemudian akan mencari
\textit{scheduler} didefinisikan dengan nilai \code{opt\_name} sama dengan "scheduler".

\begin{lstlisting}[
	caption={\textit{Interface} dari fungsi \textit{scheduler} yang dimodifikasi/diimplementasi},
	label=code:struct_scheduler
]
struct scheduler {
    char ∗name ;
    char ∗opt name ;
    unsigned int sched id ;

    void         (*free_domdata)   (const struct scheduler *, void *);
    void *       (*alloc_domdata)  (const struct scheduler *, struct domain *);
    int          (*init_domain)    (const struct scheduler *, struct domain *);
    void         (*destroy_domain) (const struct scheduler *, struct domain *);

    struct task_slice (*do_schedule) (const struct scheduler *, s_time_t,

    int          (*adjust)         (const struct scheduler *, struct domain *,
    int          (*adjust_global)  (const struct scheduler *,
};
\end{lstlisting}

Untuk mengimplementasikan \textit{scheduler}, fungsi \code{do\_schedule()} harus diisi dengan
alamat dari fungsi yang akan melakukan pemilihan partisi. Pada implementasi
\textit{primary-backup scheduler}, fungsi \code{do\_schedule()} akan diisi dengan alamat dari
fungsi \code{a653sched\_do\_schedule()} dan \code{opt\_name} akan diberi nilai "arinc653pb".
Definisi fungsi \code{a653sched\_do\_schedule()} dapat dilihat pada
\autoref{appendix:a653sched_do_schedule}.

\subsubsection{Informasi Partisi}

\textit{Scheduler} harus mengetahui beberapa informasi terkait keadaan partisi agar dapat
menentukan partisi mana yang harus mendapatkan waktu CPU.  \textit{Scheduler} harus mengetahui
apakah partisi tersebut mengalami \textit{failure} atau tidak. Apabila partisi sedang mengalami
kegagalan, maka partisi tidak boleh mendapatkan jatah waktu CPU. Selain itu, \textit{scheduler}
juga harus mengetahui \textit{service} apa saja yang sudah dikerjakan sebelumnya pada
\textit{major time frame} saat ini. Dengan demikian, partisi yang memiliki \textit{service} yang
tidak bersifat \textit{idempotent} tidak akan mendapatkan waktu CPU lebih dari sekali dan
partisi yang memiliki \textit{service} yang bersifat \textit{idempotent} tidak akan membuang
waktu CPU. Kedua hal tersebut dapat dilakukan dengan menyimpan informasi berupa keadaan partisi
dan identifikasi \textit{service} yang berada pada partisi tersebut.

\begin{lstlisting}[
	caption={\textit{Interface} dari fungsi \textit{scheduler} yang dimodifikasi/diimplementasi},
	label=code:struct_a653sched_domain_t
]
typedef struct a653sched_domain_s {
    domid_t parent;
    bool primary;
    bool healthy;
} a653sched_domain_t;
\end{lstlisting}

Informasi mengenai keadaan partisi dapat disimpan dengan menggunakan \textit{boolean}. Informasi
mengenai identifikasi \textit{service} yang terdapat pada partisi dapat disimpan menggunakan
identifikasi partisi, dengan asumsi \textit{service} yang terdapat hubungan satu ke satu antara
service dengan partisi sehingga identifikasi partisi dapat digunakan untuk mengidentifikasi
\textit{service}.

Pada implementasi \textit{primary-backup scheduler}, kedua informasi tersebut disimpan pada
struktur data seperti yang ditampilkan oleh \autoref{code:struct_a653sched_domain_t}.
\textit{Scheduler} mengetahui keadaan partisi apabila partisi tersebut memberikan informasi
tentang dirinya melalui \textit{hypercall} yang didefinisikan pada
\autoref{subsection:modul_public_hypercall}.

\subsection{Modul \textit{public hypercall}}
\label{subsection:modul_public_hypercall}

\subsection{Modul libxc}
\label{subsection:modul_libxc}

\subsection{Modul libxl}
\label{subsection:modul_libxl}

\section{Pengujian}



% vim: tw=96
