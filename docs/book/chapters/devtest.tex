\chapter{PENGEMBANGAN DAN PENGUJIAN}

\section{Pengembangan \textit{Primary-Backup Scheduling}}

Pengembangan dilakukan dengan memodifikasi kode \textit{scheduler} ARINC 653 \textit{source-tree} pada Xen ARINC 653.
Modifikasi kode dilakukan pada modul-modul berikut:

\begin{itemize}

	\item Libxl

	Modul ini digunakan untuk membuat \textit{tools} yang dapat digunakan pada \textit{user space} \textit{domain} \code{dom0} untuk melakukan pengaturan \textit{hypervisor}.

	\item Libxc

	Modul ini berisi fungsi-fungsi untuk menangani permintaan \textit{hypercall} dari \textit{domain}.

	\item \textit{Public hypercall}

	Modul ini berisi definisi \textit{interface} yang digunakan untuk mengatur struktur data yang dikirimkan pada saat mengirim data dari \textit{domain} kepada \textit{hypervisor}.

	\item \textit{Scheduler} ARINC 653

	Modul ini berisi fungsi-fungsi yang digunakan untuk mengisi definisi \textit{interface} \textit{scheduler} pada Xen. Fungsi-fungsi tersebut meliputi algoritma \textit{scheduler}, penanganan \textit{hypercall} terhadap \textit{scheduler}.

\end{itemize}

Agar algoritma \textit{primary-backup scheduling} yang dijelaskan pada \autoref{section:solution} dapat berjalan dengan seharusnya, perlu ada mekanisme penanganan data yang dibutuhkan pada masing-masing domain. Hal tersebut dapat dicapai dengan mendefinisikan struktur data pada mekanisme pengiriman data dari \textit{user space} kepada \textit{hypervisor}, pendefinisian struktur data yang dimengerti oleh \textit{hypervisor} dan \textit{user space} untuk digunakan pada saat melakukan \textit{hypercall}, mekanisme pengiriman data melalui \textit{hypercall}, struktur data untuk menyimpan informasi \textit{domain} pada \textit{scheduler}, serta konversi data yang didapat oleh \textit{hypervisor} setelah menangani \textit{hypercall} menjadi dengan struktur data \textit{domain} pada \textit{scheduler}.

Implementasi mekanisme pengiriman data dari \textit{user space} kepada \textit{hypervisor} akan dipaparkan pada \autoref{subsection:modul_libxl}. Mekanisme \textit{hypercall} yang diimplementasikan akan dipaparkan pada \autoref{subsection:modul_libxc}. Pendefinisian struktur data yang digunakan pada saat melakukan \textit{hypercall} akan dipaparkan pada \autoref{subsection:modul_public_hypercall}. Mekanisme konversi data yang didapat oleh \textit{hypervisor} setelah menangani \textit{hypercall} serta definisi struktur data masing-masing \textit{domain} pada \textit{scheduler} akan dijelaskan pada \autoref{subsection:modul_scheduler_arinc653}.

Penjelasan akan dilakukan secara \textit{bottom-up} dimulai dari definisi \textit{interface}/struktur data, kemudian fungsi dari \textit{interface}/struktur data yang sudah didefinisikan. Masing-masing mekanisme akan dijelaskan apabila seluruh \textit{interface}/struktur data yang digunakan pada mekanisme tersebut telah dipaparkan. Secara garis besar, penjelasan akan dilakukan dimulai dari bagian \textit{low-level} dan \textit{high-level} pada \textit{hypervisor}, kemudian memasuki bagian \textit{low-level} dan \textit{high-level} pada \textit{user space}. Urutan penjelasan dilakukan sedemikian dengan harapan akan mempermudah pembaca dalam mengerti masing-masing langkah yang akan dipaparkan pada setiap mekanisme.

\subsection{Modul \textit{scheduler} ARINC 653}
\label{subsection:modul_scheduler_arinc653}

Pada Xen, pendefinisian \textit{scheduler} dilakukan dengan membuat \textit{interface} \textit{scheduler}.
\textit{Interface} dibuat berdasarkan struktur yang telah terdefinisi pada \filename{xen/include/xen/sched-if.h}.

\begin{lstlisting}[caption={\textit{Interface} dari fungsi \textit{scheduler} yang dimodifikasi/diimplementasi}]
struct scheduler {
    void         (*free_domdata)   (const struct scheduler *, void *);
    void *       (*alloc_domdata)  (const struct scheduler *, struct domain *);
    int          (*init_domain)    (const struct scheduler *, struct domain *);
    void         (*destroy_domain) (const struct scheduler *, struct domain *);

    struct task_slice (*do_schedule) (const struct scheduler *, s_time_t,

    int          (*adjust)         (const struct scheduler *, struct domain *,
    int          (*adjust_global)  (const struct scheduler *,
};
\end{lstlisting}

Setiap fungsi pada struktur tersebut akan berlaku sebagai \textit{handler} yang akan dipanggil oleh \textit{hypervisor} apabila terjadi \textit{event} yang bersesuaian.
Asosiasi antara \textit{handler} dan \textit{event} telah ditentukan sebelumnya dan dapat dilihat pada \filename{xen/common/schedule.c}.
Setiap \textit{handler} pada struktur tersebut kemudian diisi dengan alamat dari fungsi yang nantinya akan menangani permintaan pada \textit{scheduler}. Setiap \textit{scheduler} diharuskan mengisi \textit{handler} \code{do\_schedule}.
Apabila \textit{scheduler} tidak memerlukan sebuah \textit{handler}, maka \textit{handler} dapat diisi dengan nilai \code{NULL}.

Untuk mengimplementasikan \textit{scheduler}, fungsi \code{do\_schedule} harus diisi dengan alamat dari fungsi yang akan melakukan \textit{scheduling}.
Pada implementasi \textit{primary-backup scheduler}, fungsi \code{do\_schedule} akan diisi dengan alamat dari fungsi \code{a653sched\_do\_schedule}.
Definisi fungsi \code{a653sched\_do\_schedule} dapat dilihat pada \autoref{appendix:a653sched_do_schedule}.

\subsection{Modul \textit{public hypercall}}
\label{subsection:modul_public_hypercall}

\subsection{Modul libxc}
\label{subsection:modul_libxc}

\subsection{Modul libxl}
\label{subsection:modul_libxl}

\section{Pengujian}

% vim: wrap
