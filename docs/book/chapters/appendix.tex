%! TEX root = ../final-project.tex

\begin{appendices}
\chapter{Fungsi \textit{Scheduling} pada ARLX}
\label{appendix:a653sched_do_schedule}

\begin{lstlisting}[
	linerange={1-15,25-31,34-53}
]
static struct task_slice
a653sched_do_schedule(
    const struct scheduler *ops,
    s_time_t now,
    bool_t tasklet_work_scheduled)
{
    struct task_slice ret;                      /* hold the chosen domain */
    struct vcpu * new_task = NULL;
    static unsigned int sched_index = 0;
    static s_time_t next_switch_time;
    a653sched_priv_t *sched_priv = SCHED_PRIV(ops);
    const unsigned int cpu = smp_processor_id();
    static bool task_done[ARINC653_MAX_DOMAINS_PER_SCHEDULE] = {0};
    static a653sched_domain_t * dom_priv = NULL;
    unsigned long flags;

    spin_lock_irqsave(&sched_priv->lock, flags);

    if ( sched_priv->num_schedule_entries < 1 )
        sched_priv->next_major_frame = now + DEFAULT_TIMESLICE;
    else if ( now >= sched_priv->next_major_frame )
    {
        /* time to enter a new major frame
         * the first time this function is called, this will be true */
        /* start with the first domain in the schedule */
        sched_index = 0;
        sched_priv->next_major_frame = now + sched_priv->major_frame;
        next_switch_time = now + sched_priv->schedule[0].runtime;
        dom_priv = DOM_PRIV(sched_priv->schedule[0].vc->domain);
        memset(task_done, 0, sizeof(task_done));
    }
    else
    {
        /* When switching domain, the dom_priv should point to current domain private data structure */
        BUG_ON(dom_priv == NULL);

        /* if last task elapsed, mark task  as done */
        if ( now >= next_switch_time) 
            task_done[dom_priv->parent] = true;

        while ( ( (now >= next_switch_time)
                || (!dom_priv->healthy)
                || (task_done[dom_priv->parent]) )
                && (sched_index < sched_priv->num_schedule_entries) )
        {
            /* time to switch to the next domain in this major frame */
            sched_index++;
            if (sched_index < sched_priv->num_schedule_entries) {
                dom_priv = DOM_PRIV(sched_priv->schedule[sched_index].vc->domain);
                if (!task_done[dom_priv->parent])
                    next_switch_time += sched_priv->schedule[sched_index].runtime;
            }
        }
    }

    /*
     * If we exhausted the domains in the schedule and still have time left
     * in the major frame then switch next at the next major frame.
     */
    if ( sched_index >= sched_priv->num_schedule_entries )
        next_switch_time = sched_priv->next_major_frame;

    /*
     * If there are more domains to run in the current major frame, set
     * new_task equal to the address of next domain's VCPU structure.
     * Otherwise, set new_task equal to the address of the idle task's VCPU
     * structure.
     */
    new_task = (sched_index < sched_priv->num_schedule_entries)
        ? sched_priv->schedule[sched_index].vc
        : IDLETASK(cpu);

    /* Check to see if the new task can be run (awake & runnable). */
    if ( !((new_task != NULL)
           && (AVCPU(new_task) != NULL)
           && AVCPU(new_task)->awake
           && vcpu_runnable(new_task)) )
        new_task = IDLETASK(cpu);
    BUG_ON(new_task == NULL);

    /*
     * Check to make sure we did not miss a major frame.
     * This is a good test for robust partitioning.
     */
    BUG_ON(now >= sched_priv->next_major_frame);

    spin_unlock_irqrestore(&sched_priv->lock, flags);

    /* Tasklet work (which runs in idle VCPU context) overrides all else. */
    if ( tasklet_work_scheduled )
        new_task = IDLETASK(cpu);

    /* Running this task would result in a migration */
    if ( !is_idle_vcpu(new_task)
         && (new_task->processor != cpu) )
        new_task = IDLETASK(cpu);

    /*
     * Return the amount of time the next domain has to run and the address
     * of the selected task's VCPU structure.
     */
    ret.time = next_switch_time - now;
    ret.task = new_task;
    ret.migrated = 0;

    BUG_ON(ret.time <= 0);

    return ret;
}
\end{lstlisting}
\end{appendices}
