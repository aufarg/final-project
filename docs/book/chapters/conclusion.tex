 % TEX root = ../final-project.tex
\chapter{Kesimpulan dan Saran}

\section{Kesimpulan}

Pada penelitian ini, telah berhasil dikembangkan sebuah \textit{scheduler} yang menggunakan
skema \textit{primary-backup} guna meningkatkan keandalan sistem berbasis ARINC 653. Oleh karena
\textit{scheduler} menggunakan skema \textit{primary-bacckup} dan hanya bekerja untuk melakukan
penjadwalan partisi, maka \textit{scheduler} tersebut disebut sebagai \textit{primary-backup
partition scheduler}. \textit{Scheduler} bekerja dengan cara memberikan waktu CPU pada partisi
apabila partisi tersebut dapat menyediakan layanan yang seharusnya berjalan pada waktu tersebut
dan tidak sedang mengalami kegagalan.

Untuk dapat memberitahukan keadaan partisi melalui \textit{user space}, \textit{hypercall} untuk
mengubah informasi mengenai keadaan partisi juga sudah diimplementasikan. Dengan demikian,
kegagalan pada tingkat aplikasi juga dapat diberitahukan kepada \textit{scheduler} untuk
kemudian ditindaklanjuti. \textit{Hypercall} dapat dilakukan melalui aplikasi dengan menggunakan
\textit{libxc} maupun secara manual oleh administrator sistem dengan menggunakan \textit{libxl}.

\textit{Primary-backup partition scheduler} dapat menangani kegagalan dengan sangat baik. Hasil
pengujian keandalan menunjukkan bahwa terdapat 100\% \textit{uptime} pada setiap layanan. Hasil
pengujian mungkin menaksir nilai \textit{uptime} terlalu tinggi dikarenakan tidak pernah
terjadinya kegagalan seluruh partisi penyedia setiap layanan. Selain itu, peningkatan keandalan
hanya berlaku untuk kasus dimana \textit{recovery action} yang dispesifikan oleh ARINC 653 tidak
dapat menangani kegagalan yang terjadi. Meski demikian, \textit{primary-backup partition
scheduler} tetap meningkatkan keandalan sistem pada kasus tersebut secara signifikan.

Meski hasil pengujian menunjukkan bahwa \textit{primary-backup partition scheduler} meningkatkan
keandalan sistem, terlihat bahwa terdapat permasalahan pada hasil \textit{timing} yang
didapat pada saat pengujian. Kesalahan \textit{timing} sangat signifikan karena pada suatu saat
sebuah layanan tidak memberikan pengukuran \textit{timing} pada satu siklus \textit{major time
frame}. Hal ini menunjukkan bahwa fungsionalitas aplikasi dapat terlewatkan apabila
fungsionalitas tersebut dilakukan pada selang periode tertentu. Untuk mengatasi permasalahan
tersebut, pembuat aplikasi \textit{real-time} dapat mendefinisikan keadaan sistem yang dapat
diakses secara global untuk memberitahukan kapan sebuah fungsionalitas harus dijalankan
ketimbang menggunakan sistem \textit{timing}.

\section{Saran}

Hasil implementasi \textit{primary-backup partition scheduler} sudah cukup baik. Hal ini
diindikasikan oleh tercapainya peningkatan keandalan sistem ketika menggunakan
\textit{primary-backup partition scheduler}. Namun, sebelum \textit{scheduler} tersebut dapat
digunakan pada lingkungan produksi, terdapat beberapa permasalahan yang harus diselesaikan.
Berikut adalah daftar permasalahan yang harus ditangani.

\begin{enumerate}

	\item \textit{Time drift} pada sistem

		Seperti yang sudah dibahas pada \autoref{section:pembahasan_pengujian}, terdapat
		permasalahan pengukuran waktu pada sistem. Perlu ada studi lebih lanjut mengenai
		penyebab perilaku tersebut.

	\item Kerangka uji yang lebih kuat

		Pengujian yang dilakukan tidak mengikuti sebuah standar tertentu. Untuk menguji
		\textit{scheduler}, metode yang akan menghasilkan hasil yang lebih menggambarkan
		kejadian nyata adalah dengan menggunakan pemodelan dan simulasi. Selain itu,
		penentuan saat kegagalan seharusnya terjadi pada saat pengujian sebaiknya
		disesuaikan dengan \textit{mean-time-failure} dari layanan. Penggunaan metode
		tersebut juga dapat menunjukkan nilai optimal untuk beberapa parameter
		penjadwalan yang sangat penting seperti jumlah partisi yang sebaiknya disediakan
		untuk sebuah layanan dan waktu CPU yang harus diberikan oleh \textit{hypervisor}
		untuk sebuah layanan.

\end{enumerate}

Selain itu, perlu adanya pengembangan beberapa fitur sehingga sistem menjadi layak pakai.
Berikut adalah daftar fitur yang dapat ditambahkan pada sistem.

\begin{enumerate}

	\item Minimalisasi penggunaan memori

		Memiliki partisi \textit{backup} memerlukan memori tambahan linear terhadap
		rata-rata \textit{redundancy} yang dibutuhkan agar sistem tetap berjalan
		sebagaimana seharusnya. Pada umumnya, \textit{mean time to failure} sebuah
		aplikasi akan jauh lebih lambat ketimbang waktu yang dibutuhkan untuk membuat
		partisi baru. Dengan demikian, kita dapat membuat partisi \textit{backup} pada
		saat dibutuhkan saja. Hal ini mengakibatkan \textit{trade-off} antara penggunaan
		memori dan performa sistem secara keseluruhan.

	\item Pendeteksian partisi gagal otomatis

		Untuk memberitahukan pada \textit{scheduler} bahwa sebuah partisi sedang
		mengalami kegagalan atau tidak sudah dapat dilakukan dengan menggunakan
		\textit{hypercall}. Namun, hal ini tidak dapat dilakukan apabila aplikasi yang
		bertugas untuk mendeteksi kegagalan pada partisi tersebut juga mengalami
		kegagalan. Sistem harus dapat mendeteksi kasus tersebut secara otomatis sehingga
		\textit{scheduler} dapat memilih partisi yang tepat untuk menyediakan sebuah
		layanan.

	\item Dukungan untuk \textit{multicore}

		Salah satu permasalahan yang mungkin terjadi adalah kegagalan \textit{core} dari
		sebuah CPU. Dengan memiliki beberapa \textit{core}, maka sistem dapat tetap berjalan
		meski terdapat kegagalan \textit{core}.

\end{enumerate}

% vim: tw=96
