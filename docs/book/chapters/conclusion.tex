%! TEX root = ../final-project.tex
\section{Conclusion}

In this paper, we proposed a way to increases the reliability of ARINC 653 operating systems by
using primary-backup scheme on ARINC 653 partition scheduler. The scheduler works by giving CPU
time to candidate partition only when it is currently healthy and the corresponding service has
not provided yet. In order to let the scheduler know partition status, we define a new hypercall
to modify information associated to a partition. Using primary-backup scheme on ARINC 653
scheduler increases the reliability for each service and thus prolong the mean time to failure
for any service. Since we assume that services reliability is the most unreliable component on the
system, this means the proposed scheduler also increases the whole system reliability.

While the result of the reliability testing is already good, the testing is done without regards
of processes' deadline inside of each partition. We intend to do some improvement on the testing
method so it also considers the deadline of each process inside of each partition. Doing so will
make the result more relevant to the actual use case of the system. We also plan to have
another type of test to measure performance difference caused by the overhead of the scheduler
and analyze the effect of the difference on the system.

As the paper proposed a new scheduler, there are some issues which need to be solved such as:

\begin{itemize}

	\item \textit{Space redundancy minimization}: Having backup partitions require extra
		space linear to the average redundancy rate of primary partitions for the
		system. On the assumption that the mean time to failure for each service to be
		reasonably long, we might be able to reduce system required space in overall at
		the cost of performance on rare events.

	\item \textit{Automatic partition failure detection}: Telling the scheduler whether the
		service on a partition is healthy or not can already be done by means of
		hypercalls. This cannot be done if the environment on said partition does not
		functions properly, which can be caused by transient or inherent problems. The
		hypervisor must be able detect such cases automatically so the scheduler could
		properly select one of the backup partitions.

	\item \textit{Multicore support}: Some of the inherent problems that could happen is
		failure of the CPU core which the partition is supposed to run.
		Having some other cores could help the scheduler schedule the backup partitions
		on different core than the primary to keep the service running.

	\item \textit{Automatic schedule list generator}: It is up to the engineer to provide a
		schedule list which satisfies the deadline and runtime constraints. Since the
		schedule list might need to be adjusted when a partition failure occurs, the
		scheduler list should still satisfies the deadline and runtime constraints on
		any possible partitions state. This proves to be a daunting task for the
		engineer since there will be many cases (up to $2^N$ possible cases, $N$ is the
		number of partitions).

\end{itemize}

% vim: tw=96
