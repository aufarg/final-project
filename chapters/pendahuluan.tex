\chapter{PENDAHULUAN}

\section{Latar Belakang}

Sistem penerbangan atau avionik adalah sistem yang terdiri dari perangkat elektronik yang digunakan pada pesawat terbang
atau satelit. Avionik berfungsi untuk memberikan fasilitas untuk mengoperasikan pesawat terbang. Sistem yang termasuk
dalam avionik adalah sistem komunikasi, navigasi, \textit{dashboard} untuk manajemen sistem, dan sistem-sistem yang
dipasang pada pesawat terbang untuk melakukan fungsionalitas tersendiri.

Avionik merupakan komponen penting dalam sebuah pesawat terbang dan merupakan \textit{safety-critical system}, yaitu
sistem yang apabila mengalami kegagalan dapat mengakibatkan kematian, kerugian besar, atau kerusakan lingkungan. Avionik
pada sebuah pesawat umumnya menggunakan sistem operasi \textit{real-time}, yaitu sistem operasi yang kebenarannya
ditentukan oleh kebenaran dan waktu pengerjaan sebuah operasi. Sistem operasi yang digunakan untuk avionik memiliki
beberapa kriteria yang wajib dipenuhi. Kriteria tersebut distandarkan oleh Aeronautical Radio, Incorporated (ARINC)
dalam sebuah standar spesifikasi penerbangan bernama ARINC 653.

Standar ARINC 653 menspesifikasikan partisi ruang dan waktu beserta \textit{API} pada sistem operasi \textit{real-time} agar avionik yang
menggunakan sistem operasi tersebut \textit{safety-critical}. Pada sistem tersebut, setiap perangkat lunak disebut
sebagai \textbf{partisi} dan memiliki ruang memori tersendiri. Arsitektur tersebut dapat disimulasikan dengan
menggunakan \textit{hypervisor}. \textit{Hypervisor} adalah sebuah komponen yang membuat dan menjalankan \textit{virtual
machine}. Setiap \textit{virtual machine} akan memiliki ruang memori tersendiri sehingga dapat digunakan untuk
mengimplementasikan sebuah \textbf{partisi}.

Pada saat penulisan, sudah banyak \textit{hypervisor} yang mengimplementasikan standar ARINC 653, namun sistem yang
tersedia adalah sistem berbayar maupun sistem berlangganan. Hal ini mengakibatkan eksplorasi pengembangan aplikasi,
\textit{benchmarking}, dan jaminan desain sulit untuk dilakukan tanpa investasi yang besar.

\section{Rumusan Masalah}

Perumusan masalah dari uraian latar belakang adalah tidak adanya implementasi \textit{hypervisor} yang memenuhi kriteria
pada spesifikasi ARINC 653 yang bersifat \textit{free} dan \textit{open source}.

\section{Tujuan}

Berdasarkan rumusan masalah yang ada, tujuan yang ingin dicapai adalah melakukan implementasi \textit{API} yang
yang dispesifikasikan pada standar ARINC 653 pada \textit{hypervisor} yang bersifat \textit{free} dan \textit{open
source}.

\section{Batasan Masalah}

% Text Here %

\section{Metodologi}

% Text Here %

\section{Jadwal Pelaksanaan Tugas Akhir}

% Text Here %
