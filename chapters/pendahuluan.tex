\chapter{PENDAHULUAN}

\section{Latar Belakang}

Sistem penerbangan atau avionik adalah sistem yang terdiri dari perangkat elektronik yang digunakan pada pesawat terbang
atau satelit. Avionik berfungsi untuk memberikan fasilitas untuk mengoperasikan pesawat terbang. Sistem yang termasuk
dalam avionik adalah sistem komunikasi, navigasi, \textit{dashboard} untuk manajemen sistem, dan sistem-sistem yang
dipasang pada pesawat terbang untuk melakukan fungsionalitas tersendiri.

Avionik merupakan komponen penting dalam sebuah pesawat terbang dan merupakan \textit{safety-critical system}, yaitu
sistem yang apabila mengalami kegagalan dapat mengakibatkan kematian, kerugian besar, atau kerusakan lingkungan. Avionik
memiliki beberapa kriteria yang wajib dipenuhi. Kriteria tersebut distandarkan oleh Aeronautical Radio, Incorporated
(ARINC) dalam sebuah standar spesifikasi penerbangan bernama ARINC 653.

Sudah banyak \textit{hypervisor} yang mengimplementasikan standar ARINC 653, namun sistem yang tersedia adalah sistem
berbayar maupun sistem berlangganan. Hal ini mengakibatkan eksplorasi pengembangan aplikasi, \textit{benchmarking}, dan
jaminan desain sulit untuk dilakukan tanpa investasi yang besar.

\section{Rumusan Masalah}

Perumusan masalah dari uraian latar belakang adalah tidak adanya implementasi \textit{hypervisor} yang memenuhi kriteria
pada spesifikasi ARINC 653 yang \textit{free} dan \textit{open source}.

\section{Tujuan}

Berdasarkan rumusan masalah yang ada, tujuan yang ingin dicapai adalah melakukan implementasi \textit{system call} yang
memenuhi kriteria pada spesifikasi ARINC 653.

\section{Batasan Masalah}

% Text Here %

\section{Metodologi}

% Text Here %

\section{Jadwal Pelaksanaan Tugas Akhir}

% Text Here %
